%%% shu-doc.tex --- Shu project documentation
%%
%% Copyright (C) 2018 Stewart L. Palmer
%%
%% Author: Stewart L. Pslmer <stewart@stewartpalmer.com>
%%
%% This file is NOT part of GNU Emacs.
%%
%% This is free software: you can redistribute it and/or modify it
%% under the terms of the GNU General Public License as published by
%% the Free Software Foundation, either version 3 of the License, or
%% (at your option) any later version.
%%
%% This software is distributed in the hope that it will be useful,
%% but WITHOUT ANY WARRANTY% without even the implied warranty of
%% MERCHANTABILITY or FITNESS FOR A PARTICULAR PURPOSE.  See the GNU
%% General Public License for more details.
%%
%% There is a copy of the Gnu General Public license in the file
%% LICENSE in this repository.  You should also have received a copy
%% of the GNU General Public License along with GNU Emacs.  If not,
%% see <http://www.gnu.org/licenses/>.
%%




\section{Overview}


This manual contains detailed information on all of the function, macro,
variable, and constant definitions in this repository.

What is lacks are detailed senarios and work flows.  The reader might well
say that this is an interesting collection of parts, and then go on to ask
how to use it.  How does one use all of these things in a coherent manner?

I hope to address that in the near future.

One thing I will mention is that the function \emph{shu-capture-all-latex} ceated
the entire LaTex users manual (shu-manual.tex) and the function
\emph{shu-capture-all-md} created the entire markdown version of the users manual
(shu-manual.md).


\section{shu-base}


Collection of miscellaneous functions  and constants used by other
packages in this repository.  Most of the elisp files in this repository
depend on this file.


\subsection{List of functions and variables}

List of functions and variable definitions in this package.



\vspace{1em}
\noindent
\savebox{\funcname}{\noindent\texttt{shu-all-whitespace-chars }}
\usebox{\funcname}
\index{shu-all-whitespace-chars} \hfill [Constant]

\begin{doc-string}
List of all whitespace characters.
Since the syntax table considers newline to be (among other things) a
comment terminator, the usual \\s- won't work for whitespace that includes
newlines.
\end{doc-string}

\vspace{1em}
\noindent
\savebox{\funcname}{\noindent\texttt{shu-all-whitespace-regexp }}
\usebox{\funcname}
\index{shu-all-whitespace-regexp} \hfill [Constant]

\begin{doc-string}
Regular expression to search for whitespace.  Since the syntax table considers
newline to be (among other things) a comment terminator, the usual \\s- won't work
for whitespace that includes newlines.  Note that this regular expression is a
character alternative enclosed in left and right brackets.  skip-chars-forward does
not expect character alternatives to be enclosed in square brackets and will include
the left and right brackets in the class of characters to be skipped.
\end{doc-string}

\vspace{1em}
\noindent
\savebox{\funcname}{\noindent\texttt{shu-all-whitespace-regexp-scf }}
\usebox{\funcname}
\index{shu-all-whitespace-regexp-scf} \hfill [Constant]

\begin{doc-string}
This is the regular expression contained in shu-all-whitespace-regexp but with
the enclosing left and right square brackets removed.  skip-chars-forward does
not expect character alternatives to be enclosed in square brackets and thus
will include the brackets as characters to be skipped.
\end{doc-string}

\vspace{1em}
\noindent
\savebox{\funcname}{\noindent\texttt{shu-comment-start-pattern }}
\usebox{\funcname}
\index{shu-comment-start-pattern} \hfill [Constant]

\begin{doc-string}
The regular expression that defines the delimiter used to start
a comment.
\end{doc-string}

\vspace{1em}
\noindent
\savebox{\funcname}{\noindent\texttt{shu-cpp-author }}
\usebox{\funcname}
\index{shu-cpp-author} \hfill [Custom]

\begin{doc-string}
The string to place in the doxygen author tag.
\end{doc-string}

\vspace{1em}
\noindent
\savebox{\funcname}{\noindent\texttt{shu-cpp-comment-end }}
\usebox{\funcname}
\index{shu-cpp-comment-end} \hfill [Custom]

\begin{doc-string}
Standard end point (right hand margin) for C style comments.
\end{doc-string}

\vspace{1em}
\noindent
\savebox{\funcname}{\noindent\texttt{shu-cpp-comment-start }}
\usebox{\funcname}
\index{shu-cpp-comment-start} \hfill [Custom]

\begin{doc-string}
Column in which a standard comment starts.  Any comment that starts to the left of
this point is assumed to be a block comment.
\end{doc-string}

\vspace{1em}
\noindent
\savebox{\funcname}{\noindent\texttt{shu-cpp-default-global-namespace }}
\usebox{\funcname}
\index{shu-cpp-default-global-namespace} \hfill [Custom]

\begin{doc-string}
The string that defines the default global C++ namepace, if any.
If this has a value, then C++ classes are declared with a two level
namespace with the global namespace encompassing the local one
\end{doc-string}

\vspace{1em}
\noindent
\savebox{\funcname}{\noindent\texttt{shu-cpp-default-namespace }}
\usebox{\funcname}
\index{shu-cpp-default-namespace} \hfill [Custom]

\begin{doc-string}
The string that defines the default C++ namepace, if any.
\end{doc-string}

\vspace{1em}
\noindent
\savebox{\funcname}{\noindent\texttt{shu-cpp-file-name }}
\usebox{\funcname}
\index{shu-cpp-file-name} \hfill [Constant]

\begin{doc-string}
A regular expression to match the name of a C or C++ file in the file system.
\end{doc-string}

\vspace{1em}
\noindent
\savebox{\funcname}{\noindent\texttt{shu-cpp-include-user-brackets }}
\usebox{\funcname}
\index{shu-cpp-include-user-brackets} \hfill [Custom]

\begin{doc-string}
Set non-nil if user written include files are to be delimited by
angle brackets instead of quotes.
In many C anc C++ environments, system include files such as stdio.h are delimited
by angle brackets, for example:

\begin{verbatim}
      #include <stdio.h>
\end{verbatim}

while user written include files are delimited by quotes, for example:

\begin{verbatim}
      #include ``myclass.h''
\end{verbatim}

If this variable is non-nil, then user written include files are delimited
by angle brackets and an include of ``myclass.h'' would be written as

\begin{verbatim}
      #include <myclass.h>
\end{verbatim}
\end{doc-string}

\vspace{1em}
\noindent
\savebox{\funcname}{\noindent\texttt{shu-cpp-indent-length }}
\usebox{\funcname}
\index{shu-cpp-indent-length} \hfill [Custom]

\begin{doc-string}
Size of the standard indent for names within class declarations, etc.
\end{doc-string}

\vspace{1em}
\noindent
\savebox{\funcname}{\noindent\texttt{shu-cpp-name }}
\usebox{\funcname}
\index{shu-cpp-name} \hfill [Constant]

\begin{doc-string}
A regular expression to match a variable name in a C or C++ program.
\end{doc-string}

\vspace{1em}
\noindent
\savebox{\funcname}{\noindent\texttt{shu-cpp-name-list }}
\usebox{\funcname}
\index{shu-cpp-name-list} \hfill [Constant]

\begin{doc-string}
List of all characters that can be present in a C++ file name.
\end{doc-string}

\vspace{1em}
\noindent
\savebox{\funcname}{\noindent\texttt{shu-cpp-use-bde-library }}
\usebox{\funcname}
\index{shu-cpp-use-bde-library} \hfill [Custom]

\begin{doc-string}
Set non-nil if the BDE library is to be used for generated C++ code.
\end{doc-string}

\vspace{1em}
\noindent
\savebox{\funcname}{\noindent\texttt{shu-current-line }}
\usebox{\funcname}
\index{shu-current-line} \hfill [Function]

\begin{doc-string}
Return the line number of the current line relative to the start of the buffer.
\end{doc-string}

\vspace{1em}
\noindent
\savebox{\funcname}{\noindent\texttt{shu-end-of-string }}
\usebox{\funcname}\emph{string-term}
\index{shu-end-of-string} \hfill [Function]

\begin{doc-string}
Return the point that terminates the current quoted string in the buffer.
The single argument \emph{string-term} is a string containing the character that
started the string (single or double quote).  Return nil if the current
string is not terminated in the buffer.
\end{doc-string}

\vspace{1em}
\noindent
\savebox{\funcname}{\noindent\texttt{shu-fixed-format-num }}
\usebox{\funcname}\emph{num} \emph{width}
\index{shu-fixed-format-num} \hfill [Function]

\begin{doc-string}
Return a printable representation of \emph{num} in a string right justified
and pad filled to length \emph{width}.  The number is formatted as comma separated
as defined by shu-group-number.
\end{doc-string}

\vspace{1em}
\noindent
\savebox{\funcname}{\noindent\texttt{shu-format-num }}
\usebox{\funcname}\emph{num} \emph{width}
\index{shu-format-num} \hfill [Function]

\begin{doc-string}
Return a printable representation of \emph{num} in a string right justified
and pad filled to length \emph{width}.
\end{doc-string}

\vspace{1em}
\noindent
\savebox{\funcname}{\noindent\texttt{shu-global-buffer-name }}
\usebox{\funcname}
\index{shu-global-buffer-name} \hfill [Constant]

\begin{doc-string}
The name of the buffer into which shu-global-operation places its output.
\end{doc-string}

\vspace{1em}
\noindent
\savebox{\funcname}{\noindent\texttt{shu-goto-line }}
\usebox{\funcname}\emph{line-number}
\index{shu-goto-line} \hfill [Function]

\begin{doc-string}
Move point to line \emph{line-number}.
\end{doc-string}

\vspace{1em}
\noindent
\savebox{\funcname}{\noindent\texttt{shu-group-number }}
\usebox{\funcname}\emph{num} \textbf{\&optional} \emph{size} \emph{char}
\index{shu-group-number} \hfill [Function]

\begin{doc-string}
Format \emph{num} as string grouped to \emph{size} with \emph{char}.  Default \emph{size} if 3.  Default \emph{char}
is ','.  e.g., 1234567 is formatted as 1,234,567.  Argument to be formatted may be
either a string or a number.
\end{doc-string}

\vspace{1em}
\noindent
\savebox{\funcname}{\noindent\texttt{shu-kill-new }}
\usebox{\funcname}\emph{string}
\index{shu-kill-new} \hfill [Function]

\begin{doc-string}
Effectively do a kill-new with \emph{string} but use kill-ring-save from
a temporary buffer.  This seems to to a better job of putting \emph{string}
in a place from which other programs running on Linux and Windows can
do a paste.
\end{doc-string}

\vspace{1em}
\noindent
\savebox{\funcname}{\noindent\texttt{shu-line-and-column-at }}
\usebox{\funcname}\emph{arg}
\index{shu-line-and-column-at} \hfill [Function]

\begin{doc-string}
Return the line number and column number of the point passed in as an argument.
\end{doc-string}

\vspace{1em}
\noindent
\savebox{\funcname}{\noindent\texttt{shu-minimum-leading-space }}
\usebox{\funcname}\emph{arg} \textbf{\&optional}
\index{shu-minimum-leading-space} \hfill [Function]
\hspace*{\wd\funcname}

\begin{doc-string}
Find the amount of white space in front of point and return either that
count or \emph{arg}, whichever is smaller.  Used by functions that wish to
safely delete \emph{arg} characters of white space from the current position
without deleting any characters that are not white space.
An optional second argument is a string that defines what is meant
by white space.  The default definition is blanks and tabs.
\end{doc-string}

\vspace{1em}
\noindent
\savebox{\funcname}{\noindent\texttt{shu-non-cpp-name }}
\usebox{\funcname}
\index{shu-non-cpp-name} \hfill [Constant]

\begin{doc-string}
A regular expression to match a character not valid in a variable name
in a C or C++ program.
\end{doc-string}

\vspace{1em}
\noindent
\savebox{\funcname}{\noindent\texttt{shu-not-all-whitespace-regexp }}
\usebox{\funcname}
\index{shu-not-all-whitespace-regexp} \hfill [Constant]

\begin{doc-string}
Regular expression to search for non-whitespace.  Since the syntax table considers
newline to be (among other things) a comment terminator, the usual \\s- won't work
for whitespace that includes newlines.  Note that this regular expression is a
character alternative enclosed in left and right brackets.  skip-chars-forward does
not expect character alternatives to be enclosed in square brackets and will include
the left and right brackets in the class of characters to be skipped.
\end{doc-string}

\vspace{1em}
\noindent
\savebox{\funcname}{\noindent\texttt{shu-point-in-string }}
\usebox{\funcname}
\index{shu-point-in-string} \hfill [Function]

\begin{doc-string}
Return the start position of the string text if point is sitting between a pair
of non-escaped quotes (double quotes).  The left-hand quote (opening quote) must be
on the same line as point.  Does not take into account comments, \#if 0, etc.  It is
assumed that someone who wants to operate on a string will generally position point
within a legitimate string.
\end{doc-string}

\vspace{1em}
\noindent
\savebox{\funcname}{\noindent\texttt{shu-remove-trailing-all-whitespace }}
\usebox{\funcname}\emph{input-string}
\index{shu-remove-trailing-all-whitespace} \hfill [Function]

\begin{doc-string}
Return a copy of \emph{input-string} with all trailing whitespace removed.  All control
characters are considered whitespace.
\end{doc-string}

\vspace{1em}
\noindent
\savebox{\funcname}{\noindent\texttt{shu-the-column-at }}
\usebox{\funcname}\emph{arg}
\index{shu-the-column-at} \hfill [Function]

\begin{doc-string}
Return the column number of the point passed in as an argument.
\end{doc-string}

\vspace{1em}
\noindent
\savebox{\funcname}{\noindent\texttt{shu-the-line-at }}
\usebox{\funcname}\emph{arg}
\index{shu-the-line-at} \hfill [Function]

\begin{doc-string}
Return the line number of the point passed in as an argument.  The line number is
relative to the start of the buffer, whether or not narrowing is in effect.
\end{doc-string}

\vspace{1em}
\noindent
\savebox{\funcname}{\noindent\texttt{shu-unit-test-buffer }}
\usebox{\funcname}
\index{shu-unit-test-buffer} \hfill [Constant]

\begin{doc-string}
The name of the buffer into which unit tests place their output and debug trace.
\end{doc-string}

\section{shu-bde-cp}


A collection of useful functions for generating C++ skeleton code files
and classes for code written in Bloomberg, L.P. BDE style.


\subsection{List of functions by alias name}

A list of aliases and associated function names.



\vspace{1em}
\noindent
\savebox{\funcname}{\noindent\texttt{gen-bb-component }}
\usebox{\funcname}\emph{class-name}
\index{gen-bb-component} \hfill [Command]\\%
 (Function: shu-gen-bb-component)

\begin{doc-string}
Generate the three files for a new component: .cpp, .h, and .t.cpp
\end{doc-string}

\subsection{List of functions and variables}

List of functions and variable definitions in this package.



\vspace{1em}
\noindent
\savebox{\funcname}{\noindent\texttt{shu-bb-cpp-set-alias }}
\usebox{\funcname}
\index{shu-bb-cpp-set-alias} \hfill [Function]

\begin{doc-string}
Set the common alias names for the functions in shu-bb-cpp.
These are generally the same as the function names with the leading
shu- prefix removed.
\end{doc-string}

\vspace{1em}
\noindent
\savebox{\funcname}{\noindent\texttt{shu-gen-bb-component }}
\usebox{\funcname}\emph{class-name}
\index{shu-gen-bb-component} \hfill [Command]\\%
 (Alias: gen-bb-component)

\begin{doc-string}
Generate the three files for a new component: .cpp, .h, and .t.cpp
\end{doc-string}

\vspace{1em}
\noindent
\savebox{\funcname}{\noindent\texttt{shu-generate-bb-cfile }}
\usebox{\funcname}\emph{author} \emph{namespace} \emph{class-name}
\index{shu-generate-bb-cfile} \hfill [Function]

\begin{doc-string}
Generate a skeleton cpp file
\end{doc-string}

\vspace{1em}
\noindent
\savebox{\funcname}{\noindent\texttt{shu-generate-bb-hfile }}
\usebox{\funcname}\emph{author} \emph{namespace} \emph{class-name}
\index{shu-generate-bb-hfile} \hfill [Function]

\begin{doc-string}
Generate a skeleton header file
\end{doc-string}

\vspace{1em}
\noindent
\savebox{\funcname}{\noindent\texttt{shu-generate-bb-tfile }}
\usebox{\funcname}\emph{author} \emph{namespace} \emph{class-name}
\index{shu-generate-bb-tfile} \hfill [Command]

\begin{doc-string}
Generate a skeleton t.cpp file
\end{doc-string}

\section{shu-bde}


A collection of useful functions for generating C++ skeleton code files
and classes for code written in Bloomberg, L.P. BDE style.


\subsection{List of functions by alias name}

A list of aliases and associated function names.



\vspace{1em}
\noindent
\savebox{\funcname}{\noindent\texttt{bde-add-guard }}
\usebox{\funcname}
\index{bde-add-guard} \hfill [Command]\\%
 (Function: shu-bde-add-guard)

\begin{doc-string}
Add the BDE include guards around an existing \#include directive.  If the line
before the \#include directive contains a valid guard, then we do not add a guard
and position point to the line following the \#include.  This makes it possible to
run bde-all-guard on a file that contains some guarded \#includes and some unguarded
\#includes.  Only the unguarded ones will have the guard added.
\end{doc-string}

\vspace{1em}
\noindent
\savebox{\funcname}{\noindent\texttt{bde-all-guard }}
\usebox{\funcname}
\index{bde-all-guard} \hfill [Command]\\%
 (Function: shu-bde-all-guard)

\begin{doc-string}
Add the BDE include guards around all of the \#include directives in a file
or narrowed region.
\end{doc-string}

\vspace{1em}
\noindent
\savebox{\funcname}{\noindent\texttt{bde-decl }}
\usebox{\funcname}\emph{class-name}
\index{bde-decl} \hfill [Command]\\%
 (Function: shu-bde-decl)

\begin{doc-string}
Generate a skeleton BDE class declaration at point.
\end{doc-string}

\vspace{1em}
\noindent
\savebox{\funcname}{\noindent\texttt{bde-gen }}
\usebox{\funcname}\emph{class-name}
\index{bde-gen} \hfill [Command]\\%
 (Function: shu-bde-gen)

\begin{doc-string}
Generate a skeleton BDE class code generation at point.
\end{doc-string}

\vspace{1em}
\noindent
\savebox{\funcname}{\noindent\texttt{bde-include }}
\usebox{\funcname}\emph{fn}
\index{bde-include} \hfill [Command]\\%
 (Function: shu-bde-include)

\begin{doc-string}
Insert at the current line, the BDE include guard sequence of
\#ifndef INCLUDED\_GUARD
\#include $<$guard.h$>$
\#endif
\end{doc-string}

\vspace{1em}
\noindent
\savebox{\funcname}{\noindent\texttt{bde-sdecl }}
\usebox{\funcname}\emph{class-name}
\index{bde-sdecl} \hfill [Command]\\%
 (Function: shu-bde-sdecl)

\begin{doc-string}
Generate a skeleton BDE struct definition at point.
\end{doc-string}

\vspace{1em}
\noindent
\savebox{\funcname}{\noindent\texttt{bde-sgen }}
\usebox{\funcname}\emph{class-name}
\index{bde-sgen} \hfill [Command]\\%
 (Function: shu-bde-sgen)

\begin{doc-string}
Generate a skeleton BDE struct code generation at point.
\end{doc-string}

\vspace{1em}
\noindent
\savebox{\funcname}{\noindent\texttt{gen-bde-component }}
\usebox{\funcname}\emph{class-name}
\index{gen-bde-component} \hfill [Command]\\%
 (Function: shu-gen-bde-component)

\begin{doc-string}
Generate the three files for a new component: .cpp, .h, and .t.cpp
\end{doc-string}

\subsection{List of functions and variables}

List of functions and variable definitions in this package.



\vspace{1em}
\noindent
\savebox{\funcname}{\noindent\texttt{shu-bde-add-guard }}
\usebox{\funcname}
\index{shu-bde-add-guard} \hfill [Command]\\%
 (Alias: bde-add-guard)

\begin{doc-string}
Add the BDE include guards around an existing \#include directive.  If the line
before the \#include directive contains a valid guard, then we do not add a guard
and position point to the line following the \#include.  This makes it possible to
run bde-all-guard on a file that contains some guarded \#includes and some unguarded
\#includes.  Only the unguarded ones will have the guard added.
\end{doc-string}

\vspace{1em}
\noindent
\savebox{\funcname}{\noindent\texttt{shu-bde-all-guard }}
\usebox{\funcname}
\index{shu-bde-all-guard} \hfill [Command]\\%
 (Alias: bde-all-guard)

\begin{doc-string}
Add the BDE include guards around all of the \#include directives in a file
or narrowed region.
\end{doc-string}

\vspace{1em}
\noindent
\savebox{\funcname}{\noindent\texttt{shu-bde-decl }}
\usebox{\funcname}\emph{class-name}
\index{shu-bde-decl} \hfill [Command]\\%
 (Alias: bde-decl)

\begin{doc-string}
Generate a skeleton BDE class declaration at point.
\end{doc-string}

\vspace{1em}
\noindent
\savebox{\funcname}{\noindent\texttt{shu-bde-gen }}
\usebox{\funcname}\emph{class-name}
\index{shu-bde-gen} \hfill [Command]\\%
 (Alias: bde-gen)

\begin{doc-string}
Generate a skeleton BDE class code generation at point.
\end{doc-string}

\vspace{1em}
\noindent
\savebox{\funcname}{\noindent\texttt{shu-bde-gen-cfile-copyright-hook }}
\usebox{\funcname}
\index{shu-bde-gen-cfile-copyright-hook} \hfill [Custom]

\begin{doc-string}
Generate the text that is the copyright notice placed in a code file,
if any.
\end{doc-string}

\vspace{1em}
\noindent
\savebox{\funcname}{\noindent\texttt{shu-bde-gen-file-identifier-hook }}
\usebox{\funcname}
\index{shu-bde-gen-file-identifier-hook} \hfill [Custom]

\begin{doc-string}
Generate the text that constitutes a source file identifier, if any.
\end{doc-string}

\vspace{1em}
\noindent
\savebox{\funcname}{\noindent\texttt{shu-bde-gen-h-includes-hook }}
\usebox{\funcname}
\index{shu-bde-gen-h-includes-hook} \hfill [Custom]

\begin{doc-string}
Generate the code for the standard includes in a header file.
\end{doc-string}

\vspace{1em}
\noindent
\savebox{\funcname}{\noindent\texttt{shu-bde-gen-hfile-copyright-hook }}
\usebox{\funcname}
\index{shu-bde-gen-hfile-copyright-hook} \hfill [Custom]

\begin{doc-string}
Generate the text that is the copyright notice placed in a header file,
if any.
\end{doc-string}

\vspace{1em}
\noindent
\savebox{\funcname}{\noindent\texttt{shu-bde-gen-tfile-copyright-hook }}
\usebox{\funcname}
\index{shu-bde-gen-tfile-copyright-hook} \hfill [Custom]

\begin{doc-string}
Generate the text that is the copyright notice placed in a unit test
file, if any.
\end{doc-string}

\vspace{1em}
\noindent
\savebox{\funcname}{\noindent\texttt{shu-bde-include }}
\usebox{\funcname}\emph{fn}
\index{shu-bde-include} \hfill [Command]\\%
 (Alias: bde-include)

\begin{doc-string}
Insert at the current line, the BDE include guard sequence of
\#ifndef INCLUDED\_GUARD
\#include $<$guard.h$>$
\#endif
\end{doc-string}

\vspace{1em}
\noindent
\savebox{\funcname}{\noindent\texttt{shu-bde-include-guard }}
\usebox{\funcname}\textbf{\&optional} \emph{fn}
\index{shu-bde-include-guard} \hfill [Function]

\begin{doc-string}
Return the name of the macro variable to be used in a BDE style include guard.
Name of the current buffer file name is used if no file name is passed in as the
only optional argument.  This is the name of the macro variable that is used in the
include guard.  If the name of the file is foo\_something.h, then this function
returns INCLUDED\_FOO\_SOMETHING.  See also shu-bde-include-guard-fn
\end{doc-string}

\vspace{1em}
\noindent
\savebox{\funcname}{\noindent\texttt{shu-bde-include-guard-fn }}
\usebox{\funcname}\textbf{\&optional} \emph{fn}
\index{shu-bde-include-guard-fn} \hfill [Function]

\begin{doc-string}
Return the file name name of the macro variable to be used in a BDE style include
guard.  Name of the current buffer file name is used if no file name is passed in as
the only optional argument.  This is only the file name part of the include guard.
If the name of the file is foo\_something.h, then this function returns
FOO\_SOMETHING.  The full name of the macro variable would be
INCLUDED\_FOO\_SOMETHING.  See also shu-bde-include-guard
\end{doc-string}

\vspace{1em}
\noindent
\savebox{\funcname}{\noindent\texttt{shu-bde-insert-guard }}
\usebox{\funcname}\emph{fn} \emph{at-top}
\index{shu-bde-insert-guard} \hfill [Function]

\begin{doc-string}
Insert a \#ifndef / \#endif guard around an \#include directive.  \emph{fn} is the name of
the included file.  \emph{at-top} is true if the \#include directive is located on the first
line of the file so there is no line above it.
\end{doc-string}

\vspace{1em}
\noindent
\savebox{\funcname}{\noindent\texttt{shu-bde-sdecl }}
\usebox{\funcname}\emph{class-name}
\index{shu-bde-sdecl} \hfill [Command]\\%
 (Alias: bde-sdecl)

\begin{doc-string}
Generate a skeleton BDE struct definition at point.
\end{doc-string}

\vspace{1em}
\noindent
\savebox{\funcname}{\noindent\texttt{shu-bde-set-alias }}
\usebox{\funcname}
\index{shu-bde-set-alias} \hfill [Function]

\begin{doc-string}
Set the common alias names for the functions in shu-bde.
These are generally the same as the function names with the leading
shu- prefix removed.
\end{doc-string}

\vspace{1em}
\noindent
\savebox{\funcname}{\noindent\texttt{shu-bde-sgen }}
\usebox{\funcname}\emph{class-name}
\index{shu-bde-sgen} \hfill [Command]\\%
 (Alias: bde-sgen)

\begin{doc-string}
Generate a skeleton BDE struct code generation at point.
\end{doc-string}

\vspace{1em}
\noindent
\savebox{\funcname}{\noindent\texttt{shu-gen-bde-component }}
\usebox{\funcname}\emph{class-name}
\index{shu-gen-bde-component} \hfill [Command]\\%
 (Alias: gen-bde-component)

\begin{doc-string}
Generate the three files for a new component: .cpp, .h, and .t.cpp
\end{doc-string}

\vspace{1em}
\noindent
\savebox{\funcname}{\noindent\texttt{shu-generate-bde-cfile }}
\usebox{\funcname}\emph{author} \emph{namespace} \emph{class-name}
\index{shu-generate-bde-cfile} \hfill [Function]

\begin{doc-string}
Generate a skeleton cpp file
\end{doc-string}

\vspace{1em}
\noindent
\savebox{\funcname}{\noindent\texttt{shu-generate-bde-hfile }}
\usebox{\funcname}\emph{author} \emph{namespace} \emph{class-name}
\index{shu-generate-bde-hfile} \hfill [Function]

\begin{doc-string}
Generate a skeleton header file
\end{doc-string}

\vspace{1em}
\noindent
\savebox{\funcname}{\noindent\texttt{shu-generate-bde-tfile }}
\usebox{\funcname}\emph{author} \emph{namespace} \emph{class-name}
\index{shu-generate-bde-tfile} \hfill [Command]

\begin{doc-string}
Generate a skeleton t.cpp file
\end{doc-string}

\section{shu-capture-doc}


Collection of functions used to capture function and variable definitions
along with their associated doc strings in elisp code.  It can then write
this information into a buffer in either markdown or LaTex format for
subsequent publication.

This mechanism was used to create most of the documentation for the elisp
functions in this repository.


\subsection{List of functions and variables}

List of functions and variable definitions in this package.



\vspace{1em}
\noindent
\savebox{\funcname}{\noindent\texttt{shu-capture-a-type-after }}
\usebox{\funcname}
\index{shu-capture-a-type-after} \hfill [Constant]

\begin{doc-string}
The a-list key value that identifies the string that is placed after a verbatim
code snippet.
\end{doc-string}

\vspace{1em}
\noindent
\savebox{\funcname}{\noindent\texttt{shu-capture-a-type-arg }}
\usebox{\funcname}
\index{shu-capture-a-type-arg} \hfill [Constant]

\begin{doc-string}
The a-list key value that identifies the function that converts an argument name
to markup.
\end{doc-string}

\vspace{1em}
\noindent
\savebox{\funcname}{\noindent\texttt{shu-capture-a-type-before }}
\usebox{\funcname}
\index{shu-capture-a-type-before} \hfill [Constant]

\begin{doc-string}
The a-list key value that identifies the string that is placed before a verbatim
code snippet.
\end{doc-string}

\vspace{1em}
\noindent
\savebox{\funcname}{\noindent\texttt{shu-capture-a-type-buf }}
\usebox{\funcname}
\index{shu-capture-a-type-buf} \hfill [Constant]

\begin{doc-string}
The a-list key value that identifies the function that converts a buffer name or
other name that begins and ends with asterisks to markup.
\end{doc-string}

\vspace{1em}
\noindent
\savebox{\funcname}{\noindent\texttt{shu-capture-a-type-close-quote }}
\usebox{\funcname}
\index{shu-capture-a-type-close-quote} \hfill [Constant]

\begin{doc-string}
The a-list key value that identifies the string that is a close quote.
\end{doc-string}

\vspace{1em}
\noindent
\savebox{\funcname}{\noindent\texttt{shu-capture-a-type-doc-string }}
\usebox{\funcname}
\index{shu-capture-a-type-doc-string} \hfill [Constant]

\begin{doc-string}
The a-list key value that identifies the function that converts a key word, such
as ``\&optional'' or ``\&rest'' to markup.
\end{doc-string}

\vspace{1em}
\noindent
\savebox{\funcname}{\noindent\texttt{shu-capture-a-type-enclose-doc }}
\usebox{\funcname}
\index{shu-capture-a-type-enclose-doc} \hfill [Constant]

\begin{doc-string}
The a-list key value that identifies the function that converts a key word, such
as ``\&optional'' or ``\&rest'' to markup.
\end{doc-string}

\vspace{1em}
\noindent
\savebox{\funcname}{\noindent\texttt{shu-capture-a-type-func }}
\usebox{\funcname}
\index{shu-capture-a-type-func} \hfill [Constant]

\begin{doc-string}
The a-list key value that identifies the function that formats a function signature'
\end{doc-string}

\vspace{1em}
\noindent
\savebox{\funcname}{\noindent\texttt{shu-capture-a-type-hdr }}
\usebox{\funcname}
\index{shu-capture-a-type-hdr} \hfill [Constant]

\begin{doc-string}
The a-list key value that identifies the function that emits section headers
\end{doc-string}

\vspace{1em}
\noindent
\savebox{\funcname}{\noindent\texttt{shu-capture-a-type-keywd }}
\usebox{\funcname}
\index{shu-capture-a-type-keywd} \hfill [Constant]

\begin{doc-string}
The a-list key value that identifies the function that converts a key word, such
as ``\&optional'' or ``\&rest'' to markup.
\end{doc-string}

\vspace{1em}
\noindent
\savebox{\funcname}{\noindent\texttt{shu-capture-a-type-open-quote }}
\usebox{\funcname}
\index{shu-capture-a-type-open-quote} \hfill [Constant]

\begin{doc-string}
The a-list key value that identifies the string that is an open quote.
\end{doc-string}

\vspace{1em}
\noindent
\savebox{\funcname}{\noindent\texttt{shu-capture-alias-list }}
\usebox{\funcname}
\index{shu-capture-alias-list} \hfill [Variable]

\begin{doc-string}
The alist that holds all of the alias names.
\end{doc-string}

\vspace{1em}
\noindent
\savebox{\funcname}{\noindent\texttt{shu-capture-aliases }}
\usebox{\funcname}
\index{shu-capture-aliases} \hfill [Function]

\begin{doc-string}
Undocumented
\end{doc-string}

\vspace{1em}
\noindent
\savebox{\funcname}{\noindent\texttt{shu-capture-all-latex }}
\usebox{\funcname}
\index{shu-capture-all-latex} \hfill [Command]

\begin{doc-string}
Visit all of the files in \emph{shu-capture-file-list}, invoking \emph{shu-capture-latex} on
each file to capture its documentation and turn it into LaTex source.
\end{doc-string}

\vspace{1em}
\noindent
\savebox{\funcname}{\noindent\texttt{shu-capture-all-md }}
\usebox{\funcname}
\index{shu-capture-all-md} \hfill [Command]

\begin{doc-string}
Visit all of the files in \emph{shu-capture-file-list}, invoking \emph{shu-capture-md} on each
file to capture its documentation and turn it into markdown source.
\end{doc-string}

\vspace{1em}
\noindent
\savebox{\funcname}{\noindent\texttt{shu-capture-arg-to-latex }}
\usebox{\funcname}\emph{arg-name}
\index{shu-capture-arg-to-latex} \hfill [Function]

\begin{doc-string}
Convert a function argument in a doc-string or argument list to LaTex.
\end{doc-string}

\vspace{1em}
\noindent
\savebox{\funcname}{\noindent\texttt{shu-capture-arg-to-md }}
\usebox{\funcname}\emph{arg-name}
\index{shu-capture-arg-to-md} \hfill [Function]

\begin{doc-string}
Convert a function argument in a doc-string or argument list to markdown.
\end{doc-string}

\vspace{1em}
\noindent
\savebox{\funcname}{\noindent\texttt{shu-capture-attr-alias }}
\usebox{\funcname}
\index{shu-capture-attr-alias} \hfill [Constant]

\begin{doc-string}
Bit that indicates that a function is identified by its alias name
\end{doc-string}

\vspace{1em}
\noindent
\savebox{\funcname}{\noindent\texttt{shu-capture-attr-const }}
\usebox{\funcname}
\index{shu-capture-attr-const} \hfill [Constant]

\begin{doc-string}
Bit that indicates that a definition is a defconst
\end{doc-string}

\vspace{1em}
\noindent
\savebox{\funcname}{\noindent\texttt{shu-capture-attr-custom }}
\usebox{\funcname}
\index{shu-capture-attr-custom} \hfill [Variable]

\begin{doc-string}
Bit that indicates that a definition is a defcustom
\end{doc-string}

\vspace{1em}
\noindent
\savebox{\funcname}{\noindent\texttt{shu-capture-attr-inter }}
\usebox{\funcname}
\index{shu-capture-attr-inter} \hfill [Constant]

\begin{doc-string}
Bit that indicates that a function is interactive
\end{doc-string}

\vspace{1em}
\noindent
\savebox{\funcname}{\noindent\texttt{shu-capture-attr-macro }}
\usebox{\funcname}
\index{shu-capture-attr-macro} \hfill [Constant]

\begin{doc-string}
Bit that indicates that a function is a macro
\end{doc-string}

\vspace{1em}
\noindent
\savebox{\funcname}{\noindent\texttt{shu-capture-attr-var }}
\usebox{\funcname}
\index{shu-capture-attr-var} \hfill [Variable]

\begin{doc-string}
Bit that indicates that a definition is a defvar
\end{doc-string}

\vspace{1em}
\noindent
\savebox{\funcname}{\noindent\texttt{shu-capture-buf-to-latex }}
\usebox{\funcname}\emph{buf-name}
\index{shu-capture-buf-to-latex} \hfill [Function]

\begin{doc-string}
Convert a buffer name or other name that starts and ends with asterisks
 in a doc-string to markdown.
\end{doc-string}

\vspace{1em}
\noindent
\savebox{\funcname}{\noindent\texttt{shu-capture-buf-to-md }}
\usebox{\funcname}\emph{buf-name}
\index{shu-capture-buf-to-md} \hfill [Function]

\begin{doc-string}
Convert a buffer name or other name that starts and ends with asterisks
 in a doc-string to markdown.
\end{doc-string}

\vspace{1em}
\noindent
\savebox{\funcname}{\noindent\texttt{shu-capture-buffer-name }}
\usebox{\funcname}
\index{shu-capture-buffer-name} \hfill [Constant]

\begin{doc-string}
Name of the buffer into which the converted documentation is written
\end{doc-string}

\vspace{1em}
\noindent
\savebox{\funcname}{\noindent\texttt{shu-capture-code-in-doc }}
\usebox{\funcname}\emph{before-code} \emph{after-code}
\index{shu-capture-code-in-doc} \hfill [Function]
\hspace*{\wd\funcname}

\begin{doc-string}
The current buffer is assumed to hold a doc string that is being converted to
markdown.  Any line that is indented to column \emph{shu-capture-doc-code-indent} or
greater is assumed to be a code snippet.  To format this as a code snippet,
\emph{before-code} is placed one line above the code snippet and \emph{after-code} is placed
one line below the code snippet.  Return the number of code snippets marked.
Because we only want to replace special characters in text that does not include
a code snippet, then each time we find the end of regular text, we call the
\emph{text-converter} function passing it the beginning and end point of the regular
text.  The \emph{text-converter} function may expand the amount of text present if it
adds characters to the text.  It is the responsibility of the \emph{text-converter}
function to return the new text end point to this function.
\end{doc-string}

\vspace{1em}
\noindent
\savebox{\funcname}{\noindent\texttt{shu-capture-code-in-md }}
\usebox{\funcname}
\index{shu-capture-code-in-md} \hfill [Function]

\begin{doc-string}
The current buffer is assumed to hold a doc string that is being converted to
markdown.  Any line that is indented to column \emph{shu-capture-doc-code-indent} or
greater is assumed to be a code snippet and will be surrounded by `````'' to make
it a code snippet in markdown.  Return the number of code snippets marked.
\end{doc-string}

\vspace{1em}
\noindent
\savebox{\funcname}{\noindent\texttt{shu-capture-commentary }}
\usebox{\funcname}
\index{shu-capture-commentary} \hfill [Function]

\begin{doc-string}
Search through an elisp file for a package name and a commentary section.
Return a cons cell whose car is the package name and whose cdr is the prose
found in the commentary section.
\end{doc-string}

\vspace{1em}
\noindent
\savebox{\funcname}{\noindent\texttt{shu-capture-convert-args-to-markup }}
\usebox{\funcname}\emph{signature}
\index{shu-capture-convert-args-to-markup} \hfill [Function]
\hspace*{\wd\funcname}\emph{keywd-converter}

\begin{doc-string}
\emph{signature} contains the function signature (both function name and arguments).
\emph{arg-converter} is the function used to convert an argument to markup.  \emph{keywd-converter}
is the function used to convert an argument list keyword, such as ``\&optional'' or ``\&rest''
to markup.

This function returns a cons cell pointing to two lists.  The first list contains the length
of each argument name prior to conversion to markup.  This is because the amount of space
on a line is largely determined by the length of the unconverted argument.  ``arg'' will
take much less space on a line than will the same word with markup added.  The second list
contains each of the argument names converted to the appropriate markup.

Given the following function signature:

\begin{verbatim}
     do-something (with these things \&optional and \&rest others)
\end{verbatim}

the length list will contain (4, 5, 6, 9, 3, 5, 6).  The converted arguments list for
markdown will contain (``\emph\{*with*\}'', ``\emph\{*these*\}'', ``\emph\{*things*\}'', ``\emph\{**\}\&optional\emph\{**\}'', ``\emph\{*and*\}'',
``\emph\{**\}\&rest\emph\{**\}'', ``\emph\{*others*\}'').

If the function signature contains no arguments, then nil is returned instead of the
above described cons cell.
\end{doc-string}

\vspace{1em}
\noindent
\savebox{\funcname}{\noindent\texttt{shu-capture-convert-doc-string }}
\usebox{\funcname}\emph{signature}
\index{shu-capture-convert-doc-string} \hfill [Function]
\hspace*{\wd\funcname}\emph{converters}

\begin{doc-string}
\emph{description} contains a doc string from a function definition (with leading
and trailing quotes removed).  \emph{converters} is an a-list of functions and strings as
follows:

\begin{verbatim}
      Key                              Value
      ---                              -----
      shu-capture-a-type-hdr           Function to format a section header
      shu-capture-a-type-func          Function to format a function signature
      shu-capture-a-type-buf           Function to format a buffer name
      shu-capture-a-type-arg           Function to format an argument name
      shu-capture-a-type-keywd         Function to format a key word
      shu-capture-a-type-doc-string    Function to finish formatting the doc string
      shu-capture-a-type-enclose-doc   Function to enclose doc string in begin / end
      shu-capture-a-type-before        String that starts a block of verbatim code
      shu-capture-a-type-after         String that ends a block of verbatim code
      shu-capture-a-type-open-quote    String that is an open quote
      shu-capture-a-type-close-quote   String that is a close quote
\end{verbatim}

This function turns escaped quotes into open and close quote strings, turns names
with leading and trailing asterisks (e.g., \emph\{**project-buffer**\}) into formatted buffer
names, turns upper case names that match any argument names into lower case,
formatted argument names.  This is an internal function of shu-capture-doc and
will likely crash if called with an invalid a-list.
\end{doc-string}

\vspace{1em}
\noindent
\savebox{\funcname}{\noindent\texttt{shu-capture-convert-func-latex }}
\usebox{\funcname}\emph{func-def} \emph{converters}
\index{shu-capture-convert-func-latex} \hfill [Function]
\hspace*{\wd\funcname}

\begin{doc-string}
Take a function definition and turn it into a string of LaTex.  Return said string.
\end{doc-string}

\vspace{1em}
\noindent
\savebox{\funcname}{\noindent\texttt{shu-capture-convert-func-md }}
\usebox{\funcname}\emph{func-def} \emph{converters}
\index{shu-capture-convert-func-md} \hfill [Function]
\hspace*{\wd\funcname}

\begin{doc-string}
Take a function definition and turn it into a string of markdown.  Return said string.
\end{doc-string}

\vspace{1em}
\noindent
\savebox{\funcname}{\noindent\texttt{shu-capture-convert-quotes }}
\usebox{\funcname}\emph{open-quote} \emph{close-quote}
\index{shu-capture-convert-quotes} \hfill [Function]

\begin{doc-string}
Go through the current buffer converting any escaped quote to either an open or
close quote.  If an escaped quote is preceded by whitespace, ``(``, ``\{'', ``$<$``, or ``$>$``,
or by a close quote, then we replace it with an open quote.  Otherwise we replace it
with a close quote.
\end{doc-string}

\vspace{1em}
\noindent
\savebox{\funcname}{\noindent\texttt{shu-capture-doc }}
\usebox{\funcname}\emph{converters}
\index{shu-capture-doc} \hfill [Function]

\begin{doc-string}
Top level function that captures all definitions and doc strings in a language
neutral manner and then uses the supplied \emph{converters} to convert the documentation to
either markdown or LaTex.
\end{doc-string}

\vspace{1em}
\noindent
\savebox{\funcname}{\noindent\texttt{shu-capture-doc-code-indent }}
\usebox{\funcname}
\index{shu-capture-doc-code-indent} \hfill [Constant]

\begin{doc-string}
Any line indented by this much in a doc string is assumed to be a sample
code snippet.
\end{doc-string}

\vspace{1em}
\noindent
\savebox{\funcname}{\noindent\texttt{shu-capture-doc-convert-args }}
\usebox{\funcname}\emph{signature} \emph{converters}
\index{shu-capture-doc-convert-args} \hfill [Function]

\begin{doc-string}
The current buffer contains a doc string from a function.  The argument to this
function is the \emph{signature} of the function for which the doc string was written.
This function goes through the doc string buffer looking for any word that is all
upper case.  If the upper case word matches the name of an argument to the function,
it is passed to the CONVERTER function for conversion into a markup language, which
is probably markdown or LaTex, and it is then replaced in the doc string buffer.

For example, if the function has the following signature:

\begin{verbatim}
     do-something (hat cat)
\end{verbatim}

with the following doc string:

  ``The Linux HAT is converted to an IBM CAT.''

would be converted to:

  ``The Linux \emph\{*hat*\} is converted to an IBM \emph\{*cat*.\}''
\end{doc-string}

\vspace{1em}
\noindent
\savebox{\funcname}{\noindent\texttt{shu-capture-doc-convert-args-to-latex }}
\usebox{\funcname}\emph{signature}
\index{shu-capture-doc-convert-args-to-latex} \hfill [Function]

\begin{doc-string}
Undocumented
\end{doc-string}

\vspace{1em}
\noindent
\savebox{\funcname}{\noindent\texttt{shu-capture-doc-convert-args-to-md }}
\usebox{\funcname}\emph{signature}
\index{shu-capture-doc-convert-args-to-md} \hfill [Function]

\begin{doc-string}
Undocumented
\end{doc-string}

\vspace{1em}
\noindent
\savebox{\funcname}{\noindent\texttt{shu-capture-enclose-doc-latex }}
\usebox{\funcname}
\index{shu-capture-enclose-doc-latex} \hfill [Function]

\begin{doc-string}
Enclose the doc-string with the appropriate begin / end pair for LaTex.
\end{doc-string}

\vspace{1em}
\noindent
\savebox{\funcname}{\noindent\texttt{shu-capture-enclose-doc-md }}
\usebox{\funcname}
\index{shu-capture-enclose-doc-md} \hfill [Function]

\begin{doc-string}
Enclose the doc-string with the appropriate begin / end pair for markdown.
\end{doc-string}

\vspace{1em}
\noindent
\savebox{\funcname}{\noindent\texttt{shu-capture-file-list }}
\usebox{\funcname}
\index{shu-capture-file-list} \hfill [Constant]

\begin{doc-string}
This is a list of all of the files in this repository from which documentation
should be extracted.
\end{doc-string}

\vspace{1em}
\noindent
\savebox{\funcname}{\noindent\texttt{shu-capture-finish-doc-string-latex }}
\usebox{\funcname}
\index{shu-capture-finish-doc-string-latex} \hfill [Command]

\begin{doc-string}
Function that executes last step in the conversion of a doc-string to
markdown.
\end{doc-string}

\vspace{1em}
\noindent
\savebox{\funcname}{\noindent\texttt{shu-capture-finish-doc-string-md }}
\usebox{\funcname}
\index{shu-capture-finish-doc-string-md} \hfill [Function]

\begin{doc-string}
Function that executes last step in the conversion of a doc-string to
markdown.
\end{doc-string}

\vspace{1em}
\noindent
\savebox{\funcname}{\noindent\texttt{shu-capture-func-type-name }}
\usebox{\funcname}\emph{attributes}
\index{shu-capture-func-type-name} \hfill [Command]

\begin{doc-string}
Return the name of the type ``Alias,'' ``Macro,'' ``Constant,'' ``Variable,'' or
``Function'' based on the \emph{attributes} passed in.
\end{doc-string}

\vspace{1em}
\noindent
\savebox{\funcname}{\noindent\texttt{shu-capture-get-args-as-alist }}
\usebox{\funcname}\emph{signature}
\index{shu-capture-get-args-as-alist} \hfill [Function]

\begin{doc-string}
\emph{signature} contains the function signature (both function name and arguments).
This function returns the arguments as an a-list in which all of the argument
names are the keys.  The special argument names ``\&optional'' and ``\&rest'', if
present, are not copied into the a-list.

For example, if \emph{signature} holds the following:

\begin{verbatim}
     do-something (with these things \&optional and \&rest others)
\end{verbatim}

an a-list is returned with the keys ``others,'' ``and,'' ``things,'' ``these,'' and
``with.''
\end{doc-string}

\vspace{1em}
\noindent
\savebox{\funcname}{\noindent\texttt{shu-capture-get-doc-string }}
\usebox{\funcname}\emph{eof}
\index{shu-capture-get-doc-string} \hfill [Command]

\begin{doc-string}
Enter with point positioned immediately after a function declaration.  Try to
fetch the associated doc string as follows:  1. Look for the first open or close
parenthesis.  2. Look for the first quote.  If the first parenthesis comes before
the first quote, then there is no doc string.  In the following function, there is
no doc string:

\begin{verbatim}
     (defun foo (name)
       (interactive ``sName?: ``))
\end{verbatim}

but if we do not notice that the first parenthesis comes before the first quote, then
we might think that there is a doc string that contains ``sName?: ``.

Return the doc string if there is one, nil otherwise.
\end{doc-string}

\vspace{1em}
\noindent
\savebox{\funcname}{\noindent\texttt{shu-capture-get-func-def }}
\usebox{\funcname}\emph{func-def} \emph{signature}
\index{shu-capture-get-func-def} \hfill [Macro]
\hspace*{\wd\funcname}\emph{description} \emph{alias}

\begin{doc-string}
Extract the information from the func-def
\end{doc-string}

\vspace{1em}
\noindent
\savebox{\funcname}{\noindent\texttt{shu-capture-get-func-def-alias }}
\usebox{\funcname}\emph{func-def} \emph{alias}
\index{shu-capture-get-func-def-alias} \hfill [Macro]

\begin{doc-string}
Extract the function alias from the func-def
\end{doc-string}

\vspace{1em}
\noindent
\savebox{\funcname}{\noindent\texttt{shu-capture-get-func-def-sig }}
\usebox{\funcname}\emph{func-def} \emph{signature}
\index{shu-capture-get-func-def-sig} \hfill [Macro]

\begin{doc-string}
Extract the function signature from the func-def
\end{doc-string}

\vspace{1em}
\noindent
\savebox{\funcname}{\noindent\texttt{shu-capture-get-name-and-args }}
\usebox{\funcname}\emph{signature} \emph{func-name}
\index{shu-capture-get-name-and-args} \hfill [Macro]
\hspace*{\wd\funcname}

\begin{doc-string}
Extract the function and the string of arguments from a whole signature that
includes both the function name and the arguments.  If \emph{signature} contains:

\begin{verbatim}
     ``do-something (to something)''
\end{verbatim}

The on return \emph{func-name} will hold ``do-something'' and \emph{args} will contain the
string ``to something)''.  If there are no arguments, \emph{args} will contain a string
of length zero.  If there is no function name, \emph{func-name} will contain a string
of length zero
\end{doc-string}

\vspace{1em}
\noindent
\savebox{\funcname}{\noindent\texttt{shu-capture-index-buffer }}
\usebox{\funcname}
\index{shu-capture-index-buffer} \hfill [Constant]

\begin{doc-string}
Name of the buffer into which the markdown index is written
\end{doc-string}

\vspace{1em}
\noindent
\savebox{\funcname}{\noindent\texttt{shu-capture-internal-all }}
\usebox{\funcname}\emph{file-list} \emph{capture-func}
\index{shu-capture-internal-all} \hfill [Function]

\begin{doc-string}
Visit all of the files in \emph{file-list}, invoking \emph{capture-func} on each
file to capture its documentation and turn it into either LaTex or
markdown.
\end{doc-string}

\vspace{1em}
\noindent
\savebox{\funcname}{\noindent\texttt{shu-capture-internal-convert-doc-string }}
\usebox{\funcname}\emph{signature}
\index{shu-capture-internal-convert-doc-string} \hfill [Function]
\hspace*{\wd\funcname}\emph{converters}

\begin{doc-string}
\emph{description} contains a doc string from a function definition (with leading
and trailing quotes removed).  \emph{converters} is an a-list of functions and strings as
follows:

\begin{verbatim}
      Key                              Value
      ---                              -----
      shu-capture-a-type-hdr           Function to format a section header
      shu-capture-a-type-func          Function to format a function signature
      shu-capture-a-type-buf           Function to format a buffer name
      shu-capture-a-type-arg           Function to format an argument name
      shu-capture-a-type-keywd         Function to format a key word
      shu-capture-a-type-doc-string    Function to finish formatting the doc string
      shu-capture-a-type-enclose-doc   Function to enclose doc string in begin / end
      shu-capture-a-type-before        String that starts a block of verbatim code
      shu-capture-a-type-after         String that ends a block of verbatim code
      shu-capture-a-type-open-quote    String that is an open quote
      shu-capture-a-type-close-quote   String that is a close quote
\end{verbatim}

This function turns escaped quotes into open and close quote strings, turns names
with leading and trailing asterisks (e.g., \emph\{**project-buffer**\}) into formatted buffer
names, turns upper case names that match any argument names into lower case,
formatted argument names.  This is an internal function of shu-capture-doc and
will likely crash if called with an invalid a-list.
\end{doc-string}

\vspace{1em}
\noindent
\savebox{\funcname}{\noindent\texttt{shu-capture-internal-doc }}
\usebox{\funcname}
\index{shu-capture-internal-doc} \hfill [Command]

\begin{doc-string}
Function that captures documentation for all instances of ``defun,'' ``defsubst,''
and ``defmacro.''
\end{doc-string}

\vspace{1em}
\noindent
\savebox{\funcname}{\noindent\texttt{shu-capture-keywd-optional }}
\usebox{\funcname}
\index{shu-capture-keywd-optional} \hfill [Constant]

\begin{doc-string}
The argument list keyword for an optional argument.
\end{doc-string}

\vspace{1em}
\noindent
\savebox{\funcname}{\noindent\texttt{shu-capture-keywd-rest }}
\usebox{\funcname}
\index{shu-capture-keywd-rest} \hfill [Constant]

\begin{doc-string}
The argument list keyword for a multiple optional arguments.
\end{doc-string}

\vspace{1em}
\noindent
\savebox{\funcname}{\noindent\texttt{shu-capture-keywd-to-latex }}
\usebox{\funcname}\emph{keywd-name}
\index{shu-capture-keywd-to-latex} \hfill [Function]

\begin{doc-string}
Convert a function argument key word in a doc-string or argument list
to LaTex.
\end{doc-string}

\vspace{1em}
\noindent
\savebox{\funcname}{\noindent\texttt{shu-capture-keywd-to-md }}
\usebox{\funcname}\emph{arg-name}
\index{shu-capture-keywd-to-md} \hfill [Function]

\begin{doc-string}
Convert a function argument key word in a doc-string or argument list
to markdown.
\end{doc-string}

\vspace{1em}
\noindent
\savebox{\funcname}{\noindent\texttt{shu-capture-latex }}
\usebox{\funcname}
\index{shu-capture-latex} \hfill [Command]

\begin{doc-string}
Capture all of the function and macro definitions in a .el source file and turn
them into a LaTex text that documents the functions and their doc strings.
\end{doc-string}

\vspace{1em}
\noindent
\savebox{\funcname}{\noindent\texttt{shu-capture-latex-arg-end }}
\usebox{\funcname}
\index{shu-capture-latex-arg-end} \hfill [Constant]

\begin{doc-string}
Define the latex string that is used to terminate an argument name.
\end{doc-string}

\vspace{1em}
\noindent
\savebox{\funcname}{\noindent\texttt{shu-capture-latex-arg-start }}
\usebox{\funcname}
\index{shu-capture-latex-arg-start} \hfill [Constant]

\begin{doc-string}
Define the latex string that is used to prepended to an argument name.
\end{doc-string}

\vspace{1em}
\noindent
\savebox{\funcname}{\noindent\texttt{shu-capture-latex-buf-end }}
\usebox{\funcname}
\index{shu-capture-latex-buf-end} \hfill [Constant]

\begin{doc-string}
Define the LaTex string that is used at the end of a buffer name or
any other name that has leading and trailing asterisks
\end{doc-string}

\vspace{1em}
\noindent
\savebox{\funcname}{\noindent\texttt{shu-capture-latex-buf-start }}
\usebox{\funcname}
\index{shu-capture-latex-buf-start} \hfill [Constant]

\begin{doc-string}
Define the LaTex string that is used in front of a buffer name or
any other name that has leading and trailing asterisks
\end{doc-string}

\vspace{1em}
\noindent
\savebox{\funcname}{\noindent\texttt{shu-capture-latex-close-quote }}
\usebox{\funcname}
\index{shu-capture-latex-close-quote} \hfill [Constant]

\begin{doc-string}
Define the LaTex string that is a close quote.
\end{doc-string}

\vspace{1em}
\noindent
\savebox{\funcname}{\noindent\texttt{shu-capture-latex-code-end }}
\usebox{\funcname}
\index{shu-capture-latex-code-end} \hfill [Constant]

\begin{doc-string}
Define the LaTex string that is at the end of a verbatim
code snippet.
\end{doc-string}

\vspace{1em}
\noindent
\savebox{\funcname}{\noindent\texttt{shu-capture-latex-code-start }}
\usebox{\funcname}
\index{shu-capture-latex-code-start} \hfill [Constant]

\begin{doc-string}
Define the LaTex string that is at the beginning of a verbatim
code snippet.
\end{doc-string}

\vspace{1em}
\noindent
\savebox{\funcname}{\noindent\texttt{shu-capture-latex-converters }}
\usebox{\funcname}
\index{shu-capture-latex-converters} \hfill [Constant]

\begin{doc-string}
This is the association list of functions and strings that is used to take an elisp
function and its associated doc string and convert it to LaTex.
\end{doc-string}

\vspace{1em}
\noindent
\savebox{\funcname}{\noindent\texttt{shu-capture-latex-doc-end }}
\usebox{\funcname}
\index{shu-capture-latex-doc-end} \hfill [Constant]

\begin{doc-string}
Define the LaTex string that ends a doc string.
\end{doc-string}

\vspace{1em}
\noindent
\savebox{\funcname}{\noindent\texttt{shu-capture-latex-doc-start }}
\usebox{\funcname}
\index{shu-capture-latex-doc-start} \hfill [Constant]

\begin{doc-string}
Define the LaTex string that starts a doc string.
\end{doc-string}

\vspace{1em}
\noindent
\savebox{\funcname}{\noindent\texttt{shu-capture-latex-keywd-end }}
\usebox{\funcname}
\index{shu-capture-latex-keywd-end} \hfill [Constant]

\begin{doc-string}
Define the latex string that is used to terminate an argument name.
\end{doc-string}

\vspace{1em}
\noindent
\savebox{\funcname}{\noindent\texttt{shu-capture-latex-keywd-start }}
\usebox{\funcname}
\index{shu-capture-latex-keywd-start} \hfill [Constant]

\begin{doc-string}
Define the latex string that is used to prepended to an argument name.
\end{doc-string}

\vspace{1em}
\noindent
\savebox{\funcname}{\noindent\texttt{shu-capture-latex-open-quote }}
\usebox{\funcname}
\index{shu-capture-latex-open-quote} \hfill [Constant]

\begin{doc-string}
Define the LaTex string that is an open quote.
\end{doc-string}

\vspace{1em}
\noindent
\savebox{\funcname}{\noindent\texttt{shu-capture-latex-section-end }}
\usebox{\funcname}
\index{shu-capture-latex-section-end} \hfill [Constant]

\begin{doc-string}
Define the LaTex tag that is used to identify the start of a section heading.
\end{doc-string}

\vspace{1em}
\noindent
\savebox{\funcname}{\noindent\texttt{shu-capture-latex-section-start }}
\usebox{\funcname}
\index{shu-capture-latex-section-start} \hfill [Constant]

\begin{doc-string}
Define the LaTex tag that is used to identify the start of a section heading.
\end{doc-string}

\vspace{1em}
\noindent
\savebox{\funcname}{\noindent\texttt{shu-capture-make-args-latex }}
\usebox{\funcname}\emph{func-name} \emph{markups}
\index{shu-capture-make-args-latex} \hfill [Function]
\hspace*{\wd\funcname}

\begin{doc-string}
\emph{func-name} is the name of the function, macro, alias, etc.  \emph{func-type} is a
string that represents the function type.  This will be part of the argument
display.  \emph{markups} is either nil or is a cons cell that points to two lists.  If
\emph{markups} is nil, the function has no arguments.  If \emph{markups} is non-nil, it is a
cons cell that points to two lists.  The car of \emph{markups} is a list of the lengths
of each argument before any markup was added to the argument.  If an argument
name is ``arg1,'' its length is 4 even though the length of the argument name
after markup is applied may be longer.  The cdr of \emph{markups} is a list of the
arguments with markup applied to them.
\end{doc-string}

\vspace{1em}
\noindent
\savebox{\funcname}{\noindent\texttt{shu-capture-make-args-md }}
\usebox{\funcname}\emph{func-name} \emph{markups}
\index{shu-capture-make-args-md} \hfill [Function]
\hspace*{\wd\funcname}\emph{section-converter}

\begin{doc-string}
\emph{func-name} is the name of the function, macro, alias, etc.  \emph{func-type} is a
string that represents the function type.  This will be part of the argument
display.  \emph{markups} is either nil or is a cons cell that points to two lists.  If
\emph{markups} is nil, the function has no arguments.  If \emph{markups} is non-nil, it is a
cons cell that points to two lists.  The car of \emph{markups} is a list of the lengths
of each argument before any markup was added to the argument.  If an argument
name is ``arg1,'' its length is 4 even though the length of the argument name
after markup is applied may be longer.  The cdr of \emph{markups} is a list of the
arguments with markup applied to them.  \emph{section-converter} is the function that
will turn a string into a section heading.
\end{doc-string}

\vspace{1em}
\noindent
\savebox{\funcname}{\noindent\texttt{shu-capture-make-latex-section }}
\usebox{\funcname}\emph{level} \emph{hdr}
\index{shu-capture-make-latex-section} \hfill [Function]

\begin{doc-string}
Turn \emph{hdr} into a LaTex section header of level \emph{level}, where 1 is a section,
2 a subsection, etc.  Return the LaTex string.
\end{doc-string}

\vspace{1em}
\noindent
\savebox{\funcname}{\noindent\texttt{shu-capture-make-md-section }}
\usebox{\funcname}\emph{level} \emph{hdr}
\index{shu-capture-make-md-section} \hfill [Function]

\begin{doc-string}
Turn \emph{hdr} into a markdown section header of level \emph{level}, where 1 is a section,
2 a subsection, etc.  Return the markdown string.  If level is one (major heading),
write a corresponding entry into the markdown table of contents buffer.
\end{doc-string}

\vspace{1em}
\noindent
\savebox{\funcname}{\noindent\texttt{shu-capture-md }}
\usebox{\funcname}
\index{shu-capture-md} \hfill [Command]

\begin{doc-string}
Capture all of the function and macro definitions in a .el source file and turn
them into markdown text that documents the functions and their doc strings.
\end{doc-string}

\vspace{1em}
\noindent
\savebox{\funcname}{\noindent\texttt{shu-capture-md-arg-delimiter }}
\usebox{\funcname}
\index{shu-capture-md-arg-delimiter} \hfill [Constant]

\begin{doc-string}
Define the markdown delimiter that is used to surround an argument name.
\end{doc-string}

\vspace{1em}
\noindent
\savebox{\funcname}{\noindent\texttt{shu-capture-md-buf-delimiter }}
\usebox{\funcname}
\index{shu-capture-md-buf-delimiter} \hfill [Constant]

\begin{doc-string}
Define the markdown delimiter that is used to surround a buffer name or
any other name that has leading and trailing asterisks
\end{doc-string}

\vspace{1em}
\noindent
\savebox{\funcname}{\noindent\texttt{shu-capture-md-code-delimiter }}
\usebox{\funcname}
\index{shu-capture-md-code-delimiter} \hfill [Constant]

\begin{doc-string}
Define the markdown delimiter that is used to surround a code snippet.
\end{doc-string}

\vspace{1em}
\noindent
\savebox{\funcname}{\noindent\texttt{shu-capture-md-converters }}
\usebox{\funcname}
\index{shu-capture-md-converters} \hfill [Constant]

\begin{doc-string}
This is the association list of functions and strings that is used to take an elisp
function and its associated doc string and convert it to markdown.
\end{doc-string}

\vspace{1em}
\noindent
\savebox{\funcname}{\noindent\texttt{shu-capture-md-keywd-delimiter }}
\usebox{\funcname}
\index{shu-capture-md-keywd-delimiter} \hfill [Constant]

\begin{doc-string}
Define the markdown delimiter that is used to surround an key word such as
``\&optional'' or ``\&rest''.
\end{doc-string}

\vspace{1em}
\noindent
\savebox{\funcname}{\noindent\texttt{shu-capture-md-quote-delimiter }}
\usebox{\funcname}
\index{shu-capture-md-quote-delimiter} \hfill [Constant]

\begin{doc-string}
Define the markdown delimiter that is used for open and close quote.
\end{doc-string}

\vspace{1em}
\noindent
\savebox{\funcname}{\noindent\texttt{shu-capture-md-section-delimiter }}
\usebox{\funcname}
\index{shu-capture-md-section-delimiter} \hfill [Constant]

\begin{doc-string}
Define the markdown delimiter that is used to identify a section.  This is separated
from the section name by a space.
\end{doc-string}

\vspace{1em}
\noindent
\savebox{\funcname}{\noindent\texttt{shu-capture-pre-code-in-doc }}
\usebox{\funcname}
\index{shu-capture-pre-code-in-doc} \hfill [Constant]

\begin{doc-string}
The a-list key value that identifies the function that converts characters in a
doc string right before the code snippets are captured.
\end{doc-string}

\vspace{1em}
\noindent
\savebox{\funcname}{\noindent\texttt{shu-capture-pre-code-latex }}
\usebox{\funcname}\emph{min-point} \emph{max-point}
\index{shu-capture-pre-code-latex} \hfill [Function]

\begin{doc-string}
Function that prepares a doc string to capture code snippets in LaTex.
Enter with \emph{min-point} and \emph{max-point} defining the region to be changed.
\emph{min-point} cannot change because all changes are made after it.  But
\emph{max-point} will change if replacements add extra characters.  Return the
new value of \emph{max-point} which takes into account the number of characters
added to the text.
\end{doc-string}

\vspace{1em}
\noindent
\savebox{\funcname}{\noindent\texttt{shu-capture-pre-code-md }}
\usebox{\funcname}\emph{min-point} \emph{max-point}
\index{shu-capture-pre-code-md} \hfill [Function]

\begin{doc-string}
Function that prepares a doc string to capture code snippets in markdown.
\end{doc-string}

\vspace{1em}
\noindent
\savebox{\funcname}{\noindent\texttt{shu-capture-set-func-def }}
\usebox{\funcname}\emph{func-def} \emph{signature}
\index{shu-capture-set-func-def} \hfill [Macro]
\hspace*{\wd\funcname}\emph{description}

\begin{doc-string}
Create a func-def to describe the function
\end{doc-string}

\vspace{1em}
\noindent
\savebox{\funcname}{\noindent\texttt{shu-capture-set-func-def-alias }}
\usebox{\funcname}\emph{func-def} \emph{signature}
\index{shu-capture-set-func-def-alias} \hfill [Macro]
\hspace*{\wd\funcname}\emph{description} \emph{alias}

\begin{doc-string}
Create a func-def to describe the function
\end{doc-string}

\vspace{1em}
\noindent
\savebox{\funcname}{\noindent\texttt{shu-capture-show-list }}
\usebox{\funcname}\emph{func-list} \emph{converters} \emph{buffer}
\index{shu-capture-show-list} \hfill [Function]
\hspace*{\wd\funcname}

\begin{doc-string}
\emph{func-list} is a list of function and macro definitions.  \emph{converters}
is an a-list of functions and strings as
follows:

\begin{verbatim}
      Key                              Value
      ---                              -----
      shu-capture-a-type-hdr           Function to format a section header
      shu-capture-a-type-func          Function to format a function signature
      shu-capture-a-type-buf           Function to format a buffer name
      shu-capture-a-type-arg           Function to format an argument name
      shu-capture-a-type-keywd         Function to format a key word
      shu-capture-a-type-doc-string    Function to finish formatting the doc string
      shu-capture-a-type-enclose-doc   Function to enclose doc string in begin / end
      shu-capture-a-type-before        String that starts a block of verbatim code
      shu-capture-a-type-after         String that ends a block of verbatim code
      shu-capture-a-type-open-quote    String that is an open quote
      shu-capture-a-type-close-quote   String that is a close quote
\end{verbatim}

This function goes through the list and uses the \emph{converters} to turn the set of
function definitions into either markdown or LaTex.
\end{doc-string}

\vspace{1em}
\noindent
\savebox{\funcname}{\noindent\texttt{shu-capture-show-list-md }}
\usebox{\funcname}\emph{func-list} \emph{buffer}
\index{shu-capture-show-list-md} \hfill [Function]

\begin{doc-string}
Show a list
\end{doc-string}

\vspace{1em}
\noindent
\savebox{\funcname}{\noindent\texttt{shu-capture-toc-buffer }}
\usebox{\funcname}
\index{shu-capture-toc-buffer} \hfill [Constant]

\begin{doc-string}
Name of the buffer into which the markdown yable of contents is written
\end{doc-string}

\vspace{1em}
\noindent
\savebox{\funcname}{\noindent\texttt{shu-capture-vars }}
\usebox{\funcname}\emph{func-list}
\index{shu-capture-vars} \hfill [Function]

\begin{doc-string}
Find the name and doc-string for instances of ``defvar'' or ``defconst.''
\end{doc-string}

\vspace{1em}
\noindent
\savebox{\funcname}{\noindent\texttt{shu-doc-internal-func-to-md }}
\usebox{\funcname}\emph{func-def}
\index{shu-doc-internal-func-to-md} \hfill [Function]

\begin{doc-string}
Take a function definition and turn it into a string of markdown text.
\end{doc-string}

\vspace{1em}
\noindent
\savebox{\funcname}{\noindent\texttt{shu-doc-internal-to-md }}
\usebox{\funcname}\emph{description}
\index{shu-doc-internal-to-md} \hfill [Function]

\begin{doc-string}
\emph{description} contains a doc string from a function definition (with leading
and trailing quotes removed).  This function turns escaped quotes into regular
(non-escaped) quotes and turns names with leading and trailing asterisks (e.g.,
\emph\{**project-count-buffer**\}) into short code blocks surrounded by back ticks.  It also
turns upper case names into lower case names surrounded by markdown ticks.
\end{doc-string}

\vspace{1em}
\noindent
\savebox{\funcname}{\noindent\texttt{shu-doc-sort-compare }}
\usebox{\funcname}\emph{lhs} \emph{rhs}
\index{shu-doc-sort-compare} \hfill [Function]

\begin{doc-string}
Compare two function names in a sort.
\end{doc-string}

\section{shu-cpp-general}


A collection of useful functions for dealing with C++ code


\subsection{List of functions by alias name}

A list of aliases and associated function names.



\vspace{1em}
\noindent
\savebox{\funcname}{\noindent\texttt{author }}
\usebox{\funcname}
\index{author} \hfill [Command]\\%
 (Function: shu-author)

\begin{doc-string}
Insert the doxygen author tag in an existing file.
\end{doc-string}

\vspace{1em}
\noindent
\savebox{\funcname}{\noindent\texttt{cdo }}
\usebox{\funcname}
\index{cdo} \hfill [Command]\\%
 (Function: shu-cdo)

\begin{doc-string}
Insert an empty do statement.
\end{doc-string}

\vspace{1em}
\noindent
\savebox{\funcname}{\noindent\texttt{celse }}
\usebox{\funcname}
\index{celse} \hfill [Command]\\%
 (Function: shu-celse)

\begin{doc-string}
Insert an empty else statement.
\end{doc-string}

\vspace{1em}
\noindent
\savebox{\funcname}{\noindent\texttt{cfor }}
\usebox{\funcname}
\index{cfor} \hfill [Command]\\%
 (Function: shu-cfor)

\begin{doc-string}
Insert an empty for statement.
\end{doc-string}

\vspace{1em}
\noindent
\savebox{\funcname}{\noindent\texttt{cif }}
\usebox{\funcname}
\index{cif} \hfill [Command]\\%
 (Function: shu-cif)

\begin{doc-string}
Insert an empty if statement.
\end{doc-string}

\vspace{1em}
\noindent
\savebox{\funcname}{\noindent\texttt{ck }}
\usebox{\funcname}\emph{start} \emph{end}
\index{ck} \hfill [Command]\\%
 (Function: shu-cpp-check-streaming-op)

\begin{doc-string}
Check a streaming operation.   Mark a region that contains a set of streaming
operators and invoke this function.  It will make sure that you have no unterminated
strings and that you are not missing any occurrences of $<$$<$.
\end{doc-string}

\vspace{1em}
\noindent
\savebox{\funcname}{\noindent\texttt{clc }}
\usebox{\funcname}
\index{clc} \hfill [Command]\\%
 (Function: shu-clc)

\begin{doc-string}
Place a skeleton Doxygen header definition at point.
\end{doc-string}

\vspace{1em}
\noindent
\savebox{\funcname}{\noindent\texttt{cother }}
\usebox{\funcname}
\index{cother} \hfill [Command]\\%
 (Function: shu-cother)

\begin{doc-string}
Visit a .cpp file from the corresponding .t.cpp or .h file.  If visiting a t.cpp or .h
file, invoke \emph{shu-cother} and you will be taken to the corresponding .cpp or .c file.
\end{doc-string}

\vspace{1em}
\noindent
\savebox{\funcname}{\noindent\texttt{cpp1-class }}
\usebox{\funcname}\emph{class-name}
\index{cpp1-class} \hfill [Command]\\%
 (Function: shu-cpp1-class)

\begin{doc-string}
Place a skeleton class definition in the current buffer at point.
\end{doc-string}

\vspace{1em}
\noindent
\savebox{\funcname}{\noindent\texttt{cpp2-class }}
\usebox{\funcname}\emph{class-name}
\index{cpp2-class} \hfill [Command]\\%
 (Function: shu-cpp2-class)

\begin{doc-string}
Place a skeleton class definition in the current buffer at point.
\end{doc-string}

\vspace{1em}
\noindent
\savebox{\funcname}{\noindent\texttt{creplace }}
\usebox{\funcname}
\index{creplace} \hfill [Command]\\%
 (Function: shu-creplace)

\begin{doc-string}
This function will replace the C++ string in which point is placed with the C++ string
in the kill ring.  The C++ string in the kill ring is expected to be a single string with
or without quotes.  The C++ string in which point is placed may have been split into
smaller substrings in order to avoid long lines.
\end{doc-string}

\vspace{1em}
\noindent
\savebox{\funcname}{\noindent\texttt{csplit }}
\usebox{\funcname}
\index{csplit} \hfill [Command]\\%
 (Function: shu-csplit)

\begin{doc-string}
Split a C++ string into multiple strings in order to keep the line length below a
certain minimum length, currently hard coded to column 76.
\end{doc-string}

\vspace{1em}
\noindent
\savebox{\funcname}{\noindent\texttt{cunsplit }}
\usebox{\funcname}
\index{cunsplit} \hfill [Command]\\%
 (Function: shu-cunsplit)

\begin{doc-string}
Undo the split that was done by csplit.
\end{doc-string}

\vspace{1em}
\noindent
\savebox{\funcname}{\noindent\texttt{cwhile }}
\usebox{\funcname}
\index{cwhile} \hfill [Command]\\%
 (Function: shu-cwhile)

\begin{doc-string}
Insert an empty while statement.
\end{doc-string}

\vspace{1em}
\noindent
\savebox{\funcname}{\noindent\texttt{dbx-malloc }}
\usebox{\funcname}
\index{dbx-malloc} \hfill [Command]\\%
 (Function: shu-dbx-summarize-malloc)

\begin{doc-string}
Go through the output of a dbx malloc dump and generate a summary.  dbx is
the AIX debugger.  It has a malloc command that goes through the heap and prints
one line for every allocated buffer.  Here is a sample of some of its output:

\begin{verbatim}
         ADDRESS         SIZE HEAP    ALLOCATOR
      0x30635678          680    0     YORKTOWN
      0x30635928          680    0     YORKTOWN
      0x30635bd8          680    0     YORKTOWN
\end{verbatim}

YORKTOWN is the name of the default allocator on AIX.  This function goes
through the malloc output and gets the number and sizes of all buffers
allocated.  This tells you how many buffers were allocated, the total number of
bytes allocated, and the total number of buffers allocated by size.  The output
is placed in a separate buffer called \emph\{**shu-aix-malloc**.\}
\end{doc-string}

\vspace{1em}
\noindent
\savebox{\funcname}{\noindent\texttt{dcc }}
\usebox{\funcname}
\index{dcc} \hfill [Command]\\%
 (Function: shu-dcc)

\begin{doc-string}
Place a skeleton Doxygen header definition at point.
\end{doc-string}

\vspace{1em}
\noindent
\savebox{\funcname}{\noindent\texttt{dce }}
\usebox{\funcname}
\index{dce} \hfill [Command]\\%
 (Function: shu-dce)

\begin{doc-string}
Place a skeleton Doxygen header definition at point.
\end{doc-string}

\vspace{1em}
\noindent
\savebox{\funcname}{\noindent\texttt{dox-brief }}
\usebox{\funcname}
\index{dox-brief} \hfill [Command]\\%
 (Function: shu-dox-brief)

\begin{doc-string}
Place a skeleton Doxygen header definition at point.
\end{doc-string}

\vspace{1em}
\noindent
\savebox{\funcname}{\noindent\texttt{dox-cbt }}
\usebox{\funcname}
\index{dox-cbt} \hfill [Command]\\%
 (Function: shu-dox-cbt)

\begin{doc-string}
Convert a section of comments delimited by //! into Doxygen brief format.
\end{doc-string}

\vspace{1em}
\noindent
\savebox{\funcname}{\noindent\texttt{dox-cvt }}
\usebox{\funcname}
\index{dox-cvt} \hfill [Command]\\%
 (Function: shu-dox-cvt)

\begin{doc-string}
Convert a section of comments delimited by // into Doxygen format.
\end{doc-string}

\vspace{1em}
\noindent
\savebox{\funcname}{\noindent\texttt{dox2-hdr }}
\usebox{\funcname}
\index{dox2-hdr} \hfill [Command]\\%
 (Function: shu-dox2-hdr)

\begin{doc-string}
Place a skeleton Doxygen header definition at point.
\end{doc-string}

\vspace{1em}
\noindent
\savebox{\funcname}{\noindent\texttt{drc }}
\usebox{\funcname}
\index{drc} \hfill [Command]\\%
 (Function: shu-drc)

\begin{doc-string}
Place a skeleton Doxygen header definition at point.
\end{doc-string}

\vspace{1em}
\noindent
\savebox{\funcname}{\noindent\texttt{get-set }}
\usebox{\funcname}
\index{get-set} \hfill [Command]\\%
 (Function: shu-get-set)

\begin{doc-string}
Generate get and set functions for an instance variable in a C++ class.
Position the cursor ahead of the Doxygen comment above the variable.  The get
and set functions will be placed in the buffer \emph\{*get-set*.\}
\end{doc-string}

\vspace{1em}
\noindent
\savebox{\funcname}{\noindent\texttt{getters }}
\usebox{\funcname}\emph{start} \emph{end}
\index{getters} \hfill [Command]\\%
 (Function: shu-getters)

\begin{doc-string}
Mark a region in a file that contains C++ instance variable declarations.
This function will create get and set functions for all of the instance
variables.
\end{doc-string}

\vspace{1em}
\noindent
\savebox{\funcname}{\noindent\texttt{hother }}
\usebox{\funcname}
\index{hother} \hfill [Command]\\%
 (Function: shu-hother)

\begin{doc-string}
Visit a .h file from the corresponding .cpp or t.cpp file.  If visiting a .cpp or
t.cpp file, invoke \emph{shu-hother} and you will be taken to the corresponding .h file.
\end{doc-string}

\vspace{1em}
\noindent
\savebox{\funcname}{\noindent\texttt{new-c-class }}
\usebox{\funcname}
\index{new-c-class} \hfill [Command]\\%
 (Function: shu-new-c-class)

\begin{doc-string}
Place a skeleton class definition in the current buffer at point.
\end{doc-string}

\vspace{1em}
\noindent
\savebox{\funcname}{\noindent\texttt{new-c-file }}
\usebox{\funcname}
\index{new-c-file} \hfill [Command]\\%
 (Function: shu-new-c-file)

\begin{doc-string}
Generate a skeleton code file for a C or C++ file.
\end{doc-string}

\vspace{1em}
\noindent
\savebox{\funcname}{\noindent\texttt{new-h-file }}
\usebox{\funcname}
\index{new-h-file} \hfill [Command]\\%
 (Function: shu-new-h-file)

\begin{doc-string}
Generate a skeleton header file for C or C++ file.
\end{doc-string}

\vspace{1em}
\noindent
\savebox{\funcname}{\noindent\texttt{new-x-file }}
\usebox{\funcname}
\index{new-x-file} \hfill [Command]\\%
 (Function: shu-new-x-file)

\begin{doc-string}
Generate a skeleton Doxygen \\file directive.
\end{doc-string}

\vspace{1em}
\noindent
\savebox{\funcname}{\noindent\texttt{operators }}
\usebox{\funcname}\emph{class-name}
\index{operators} \hfill [Command]\\%
 (Function: shu-operators)

\begin{doc-string}
Place skeletons of all of the standard c++ operator functions at point.
\end{doc-string}

\vspace{1em}
\noindent
\savebox{\funcname}{\noindent\texttt{other }}
\usebox{\funcname}
\index{other} \hfill [Command]\\%
 (Function: shu-other)

\begin{doc-string}
Visit an h file from a c file or a c file from an h file
If visiting a .h file, invoke \emph{shu-other} and you will be taken to the
.c or .cpp file.  If visiting a .c or .cpp file, invoke other and you
will be taken to the corresponding .h file
\end{doc-string}

\vspace{1em}
\noindent
\savebox{\funcname}{\noindent\texttt{qualify-bsl }}
\usebox{\funcname}
\index{qualify-bsl} \hfill [Command]\\%
 (Function: shu-qualify-namespace-bsl)

\begin{doc-string}
Add ``bsl'' namespace qualifier to some of the classes in ``bsl''.  Return the
count of class names changed.
\end{doc-string}

\vspace{1em}
\noindent
\savebox{\funcname}{\noindent\texttt{qualify-class }}
\usebox{\funcname}
\index{qualify-class} \hfill [Command]\\%
 (Function: shu-interactive-qualify-class-name)

\begin{doc-string}
Interactively call \emph{shu-qualify-class-name} to find all instances of a class name and
add a namespace qualifier to it.  First prompt is for the class name.  If a fully qualified
class name is supplied, then the given namespace is applied to the class name.  If the name
supplied is not a namespace qualified class name, then a second prompt is given to read the
namespace.
This is intended to help rescue code that has one or more ``using namespace''
directives in it.  The problem with ``using namespace'' is that you now have
class names from other namespaces with no easy way to identify the namespace
to which they belong.  The best thing to do is get rid of the ``using
namespace'' statements and explicitly qualify the class names.  But if you
use a simple replace to do that, you will qualify variable names that resemble
class names as well as class names that are already qualified.  This function
only adds a namespace to a class name that does not already have a namespace
qualifier.
\end{doc-string}

\vspace{1em}
\noindent
\savebox{\funcname}{\noindent\texttt{qualify-std }}
\usebox{\funcname}
\index{qualify-std} \hfill [Command]\\%
 (Function: shu-qualify-namespace-std)

\begin{doc-string}
Add ``std'' namespace qualifier to some of the classes in ``std''.  Return the
count of class names changed.
\end{doc-string}

\vspace{1em}
\noindent
\savebox{\funcname}{\noindent\texttt{set-default-namespace }}
\usebox{\funcname}\emph{name}
\index{set-default-namespace} \hfill [Command]\\%
 (Function: shu-set-default-namespace)

\begin{doc-string}
Undocumented
\end{doc-string}

\vspace{1em}
\noindent
\savebox{\funcname}{\noindent\texttt{tother }}
\usebox{\funcname}
\index{tother} \hfill [Command]\\%
 (Function: shu-tother)

\begin{doc-string}
Visit a t.cpp file from the corresponding .cpp or .h file.  If visiting a .c or
.cpp file, invoke \emph{shu-tother} and you will be taken to the corresponding .t.cpp
file.
\end{doc-string}

\subsection{List of functions and variables}

List of functions and variable definitions in this package.



\vspace{1em}
\noindent
\savebox{\funcname}{\noindent\texttt{shu-add-cpp-base-types }}
\usebox{\funcname}\emph{ntypes}
\index{shu-add-cpp-base-types} \hfill [Function]

\begin{doc-string}
Add one or more data types to the list of C++ native data types defined in shu-cpp-base-types
in shu-cpp-general.el.  Argument may be a single type in a string or a list of strings.
This modifies shu-cpp-base-types.
\end{doc-string}

\vspace{1em}
\noindent
\savebox{\funcname}{\noindent\texttt{shu-aix-show-malloc-list }}
\usebox{\funcname}\emph{mlist} \emph{gb}
\index{shu-aix-show-malloc-list} \hfill [Function]

\begin{doc-string}
Print the number of buffers allocated by size from an AIX dbx malloc command.
\end{doc-string}

\vspace{1em}
\noindent
\savebox{\funcname}{\noindent\texttt{shu-attr-name }}
\usebox{\funcname}
\index{shu-attr-name} \hfill [Variable]

\begin{doc-string}
The name of an attribute.
\end{doc-string}

\vspace{1em}
\noindent
\savebox{\funcname}{\noindent\texttt{shu-author }}
\usebox{\funcname}
\index{shu-author} \hfill [Command]\\%
 (Alias: author)

\begin{doc-string}
Insert the doxygen author tag in an existing file.
\end{doc-string}

\vspace{1em}
\noindent
\savebox{\funcname}{\noindent\texttt{shu-cdo }}
\usebox{\funcname}
\index{shu-cdo} \hfill [Command]\\%
 (Alias: cdo)

\begin{doc-string}
Insert an empty do statement.
\end{doc-string}

\vspace{1em}
\noindent
\savebox{\funcname}{\noindent\texttt{shu-celse }}
\usebox{\funcname}
\index{shu-celse} \hfill [Command]\\%
 (Alias: celse)

\begin{doc-string}
Insert an empty else statement.
\end{doc-string}

\vspace{1em}
\noindent
\savebox{\funcname}{\noindent\texttt{shu-cfor }}
\usebox{\funcname}
\index{shu-cfor} \hfill [Command]\\%
 (Alias: cfor)

\begin{doc-string}
Insert an empty for statement.
\end{doc-string}

\vspace{1em}
\noindent
\savebox{\funcname}{\noindent\texttt{shu-cif }}
\usebox{\funcname}
\index{shu-cif} \hfill [Command]\\%
 (Alias: cif)

\begin{doc-string}
Insert an empty if statement.
\end{doc-string}

\vspace{1em}
\noindent
\savebox{\funcname}{\noindent\texttt{shu-clc }}
\usebox{\funcname}
\index{shu-clc} \hfill [Command]\\%
 (Alias: clc)

\begin{doc-string}
Place a skeleton Doxygen header definition at point.
\end{doc-string}

\vspace{1em}
\noindent
\savebox{\funcname}{\noindent\texttt{shu-cother }}
\usebox{\funcname}
\index{shu-cother} \hfill [Command]\\%
 (Alias: cother)

\begin{doc-string}
Visit a .cpp file from the corresponding .t.cpp or .h file.  If visiting a t.cpp or .h
file, invoke \emph{shu-cother} and you will be taken to the corresponding .cpp or .c file.
\end{doc-string}

\vspace{1em}
\noindent
\savebox{\funcname}{\noindent\texttt{shu-cpp-base-types }}
\usebox{\funcname}
\index{shu-cpp-base-types} \hfill [Constant]

\begin{doc-string}
A list of all of the base types in C and C++.  This may be modified by shu-add-cpp-base-types
\end{doc-string}

\vspace{1em}
\noindent
\savebox{\funcname}{\noindent\texttt{shu-cpp-check-streaming-op }}
\usebox{\funcname}\emph{start} \emph{end}
\index{shu-cpp-check-streaming-op} \hfill [Command]\\%
 (Alias: ck)

\begin{doc-string}
Check a streaming operation.   Mark a region that contains a set of streaming
operators and invoke this function.  It will make sure that you have no unterminated
strings and that you are not missing any occurrences of $<$$<$.
\end{doc-string}

\vspace{1em}
\noindent
\savebox{\funcname}{\noindent\texttt{shu-cpp-general-set-alias }}
\usebox{\funcname}
\index{shu-cpp-general-set-alias} \hfill [Function]

\begin{doc-string}
Set the common alias names for the functions in shu-cpp-general.
These are generally the same as the function names with the leading
shu- prefix removed.
\end{doc-string}

\vspace{1em}
\noindent
\savebox{\funcname}{\noindent\texttt{shu-cpp-internal-stream-check }}
\usebox{\funcname}\emph{token-list}
\index{shu-cpp-internal-stream-check} \hfill [Function]

\begin{doc-string}
Take a list of tokens found in a C++ streaming operation and check to
ensure that every other token is a $<$$<$ operator.  Two adjacent occurrences of $<$$<$
represent an extraneous $<$$<$ operator.  Two adjacent occurrences of tokens that
are not $<$$<$ represent a missing $<$$<$ operator.
\end{doc-string}

\vspace{1em}
\noindent
\savebox{\funcname}{\noindent\texttt{shu-cpp-is-enclosing-op }}
\usebox{\funcname}\emph{op}
\index{shu-cpp-is-enclosing-op} \hfill [Function]

\begin{doc-string}
Undocumented
\end{doc-string}

\vspace{1em}
\noindent
\savebox{\funcname}{\noindent\texttt{shu-cpp-member-prefix }}
\usebox{\funcname}
\index{shu-cpp-member-prefix} \hfill [Variable]

\begin{doc-string}
The character string that is used as the prefix to member variables of a C++ class.
This is used by shu-internal-get-set when generating getters and setters for a class.
\end{doc-string}

\vspace{1em}
\noindent
\savebox{\funcname}{\noindent\texttt{shu-cpp-qualify-classes }}
\usebox{\funcname}\emph{class-list} \emph{namespace}
\index{shu-cpp-qualify-classes} \hfill [Function]
\hspace*{\wd\funcname}\emph{buffer}

\begin{doc-string}
Repeatedly call \emph{shu-qualify-class-name} for all class names in \emph{class-list}.
\emph{namespace} is either the name of a single namespace to apply to all classes
in \emph{class-list} or is a list of namespaces each of which has a one to one
correspondence with a class name in \emph{class-list}.  The optional \emph{buffer}
argument may be a buffer in which the actions are recorded.  Return the
number of names changed.
\end{doc-string}

\vspace{1em}
\noindent
\savebox{\funcname}{\noindent\texttt{shu-cpp1-class }}
\usebox{\funcname}\emph{class-name}
\index{shu-cpp1-class} \hfill [Command]\\%
 (Alias: cpp1-class)

\begin{doc-string}
Place a skeleton class definition in the current buffer at point.
\end{doc-string}

\vspace{1em}
\noindent
\savebox{\funcname}{\noindent\texttt{shu-cpp2-class }}
\usebox{\funcname}\emph{class-name}
\index{shu-cpp2-class} \hfill [Command]\\%
 (Alias: cpp2-class)

\begin{doc-string}
Place a skeleton class definition in the current buffer at point.
\end{doc-string}

\vspace{1em}
\noindent
\savebox{\funcname}{\noindent\texttt{shu-creplace }}
\usebox{\funcname}
\index{shu-creplace} \hfill [Command]\\%
 (Alias: creplace)

\begin{doc-string}
This function will replace the C++ string in which point is placed with the C++ string
in the kill ring.  The C++ string in the kill ring is expected to be a single string with
or without quotes.  The C++ string in which point is placed may have been split into
smaller substrings in order to avoid long lines.
\end{doc-string}

\vspace{1em}
\noindent
\savebox{\funcname}{\noindent\texttt{shu-csplit }}
\usebox{\funcname}
\index{shu-csplit} \hfill [Command]\\%
 (Alias: csplit)

\begin{doc-string}
Split a C++ string into multiple strings in order to keep the line length below a
certain minimum length, currently hard coded to column 76.
\end{doc-string}

\vspace{1em}
\noindent
\savebox{\funcname}{\noindent\texttt{shu-cunsplit }}
\usebox{\funcname}
\index{shu-cunsplit} \hfill [Command]\\%
 (Alias: cunsplit)

\begin{doc-string}
Undo the split that was done by csplit.
\end{doc-string}

\vspace{1em}
\noindent
\savebox{\funcname}{\noindent\texttt{shu-cwhile }}
\usebox{\funcname}
\index{shu-cwhile} \hfill [Command]\\%
 (Alias: cwhile)

\begin{doc-string}
Insert an empty while statement.
\end{doc-string}

\vspace{1em}
\noindent
\savebox{\funcname}{\noindent\texttt{shu-dbx-summarize-malloc }}
\usebox{\funcname}
\index{shu-dbx-summarize-malloc} \hfill [Command]\\%
 (Alias: dbx-malloc)

\begin{doc-string}
Go through the output of a dbx malloc dump and generate a summary.  dbx is
the AIX debugger.  It has a malloc command that goes through the heap and prints
one line for every allocated buffer.  Here is a sample of some of its output:

\begin{verbatim}
         ADDRESS         SIZE HEAP    ALLOCATOR
      0x30635678          680    0     YORKTOWN
      0x30635928          680    0     YORKTOWN
      0x30635bd8          680    0     YORKTOWN
\end{verbatim}

YORKTOWN is the name of the default allocator on AIX.  This function goes
through the malloc output and gets the number and sizes of all buffers
allocated.  This tells you how many buffers were allocated, the total number of
bytes allocated, and the total number of buffers allocated by size.  The output
is placed in a separate buffer called \emph\{**shu-aix-malloc**.\}
\end{doc-string}

\vspace{1em}
\noindent
\savebox{\funcname}{\noindent\texttt{shu-dcc }}
\usebox{\funcname}
\index{shu-dcc} \hfill [Command]\\%
 (Alias: dcc)

\begin{doc-string}
Place a skeleton Doxygen header definition at point.
\end{doc-string}

\vspace{1em}
\noindent
\savebox{\funcname}{\noindent\texttt{shu-dce }}
\usebox{\funcname}
\index{shu-dce} \hfill [Command]\\%
 (Alias: dce)

\begin{doc-string}
Place a skeleton Doxygen header definition at point.
\end{doc-string}

\vspace{1em}
\noindent
\savebox{\funcname}{\noindent\texttt{shu-dox-brief }}
\usebox{\funcname}
\index{shu-dox-brief} \hfill [Command]\\%
 (Alias: dox-brief)

\begin{doc-string}
Place a skeleton Doxygen header definition at point.
\end{doc-string}

\vspace{1em}
\noindent
\savebox{\funcname}{\noindent\texttt{shu-dox-cbt }}
\usebox{\funcname}
\index{shu-dox-cbt} \hfill [Command]\\%
 (Alias: dox-cbt)

\begin{doc-string}
Convert a section of comments delimited by //! into Doxygen brief format.
\end{doc-string}

\vspace{1em}
\noindent
\savebox{\funcname}{\noindent\texttt{shu-dox-cvt }}
\usebox{\funcname}
\index{shu-dox-cvt} \hfill [Command]\\%
 (Alias: dox-cvt)

\begin{doc-string}
Convert a section of comments delimited by // into Doxygen format.
\end{doc-string}

\vspace{1em}
\noindent
\savebox{\funcname}{\noindent\texttt{shu-dox-hdr }}
\usebox{\funcname}
\index{shu-dox-hdr} \hfill [Command]

\begin{doc-string}
Place a skeleton Doxygen header definition at point.
\end{doc-string}

\vspace{1em}
\noindent
\savebox{\funcname}{\noindent\texttt{shu-dox2-hdr }}
\usebox{\funcname}
\index{shu-dox2-hdr} \hfill [Command]\\%
 (Alias: dox2-hdr)

\begin{doc-string}
Place a skeleton Doxygen header definition at point.
\end{doc-string}

\vspace{1em}
\noindent
\savebox{\funcname}{\noindent\texttt{shu-drc }}
\usebox{\funcname}
\index{shu-drc} \hfill [Command]\\%
 (Alias: drc)

\begin{doc-string}
Place a skeleton Doxygen header definition at point.
\end{doc-string}

\vspace{1em}
\noindent
\savebox{\funcname}{\noindent\texttt{shu-emit-get }}
\usebox{\funcname}
\index{shu-emit-get} \hfill [Function]

\begin{doc-string}
Undocumented
\end{doc-string}

\vspace{1em}
\noindent
\savebox{\funcname}{\noindent\texttt{shu-emit-set }}
\usebox{\funcname}\emph{arg}
\index{shu-emit-set} \hfill [Function]

\begin{doc-string}
Undocumented
\end{doc-string}

\vspace{1em}
\noindent
\savebox{\funcname}{\noindent\texttt{shu-gen-return-ptr }}
\usebox{\funcname}
\index{shu-gen-return-ptr} \hfill [Function]

\begin{doc-string}
Undocumented
\end{doc-string}

\vspace{1em}
\noindent
\savebox{\funcname}{\noindent\texttt{shu-get-set }}
\usebox{\funcname}
\index{shu-get-set} \hfill [Command]\\%
 (Alias: get-set)

\begin{doc-string}
Generate get and set functions for an instance variable in a C++ class.
Position the cursor ahead of the Doxygen comment above the variable.  The get
and set functions will be placed in the buffer \emph\{*get-set*.\}
\end{doc-string}

\vspace{1em}
\noindent
\savebox{\funcname}{\noindent\texttt{shu-getters }}
\usebox{\funcname}\emph{start} \emph{end}
\index{shu-getters} \hfill [Command]\\%
 (Alias: getters)

\begin{doc-string}
Mark a region in a file that contains C++ instance variable declarations.
This function will create get and set functions for all of the instance
variables.
\end{doc-string}

\vspace{1em}
\noindent
\savebox{\funcname}{\noindent\texttt{shu-hother }}
\usebox{\funcname}
\index{shu-hother} \hfill [Command]\\%
 (Alias: hother)

\begin{doc-string}
Visit a .h file from the corresponding .cpp or t.cpp file.  If visiting a .cpp or
t.cpp file, invoke \emph{shu-hother} and you will be taken to the corresponding .h file.
\end{doc-string}

\vspace{1em}
\noindent
\savebox{\funcname}{\noindent\texttt{shu-interactive-qualify-class-name }}
\usebox{\funcname}
\index{shu-interactive-qualify-class-name} \hfill [Command]\\%
 (Alias: qualify-class)

\begin{doc-string}
Interactively call \emph{shu-qualify-class-name} to find all instances of a class name and
add a namespace qualifier to it.  First prompt is for the class name.  If a fully qualified
class name is supplied, then the given namespace is applied to the class name.  If the name
supplied is not a namespace qualified class name, then a second prompt is given to read the
namespace.
This is intended to help rescue code that has one or more ``using namespace''
directives in it.  The problem with ``using namespace'' is that you now have
class names from other namespaces with no easy way to identify the namespace
to which they belong.  The best thing to do is get rid of the ``using
namespace'' statements and explicitly qualify the class names.  But if you
use a simple replace to do that, you will qualify variable names that resemble
class names as well as class names that are already qualified.  This function
only adds a namespace to a class name that does not already have a namespace
qualifier.
\end{doc-string}

\vspace{1em}
\noindent
\savebox{\funcname}{\noindent\texttt{shu-internal-cpp2-class }}
\usebox{\funcname}\emph{class-name}
\index{shu-internal-cpp2-class} \hfill [Function]

\begin{doc-string}
Place a skeleton class definition in the current buffer at point.
\end{doc-string}

\vspace{1em}
\noindent
\savebox{\funcname}{\noindent\texttt{shu-internal-get-set }}
\usebox{\funcname}\emph{comment} \emph{shu-lc-comment}
\index{shu-internal-get-set} \hfill [Command]

\begin{doc-string}
Generate get and set functions for an instance variable in a C++ class.
\end{doc-string}

\vspace{1em}
\noindent
\savebox{\funcname}{\noindent\texttt{shu-is-const }}
\usebox{\funcname}
\index{shu-is-const} \hfill [Variable]

\begin{doc-string}
Set true if the C++ data member we are working is declared to be const.
\end{doc-string}

\vspace{1em}
\noindent
\savebox{\funcname}{\noindent\texttt{shu-lc-comment }}
\usebox{\funcname}
\index{shu-lc-comment} \hfill [Variable]

\begin{doc-string}
Comment string with the first letter downcased.
\end{doc-string}

\vspace{1em}
\noindent
\savebox{\funcname}{\noindent\texttt{shu-make-padded-line }}
\usebox{\funcname}\emph{line} \emph{tlen}
\index{shu-make-padded-line} \hfill [Function]

\begin{doc-string}
Add sufficient spaces to make \emph{line} the length \emph{tlen}.
\end{doc-string}

\vspace{1em}
\noindent
\savebox{\funcname}{\noindent\texttt{shu-nc-vtype }}
\usebox{\funcname}
\index{shu-nc-vtype} \hfill [Variable]

\begin{doc-string}
Set true if the C++ data member we are working is declared to be non-const.
\end{doc-string}

\vspace{1em}
\noindent
\savebox{\funcname}{\noindent\texttt{shu-new-c-class }}
\usebox{\funcname}
\index{shu-new-c-class} \hfill [Command]\\%
 (Alias: new-c-class)

\begin{doc-string}
Place a skeleton class definition in the current buffer at point.
\end{doc-string}

\vspace{1em}
\noindent
\savebox{\funcname}{\noindent\texttt{shu-new-c-file }}
\usebox{\funcname}
\index{shu-new-c-file} \hfill [Command]\\%
 (Alias: new-c-file)

\begin{doc-string}
Generate a skeleton code file for a C or C++ file.
\end{doc-string}

\vspace{1em}
\noindent
\savebox{\funcname}{\noindent\texttt{shu-new-h-file }}
\usebox{\funcname}
\index{shu-new-h-file} \hfill [Command]\\%
 (Alias: new-h-file)

\begin{doc-string}
Generate a skeleton header file for C or C++ file.
\end{doc-string}

\vspace{1em}
\noindent
\savebox{\funcname}{\noindent\texttt{shu-new-x-file }}
\usebox{\funcname}
\index{shu-new-x-file} \hfill [Command]\\%
 (Alias: new-x-file)

\begin{doc-string}
Generate a skeleton Doxygen \\file directive.
\end{doc-string}

\vspace{1em}
\noindent
\savebox{\funcname}{\noindent\texttt{shu-operators }}
\usebox{\funcname}\emph{class-name}
\index{shu-operators} \hfill [Command]\\%
 (Alias: operators)

\begin{doc-string}
Place skeletons of all of the standard c++ operator functions at point.
\end{doc-string}

\vspace{1em}
\noindent
\savebox{\funcname}{\noindent\texttt{shu-other }}
\usebox{\funcname}
\index{shu-other} \hfill [Command]\\%
 (Alias: other)

\begin{doc-string}
Visit an h file from a c file or a c file from an h file
If visiting a .h file, invoke \emph{shu-other} and you will be taken to the
.c or .cpp file.  If visiting a .c or .cpp file, invoke other and you
will be taken to the corresponding .h file
\end{doc-string}

\vspace{1em}
\noindent
\savebox{\funcname}{\noindent\texttt{shu-qualify-class-name }}
\usebox{\funcname}\emph{target-name} \emph{namespace}
\index{shu-qualify-class-name} \hfill [Function]

\begin{doc-string}
Find all instances of the class name \emph{target-name} and add an explicit namespace
qualifier \emph{namespace}.  If the \emph{target-name} is ``Mumble'' and the \emph{namespace} is
``abcd'', then ``Mumble'' becomes ``abcd::Mumble''.  But variable names such
as ``d\_Mumble'' or ``MumbleIn'' remain unchanged and already qualified class
names remain unchanged.
This is intended to help rescue code that has one or more ``using namespace''
directives in it.  The problem with ``using namespace'' is that you now have
class names from other namespaces with no easy way to identify the namespace
to which they belong.  The best thing to do is get rid of the ``using
namespace'' statements and explicitly qualify the class names.  But if you
use a simple replace to do that, you will qualify variable names that resemble
class names as well as class names that are already qualified.  This function
only adds a namespace to a class name that does not already have a namespace
qualifier.
\end{doc-string}

\vspace{1em}
\noindent
\savebox{\funcname}{\noindent\texttt{shu-qualify-namespace-bsl }}
\usebox{\funcname}
\index{shu-qualify-namespace-bsl} \hfill [Command]\\%
 (Alias: qualify-bsl)

\begin{doc-string}
Add ``bsl'' namespace qualifier to some of the classes in ``bsl''.  Return the
count of class names changed.
\end{doc-string}

\vspace{1em}
\noindent
\savebox{\funcname}{\noindent\texttt{shu-qualify-namespace-std }}
\usebox{\funcname}
\index{shu-qualify-namespace-std} \hfill [Command]\\%
 (Alias: qualify-std)

\begin{doc-string}
Add ``std'' namespace qualifier to some of the classes in ``std''.  Return the
count of class names changed.
\end{doc-string}

\vspace{1em}
\noindent
\savebox{\funcname}{\noindent\texttt{shu-return-ptr }}
\usebox{\funcname}
\index{shu-return-ptr} \hfill [Function]

\begin{doc-string}
Undocumented
\end{doc-string}

\vspace{1em}
\noindent
\savebox{\funcname}{\noindent\texttt{shu-return-ref }}
\usebox{\funcname}
\index{shu-return-ref} \hfill [Function]

\begin{doc-string}
Undocumented
\end{doc-string}

\vspace{1em}
\noindent
\savebox{\funcname}{\noindent\texttt{shu-s-mode-find-long-line }}
\usebox{\funcname}
\index{shu-s-mode-find-long-line} \hfill [Command]

\begin{doc-string}
Place point in column 79 of the next line whose length exceeds 79 characters.
No movement occurs if no lines, starting with the current position, exceed 79
characters in length.
\end{doc-string}

\vspace{1em}
\noindent
\savebox{\funcname}{\noindent\texttt{shu-set-author }}
\usebox{\funcname}\emph{name}
\index{shu-set-author} \hfill [Command]

\begin{doc-string}
Undocumented
\end{doc-string}

\vspace{1em}
\noindent
\savebox{\funcname}{\noindent\texttt{shu-set-default-global-namespace }}
\usebox{\funcname}\emph{name}
\index{shu-set-default-global-namespace} \hfill [Command]

\begin{doc-string}
Undocumented
\end{doc-string}

\vspace{1em}
\noindent
\savebox{\funcname}{\noindent\texttt{shu-set-default-namespace }}
\usebox{\funcname}\emph{name}
\index{shu-set-default-namespace} \hfill [Command]\\%
 (Alias: set-default-namespace)

\begin{doc-string}
Undocumented
\end{doc-string}

\vspace{1em}
\noindent
\savebox{\funcname}{\noindent\texttt{shu-set-obj }}
\usebox{\funcname}
\index{shu-set-obj} \hfill [Function]

\begin{doc-string}
Undocumented
\end{doc-string}

\vspace{1em}
\noindent
\savebox{\funcname}{\noindent\texttt{shu-set-ptr }}
\usebox{\funcname}
\index{shu-set-ptr} \hfill [Function]

\begin{doc-string}
Undocumented
\end{doc-string}

\vspace{1em}
\noindent
\savebox{\funcname}{\noindent\texttt{shu-tother }}
\usebox{\funcname}
\index{shu-tother} \hfill [Command]\\%
 (Alias: tother)

\begin{doc-string}
Visit a t.cpp file from the corresponding .cpp or .h file.  If visiting a .c or
.cpp file, invoke \emph{shu-tother} and you will be taken to the corresponding .t.cpp
file.
\end{doc-string}

\vspace{1em}
\noindent
\savebox{\funcname}{\noindent\texttt{shu-var-name }}
\usebox{\funcname}
\index{shu-var-name} \hfill [Variable]

\begin{doc-string}
The variable name that corresponds to an attribute name.
\end{doc-string}

\section{shu-cpp-misc}


A collection of useful functions for dealing with C++ code


\subsection{List of functions by alias name}

A list of aliases and associated function names.



\vspace{1em}
\noindent
\savebox{\funcname}{\noindent\texttt{acgen }}
\usebox{\funcname}\emph{class-name}
\index{acgen} \hfill [Command]\\%
 (Function: shu-cpp-acgen)

\begin{doc-string}
Generate a skeleton class code generation at point.
\end{doc-string}

\vspace{1em}
\noindent
\savebox{\funcname}{\noindent\texttt{ccdecl }}
\usebox{\funcname}\emph{class-name}
\index{ccdecl} \hfill [Command]\\%
 (Function: shu-cpp-ccdecl)

\begin{doc-string}
Generate a skeleton class declaration at point.
\end{doc-string}

\vspace{1em}
\noindent
\savebox{\funcname}{\noindent\texttt{ccgen }}
\usebox{\funcname}\emph{class-name}
\index{ccgen} \hfill [Command]\\%
 (Function: shu-cpp-ccgen)

\begin{doc-string}
Generate a skeleton class code generation at point.
\end{doc-string}

\vspace{1em}
\noindent
\savebox{\funcname}{\noindent\texttt{cdecl }}
\usebox{\funcname}\emph{class-name}
\index{cdecl} \hfill [Command]\\%
 (Function: shu-cpp-cdecl)

\begin{doc-string}
Generate a skeleton class declaration at point.
\end{doc-string}

\vspace{1em}
\noindent
\savebox{\funcname}{\noindent\texttt{cgen }}
\usebox{\funcname}\emph{class-name} \textbf{\&optional} \emph{use-allocator}
\index{cgen} \hfill [Command]\\%
 (Function: shu-cpp-cgen)

\begin{doc-string}
Generate a skeleton class code generation at point.
\end{doc-string}

\vspace{1em}
\noindent
\savebox{\funcname}{\noindent\texttt{dox-file }}
\usebox{\funcname}
\index{dox-file} \hfill [Command]\\%
 (Function: shu-dox-file)

\begin{doc-string}
Place a skeleton Doxygen file definition at point.
\end{doc-string}

\vspace{1em}
\noindent
\savebox{\funcname}{\noindent\texttt{fline }}
\usebox{\funcname}
\index{fline} \hfill [Command]\\%
 (Function: shu-fline)

\begin{doc-string}
Place a stream of \_\_FILE\_\_ and \_\_LINE\_\_ at point.
\end{doc-string}

\vspace{1em}
\noindent
\savebox{\funcname}{\noindent\texttt{gen-component }}
\usebox{\funcname}\emph{class-name}
\index{gen-component} \hfill [Command]\\%
 (Function: shu-gen-component)

\begin{doc-string}
Generate the three files for a new component: .cpp, .h, and .t.cpp
\end{doc-string}

\vspace{1em}
\noindent
\savebox{\funcname}{\noindent\texttt{hcgen }}
\usebox{\funcname}\emph{class-name}
\index{hcgen} \hfill [Command]\\%
 (Function: shu-cpp-hcgen)

\begin{doc-string}
Generate a skeleton class code generation at point.
\end{doc-string}

\subsection{List of functions and variables}

List of functions and variable definitions in this package.



\vspace{1em}
\noindent
\savebox{\funcname}{\noindent\texttt{shu-cpp-acgen }}
\usebox{\funcname}\emph{class-name}
\index{shu-cpp-acgen} \hfill [Command]\\%
 (Alias: acgen)

\begin{doc-string}
Generate a skeleton class code generation at point.
\end{doc-string}

\vspace{1em}
\noindent
\savebox{\funcname}{\noindent\texttt{shu-cpp-ccdecl }}
\usebox{\funcname}\emph{class-name}
\index{shu-cpp-ccdecl} \hfill [Command]\\%
 (Alias: ccdecl)

\begin{doc-string}
Generate a skeleton class declaration at point.
\end{doc-string}

\vspace{1em}
\noindent
\savebox{\funcname}{\noindent\texttt{shu-cpp-ccgen }}
\usebox{\funcname}\emph{class-name}
\index{shu-cpp-ccgen} \hfill [Command]\\%
 (Alias: ccgen)

\begin{doc-string}
Generate a skeleton class code generation at point.
\end{doc-string}

\vspace{1em}
\noindent
\savebox{\funcname}{\noindent\texttt{shu-cpp-cdecl }}
\usebox{\funcname}\emph{class-name}
\index{shu-cpp-cdecl} \hfill [Command]\\%
 (Alias: cdecl)

\begin{doc-string}
Generate a skeleton class declaration at point.
\end{doc-string}

\vspace{1em}
\noindent
\savebox{\funcname}{\noindent\texttt{shu-cpp-cgen }}
\usebox{\funcname}\emph{class-name} \textbf{\&optional} \emph{use-allocator}
\index{shu-cpp-cgen} \hfill [Command]\\%
 (Alias: cgen)

\begin{doc-string}
Generate a skeleton class code generation at point.
\end{doc-string}

\vspace{1em}
\noindent
\savebox{\funcname}{\noindent\texttt{shu-cpp-hcgen }}
\usebox{\funcname}\emph{class-name}
\index{shu-cpp-hcgen} \hfill [Command]\\%
 (Alias: hcgen)

\begin{doc-string}
Generate a skeleton class code generation at point.
\end{doc-string}

\vspace{1em}
\noindent
\savebox{\funcname}{\noindent\texttt{shu-cpp-inner-cdecl }}
\usebox{\funcname}\emph{class-name} \emph{copy-allowed}
\index{shu-cpp-inner-cdecl} \hfill [Function]
\hspace*{\wd\funcname}\emph{use-allocator}

\begin{doc-string}
Generate a skeleton class declaration at point.
\end{doc-string}

\vspace{1em}
\noindent
\savebox{\funcname}{\noindent\texttt{shu-cpp-misc-set-alias }}
\usebox{\funcname}
\index{shu-cpp-misc-set-alias} \hfill [Function]

\begin{doc-string}
Set the common alias names for the functions in shu-cpp-misc.
These are generally the same as the function names with the leading
shu- prefix removed.
\end{doc-string}

\vspace{1em}
\noindent
\savebox{\funcname}{\noindent\texttt{shu-dox-file }}
\usebox{\funcname}
\index{shu-dox-file} \hfill [Command]\\%
 (Alias: dox-file)

\begin{doc-string}
Place a skeleton Doxygen file definition at point.
\end{doc-string}

\vspace{1em}
\noindent
\savebox{\funcname}{\noindent\texttt{shu-fline }}
\usebox{\funcname}
\index{shu-fline} \hfill [Command]\\%
 (Alias: fline)

\begin{doc-string}
Place a stream of \_\_FILE\_\_ and \_\_LINE\_\_ at point.
\end{doc-string}

\vspace{1em}
\noindent
\savebox{\funcname}{\noindent\texttt{shu-gen-component }}
\usebox{\funcname}\emph{class-name}
\index{shu-gen-component} \hfill [Command]\\%
 (Alias: gen-component)

\begin{doc-string}
Generate the three files for a new component: .cpp, .h, and .t.cpp
\end{doc-string}

\vspace{1em}
\noindent
\savebox{\funcname}{\noindent\texttt{shu-generate-cfile }}
\usebox{\funcname}\emph{author} \emph{namespace} \emph{class-name}
\index{shu-generate-cfile} \hfill [Function]

\begin{doc-string}
Generate a skeleton cpp file
\end{doc-string}

\vspace{1em}
\noindent
\savebox{\funcname}{\noindent\texttt{shu-generate-hfile }}
\usebox{\funcname}\emph{author} \emph{namespace} \emph{class-name}
\index{shu-generate-hfile} \hfill [Function]

\begin{doc-string}
Generate a skeleton header file
\end{doc-string}

\vspace{1em}
\noindent
\savebox{\funcname}{\noindent\texttt{shu-generate-tfile }}
\usebox{\funcname}\emph{author} \emph{namespace} \emph{class-name}
\index{shu-generate-tfile} \hfill [Command]

\begin{doc-string}
Generate a skeleton t.cpp file
\end{doc-string}

\section{shu-cpp-project}


A collection of useful functions for dealing with project files and treating
a set of source files in multiple directories as a single project


\subsection{List of functions by alias name}

A list of aliases and associated function names.



\vspace{1em}
\noindent
\savebox{\funcname}{\noindent\texttt{clear-prefix }}
\usebox{\funcname}
\index{clear-prefix} \hfill [Function]\\%
 (Function: shu-clear-prefix)

\begin{doc-string}
Clear the default file name prefix for those times when we are trying to visit
a project file and point is not sitting on something that resembles a file name.
\end{doc-string}

\vspace{1em}
\noindent
\savebox{\funcname}{\noindent\texttt{count-c-project }}
\usebox{\funcname}
\index{count-c-project} \hfill [Command]\\%
 (Function: shu-count-c-project)

\begin{doc-string}
Count the number of lines of code in a project.  The final count is shown in
the minibuffer.  The counts of individual subdirectories are stored in the
temporary buffer \emph\{*shu-project-count*\}
\end{doc-string}

\vspace{1em}
\noindent
\savebox{\funcname}{\noindent\texttt{list-c-directories }}
\usebox{\funcname}
\index{list-c-directories} \hfill [Command]\\%
 (Function: shu-list-c-directories)

\begin{doc-string}
Insert into the current buffer the names of all of the directories in a project.
\end{doc-string}

\vspace{1em}
\noindent
\savebox{\funcname}{\noindent\texttt{list-c-project }}
\usebox{\funcname}
\index{list-c-project} \hfill [Command]\\%
 (Function: shu-list-c-project)

\begin{doc-string}
Insert into the current buffer the names of all of the code files in a project.
\end{doc-string}

\vspace{1em}
\noindent
\savebox{\funcname}{\noindent\texttt{make-c-project }}
\usebox{\funcname}\emph{proj-root}
\index{make-c-project} \hfill [Command]\\%
 (Function: shu-make-c-project)

\begin{doc-string}
Create a project file of all directories containing c or h files.
Starts at the specified root directory and searches all subdirectories for
any that contain c or h files.  It then inserts all of the directory names
into the current file at point.
\end{doc-string}

\vspace{1em}
\noindent
\savebox{\funcname}{\noindent\texttt{renew-c-project }}
\usebox{\funcname}
\index{renew-c-project} \hfill [Command]\\%
 (Function: shu-renew-c-project)

\begin{doc-string}
Renew a previously established project to pick up any new files.
\end{doc-string}

\vspace{1em}
\noindent
\savebox{\funcname}{\noindent\texttt{set-c-project }}
\usebox{\funcname}\emph{start} \emph{end}
\index{set-c-project} \hfill [Command]\\%
 (Function: shu-set-c-project)

\begin{doc-string}
Mark a region in a file that contains one subdirectory name per line.
Then invoke set-c-project and it will find and remember all of the c and h
files in those subdirectories.  You may then subsequently visit any of
those files by invoking M-x vh which will allow you to type in the file
name only (with auto completion) and will then visit the file in the
appropriate subdirectory.
\end{doc-string}

\vspace{1em}
\noindent
\savebox{\funcname}{\noindent\texttt{set-dir-prefix }}
\usebox{\funcname}\emph{prefix}
\index{set-dir-prefix} \hfill [Command]\\%
 (Function: shu-set-dir-prefix)

\begin{doc-string}
Set the default file name prefix to be the current directory name end for those
times when we are trying to visit a project file and point is not sitting on
something that resembles a file name.
\end{doc-string}

\vspace{1em}
\noindent
\savebox{\funcname}{\noindent\texttt{set-prefix }}
\usebox{\funcname}\emph{prefix}
\index{set-prefix} \hfill [Command]\\%
 (Function: shu-set-prefix)

\begin{doc-string}
Set the default file name prefix for those times when we are trying to visit
a project file and point is not sitting on something that resembles a file name.
\end{doc-string}

\vspace{1em}
\noindent
\savebox{\funcname}{\noindent\texttt{which-c-project }}
\usebox{\funcname}
\index{which-c-project} \hfill [Command]\\%
 (Function: shu-which-c-project)

\begin{doc-string}
Identify the current project by putting into a project buffer the name of the file
from which the project was derived as well as the name of all of the directories in the
project.  Then switch to that buffer.  The idea is to invoke this function, look at the
results in that buffer, and then quit out of the buffer.
\end{doc-string}

\subsection{List of functions and variables}

List of functions and variable definitions in this package.



\vspace{1em}
\noindent
\savebox{\funcname}{\noindent\texttt{shu-add-cpp-c-extensions }}
\usebox{\funcname}\emph{xtns}
\index{shu-add-cpp-c-extensions} \hfill [Function]

\begin{doc-string}
Add one or more file extensions to the list of C and C++ extensions recognized by the
C package functions.  Argument may be a single extension in a string or a list of strings.
This modifies both shu-cpp-c-extensions and shu-cpp-extensions.
\end{doc-string}

\vspace{1em}
\noindent
\savebox{\funcname}{\noindent\texttt{shu-add-cpp-h-extensions }}
\usebox{\funcname}\emph{xtns}
\index{shu-add-cpp-h-extensions} \hfill [Function]

\begin{doc-string}
Add one or more file extensions to the list of C and C++ extensions recognized by the
C package functions.  Argument may be a single extension in a string or a list of strings.
This modifies both shu-cpp-h-extensions and shu-cpp-extensions.
\end{doc-string}

\vspace{1em}
\noindent
\savebox{\funcname}{\noindent\texttt{shu-add-cpp-package-line }}
\usebox{\funcname}\emph{dir-name}
\index{shu-add-cpp-package-line} \hfill [Function]

\begin{doc-string}
Called with point at the beginning of the line.  Take the whole line as the
name of a directory, look into the directory, and create an alist of all of the
files in the directory as described in shu-cpp-subdir-for-package.
\end{doc-string}

\vspace{1em}
\noindent
\savebox{\funcname}{\noindent\texttt{shu-clear-prefix }}
\usebox{\funcname}
\index{shu-clear-prefix} \hfill [Function]\\%
 (Alias: clear-prefix)

\begin{doc-string}
Clear the default file name prefix for those times when we are trying to visit
a project file and point is not sitting on something that resembles a file name.
\end{doc-string}

\vspace{1em}
\noindent
\savebox{\funcname}{\noindent\texttt{shu-completion-is-directory }}
\usebox{\funcname}
\index{shu-completion-is-directory} \hfill [Variable]

\begin{doc-string}
True if we are to use the current directory name as the file name prefix.
\end{doc-string}

\vspace{1em}
\noindent
\savebox{\funcname}{\noindent\texttt{shu-count-c-project }}
\usebox{\funcname}
\index{shu-count-c-project} \hfill [Command]\\%
 (Alias: count-c-project)

\begin{doc-string}
Count the number of lines of code in a project.  The final count is shown in
the minibuffer.  The counts of individual subdirectories are stored in the
temporary buffer \emph\{*shu-project-count*\}
\end{doc-string}

\vspace{1em}
\noindent
\savebox{\funcname}{\noindent\texttt{shu-count-in-cpp-directory }}
\usebox{\funcname}\emph{directory-name} \emph{pbuf}
\index{shu-count-in-cpp-directory} \hfill [Function]
\hspace*{\wd\funcname}\emph{t-h-files} \emph{t-c-files}
\hspace*{\wd\funcname}\emph{t-c-count}

\begin{doc-string}
Count the lines of code in each of the code files in the given directory, updating
the message in the minibuffer and passing the totals back to the caller.
\end{doc-string}

\vspace{1em}
\noindent
\savebox{\funcname}{\noindent\texttt{shu-cpp-c-extensions }}
\usebox{\funcname}
\index{shu-cpp-c-extensions} \hfill [Constant]

\begin{doc-string}
A list of file extensions for all of the C file types we want to find.  This is defined
as defconst in shu-cpp-base.el but may be modified by shu-add-cpp-c-extensions.
\end{doc-string}

\vspace{1em}
\noindent
\savebox{\funcname}{\noindent\texttt{shu-cpp-c-file-count }}
\usebox{\funcname}
\index{shu-cpp-c-file-count} \hfill [Variable]

\begin{doc-string}
This is the count of the number of C files found in the project.
\end{doc-string}

\vspace{1em}
\noindent
\savebox{\funcname}{\noindent\texttt{shu-cpp-choose-file }}
\usebox{\funcname}\emph{assoc-result}
\index{shu-cpp-choose-file} \hfill [Function]

\begin{doc-string}
Choose the file to visit for a given unqualified name.  If there is
only one file associated with the name then visit it.  If there are
multiple files put all of the fully qualified file names in the completion
buffer and give the user the opportunity to select the desired file.  Then
visit that file.
\end{doc-string}

\vspace{1em}
\noindent
\savebox{\funcname}{\noindent\texttt{shu-cpp-class-list }}
\usebox{\funcname}
\index{shu-cpp-class-list} \hfill [Variable]

\begin{doc-string}
This is an alist whose keys are unqualified file names and whose
values contain a list of the fully qualified files with the same
name.
\end{doc-string}

\vspace{1em}
\noindent
\savebox{\funcname}{\noindent\texttt{shu-cpp-common-completion }}
\usebox{\funcname}
\index{shu-cpp-common-completion} \hfill [Function]

\begin{doc-string}
Called when the user hits enter or clicks mouse button 2 on completion window.
At this point the users selected choice is in the current buffer.  We get the
answer from the current buffer and call the function that is currently
pointed to by shu-cpp-completion-target.
\end{doc-string}

\vspace{1em}
\noindent
\savebox{\funcname}{\noindent\texttt{shu-cpp-completion-current-buffer }}
\usebox{\funcname}
\index{shu-cpp-completion-current-buffer} \hfill [Variable]

\begin{doc-string}
Active buffer just before we have to do a completion.
\end{doc-string}

\vspace{1em}
\noindent
\savebox{\funcname}{\noindent\texttt{shu-cpp-completion-prefix }}
\usebox{\funcname}
\index{shu-cpp-completion-prefix} \hfill [Variable]

\begin{doc-string}
The default file name prefix when we are looking for a file and point is not
sitting on something that appears to be a file name.
\end{doc-string}

\vspace{1em}
\noindent
\savebox{\funcname}{\noindent\texttt{shu-cpp-completion-scratch }}
\usebox{\funcname}
\index{shu-cpp-completion-scratch} \hfill [Variable]

\begin{doc-string}
Scratch buffer used by C file name completions.
\end{doc-string}

\vspace{1em}
\noindent
\savebox{\funcname}{\noindent\texttt{shu-cpp-completion-target }}
\usebox{\funcname}
\index{shu-cpp-completion-target} \hfill [Variable]

\begin{doc-string}
Global variable used to hold the function to be invoked at the end of the
current completion.
\end{doc-string}

\vspace{1em}
\noindent
\savebox{\funcname}{\noindent\texttt{shu-cpp-directory-prefix }}
\usebox{\funcname}
\index{shu-cpp-directory-prefix} \hfill [Function]

\begin{doc-string}
Get a directory based prefix, which is the last name in the current path.  If the current
directory is ``foo/blah/humbug'', the value returned from this function is ``humbug''
\end{doc-string}

\vspace{1em}
\noindent
\savebox{\funcname}{\noindent\texttt{shu-cpp-extensions }}
\usebox{\funcname}
\index{shu-cpp-extensions} \hfill [Constant]

\begin{doc-string}
A list of file extensions for all of the file types we want to find.  This is defined
as defconst in shu-cpp-base.el but may be modified by shu-add-cpp-c-extensions or
shu-add-cpp-h-extensions.
\end{doc-string}

\vspace{1em}
\noindent
\savebox{\funcname}{\noindent\texttt{shu-cpp-final-list }}
\usebox{\funcname}
\index{shu-cpp-final-list} \hfill [Variable]

\begin{doc-string}
The name of the shared variable that contains the list of directories assembled
by shu-make-c-project
\end{doc-string}

\vspace{1em}
\noindent
\savebox{\funcname}{\noindent\texttt{shu-cpp-finish-project }}
\usebox{\funcname}\textbf{\&optional} \emph{key-list}
\index{shu-cpp-finish-project} \hfill [Function]

\begin{doc-string}
Finish constructing a C project from a user file list.
\end{doc-string}

\vspace{1em}
\noindent
\savebox{\funcname}{\noindent\texttt{shu-cpp-found-extensions }}
\usebox{\funcname}
\index{shu-cpp-found-extensions} \hfill [Variable]

\begin{doc-string}
This is a list of all of the file extensions found in the current project.  While
shu-cpp-extensions contains all of the extensions that we look for.  This variable
contains those that we actually found in building the current project.
\end{doc-string}

\vspace{1em}
\noindent
\savebox{\funcname}{\noindent\texttt{shu-cpp-h-extensions }}
\usebox{\funcname}
\index{shu-cpp-h-extensions} \hfill [Constant]

\begin{doc-string}
A list of file extensions for all of the H file types we want to find.  This is defined
as defconst in shu-cpp-base.el but may be modified by shu-add-cpp-h-extensions
\end{doc-string}

\vspace{1em}
\noindent
\savebox{\funcname}{\noindent\texttt{shu-cpp-h-file-count }}
\usebox{\funcname}
\index{shu-cpp-h-file-count} \hfill [Variable]

\begin{doc-string}
This is the count of the number of H files found in the project.
\end{doc-string}

\vspace{1em}
\noindent
\savebox{\funcname}{\noindent\texttt{shu-cpp-project-file }}
\usebox{\funcname}
\index{shu-cpp-project-file} \hfill [Variable]

\begin{doc-string}
The name of the file from which the current project was read.
\end{doc-string}

\vspace{1em}
\noindent
\savebox{\funcname}{\noindent\texttt{shu-cpp-project-list }}
\usebox{\funcname}
\index{shu-cpp-project-list} \hfill [Variable]

\begin{doc-string}
List that holds all of the subdirectories in the current project.
\end{doc-string}

\vspace{1em}
\noindent
\savebox{\funcname}{\noindent\texttt{shu-cpp-project-set-alias }}
\usebox{\funcname}
\index{shu-cpp-project-set-alias} \hfill [Function]

\begin{doc-string}
Set the common alias names for the functions in shu-cpp-project.
These are generally the same as the function names with the leading
shu- prefix removed.
\end{doc-string}

\vspace{1em}
\noindent
\savebox{\funcname}{\noindent\texttt{shu-cpp-project-subdirs }}
\usebox{\funcname}\emph{dir-name} \emph{level}
\index{shu-cpp-project-subdirs} \hfill [Function]

\begin{doc-string}
Starting with the directory name \emph{dir-name}. create a list of subdirectories
whose head is in \emph{shu-cpp-final-list}, that contains the name of every directory and
subdirectory that contains C, C++, or H files.  This is used by shu-make-c-project
and other functions that wish to discover all directories that might contain
source code.
\end{doc-string}

\vspace{1em}
\noindent
\savebox{\funcname}{\noindent\texttt{shu-cpp-project-time }}
\usebox{\funcname}
\index{shu-cpp-project-time} \hfill [Variable]

\begin{doc-string}
This is the time at which the current project was created.
\end{doc-string}

\vspace{1em}
\noindent
\savebox{\funcname}{\noindent\texttt{shu-cpp-resolve-choice }}
\usebox{\funcname}\emph{full-name-list} \emph{target}
\index{shu-cpp-resolve-choice} \hfill [Function]

\begin{doc-string}
Choose from a number of possible file names.
We have found an unqualified file name of interest but it resolves to multiple
fully qualified file names.  Display all of the possibilities in a completion
buffer and ask the user to choose the desired one.  The string containing the
chosen fully qualified file name will then be passed to the function pointed
to by target.
\end{doc-string}

\vspace{1em}
\noindent
\savebox{\funcname}{\noindent\texttt{shu-cpp-subdir-for-package }}
\usebox{\funcname}\emph{directory-name}
\index{shu-cpp-subdir-for-package} \hfill [Function]

\begin{doc-string}
Given a subdirectory name return an alist that contains as keys the names
of all of the c and h files in the subdirectory, and as values the the
fully qualified name and path of the c or h file.  So if the directory
``/u/foo/bar'' contains thing.c and what.h the returned alist would be

\begin{verbatim}
      ( (``thing.c'' ``/u/foo/bar/thing.c'')
        (``what.h''  ``/u/foo/bar/what.h'' ) )
\end{verbatim}

This allows us to associate the key ``thing.c'' with the fully qualified
name ``/u/foo/bar/thing.c''.
\end{doc-string}

\vspace{1em}
\noindent
\savebox{\funcname}{\noindent\texttt{shu-cpp-target-file-column }}
\usebox{\funcname}
\index{shu-cpp-target-file-column} \hfill [Variable]

\begin{doc-string}
If non-nil, this represents the column number that is to be located after a
file is visited by vh() and has gone through buffer completion selection.
\end{doc-string}

\vspace{1em}
\noindent
\savebox{\funcname}{\noindent\texttt{shu-cpp-target-file-line }}
\usebox{\funcname}
\index{shu-cpp-target-file-line} \hfill [Variable]

\begin{doc-string}
If non-nil, this represents the line number that is to be located after a
file is visited by vh() and has gone through buffer completion selection.
\end{doc-string}

\vspace{1em}
\noindent
\savebox{\funcname}{\noindent\texttt{shu-cpp-visit-target }}
\usebox{\funcname}\emph{file-name}
\index{shu-cpp-visit-target} \hfill [Function]

\begin{doc-string}
This is the function that visits the file name chosen by vh() and perhaps
by a completing read from a completion buffer.
\end{doc-string}

\vspace{1em}
\noindent
\savebox{\funcname}{\noindent\texttt{shu-default-file-to-seek }}
\usebox{\funcname}
\index{shu-default-file-to-seek} \hfill [Variable]

\begin{doc-string}
The default file to seek that is proposed as a possible file when vh() finds a
file name under the cursor, possibly with a line number.  If the user chooses a
file other than this one, we need to forget the associated line number.
\end{doc-string}

\vspace{1em}
\noindent
\savebox{\funcname}{\noindent\texttt{shu-find-default-cpp-name }}
\usebox{\funcname}
\index{shu-find-default-cpp-name} \hfill [Function]

\begin{doc-string}
Find a default file name to visit.  Calls shu-find-line-and-file to find a possible file
name and possible line number within the file.  Return the file name if one is found and
sets shu-cpp-target-file-line to the line number if one is found
\end{doc-string}

\vspace{1em}
\noindent
\savebox{\funcname}{\noindent\texttt{shu-find-line-and-file }}
\usebox{\funcname}
\index{shu-find-line-and-file} \hfill [Function]

\begin{doc-string}
If point is sitting on the word ``line'', then look for a string of the form
``line 678 of frobnitz.cpp'' and return a list whose first item is the file name
and whose second item is the line number.  If point is not sitting on the word ``line'',
then check to see if point is sitting on a string that has the syntax of a valid
file name.  If that is the case, remember the file name.  If the file name is
followed by a colon, look for a line number following the colon.  If found, look
for another colon followed by a possible column number.  This function will return
nil if none of the above are found.  If only a file name is found, return a list
with one entry.  If file name and line number, a list with two entries.  If file
name, line number, and column number, a list with three entries.
\end{doc-string}

\vspace{1em}
\noindent
\savebox{\funcname}{\noindent\texttt{shu-get-line-column-of-file }}
\usebox{\funcname}
\index{shu-get-line-column-of-file} \hfill [Function]

\begin{doc-string}
Fetch the potential line number and column number within a file.  On entry,
point is positioned at the character following a file name.  This file name
may be followed by a line number and the line number may be followed by a
column number.  This function recognizes two forms of line and column
specifications.

  thing.cpp:1234:42

indicates the file thing.cpp line number 1234, column 42

  [file=thing.cpp] [line=1234]

indicates the file thing.cpp line number 1234.

The purpose of this function is only to gather the line and column
specification following the file name.  The return value is a list, which is
empty if no line or column number was found.  It has only one element, which
is the line number if only a line number was found.  It has two elements,
which are the line number and column number if both line number and column
number were found.

This should probably be turned into a hook at some point so that other line
and column number indications may be used.
\end{doc-string}

\vspace{1em}
\noindent
\savebox{\funcname}{\noindent\texttt{shu-global-operation }}
\usebox{\funcname}\emph{documentation}
\index{shu-global-operation} \hfill [Function]
\hspace*{\wd\funcname}\textbf{\&optional} \emph{search-target}
\hspace*{\wd\funcname}

\begin{doc-string}
Invoke a function on every file in the project.
documentation is the string to put in the buffer to describe the operation.
\end{doc-string}

\vspace{1em}
\noindent
\savebox{\funcname}{\noindent\texttt{shu-internal-visit-project-file }}
\usebox{\funcname}\emph{look-for-target}
\index{shu-internal-visit-project-file} \hfill [Function]

\begin{doc-string}
Visit a c or h file in a project.
\end{doc-string}

\vspace{1em}
\noindent
\savebox{\funcname}{\noindent\texttt{shu-internal-which-c-project }}
\usebox{\funcname}\emph{pbuf}
\index{shu-internal-which-c-project} \hfill [Function]

\begin{doc-string}
Undocumented
\end{doc-string}

\vspace{1em}
\noindent
\savebox{\funcname}{\noindent\texttt{shu-list-c-directories }}
\usebox{\funcname}
\index{shu-list-c-directories} \hfill [Command]\\%
 (Alias: list-c-directories)

\begin{doc-string}
Insert into the current buffer the names of all of the directories in a project.
\end{doc-string}

\vspace{1em}
\noindent
\savebox{\funcname}{\noindent\texttt{shu-list-c-project }}
\usebox{\funcname}
\index{shu-list-c-project} \hfill [Command]\\%
 (Alias: list-c-project)

\begin{doc-string}
Insert into the current buffer the names of all of the code files in a project.
\end{doc-string}

\vspace{1em}
\noindent
\savebox{\funcname}{\noindent\texttt{shu-list-in-cpp-directory }}
\usebox{\funcname}\emph{directory-name}
\index{shu-list-in-cpp-directory} \hfill [Function]

\begin{doc-string}
Insert into the current buffer the names of all of the code files in a directory.
\end{doc-string}

\vspace{1em}
\noindent
\savebox{\funcname}{\noindent\texttt{shu-make-c-project }}
\usebox{\funcname}\emph{proj-root}
\index{shu-make-c-project} \hfill [Command]\\%
 (Alias: make-c-project)

\begin{doc-string}
Create a project file of all directories containing c or h files.
Starts at the specified root directory and searches all subdirectories for
any that contain c or h files.  It then inserts all of the directory names
into the current file at point.
\end{doc-string}

\vspace{1em}
\noindent
\savebox{\funcname}{\noindent\texttt{shu-on-the-word-line }}
\usebox{\funcname}
\index{shu-on-the-word-line} \hfill [Function]

\begin{doc-string}
Return the character position of the start of the current word if point is sitting
anywhere on the word ``line''.  This is used pick up file positions of the form:
``line 628 of frobnitz.cpp''
\end{doc-string}

\vspace{1em}
\noindent
\savebox{\funcname}{\noindent\texttt{shu-possible-cpp-file-name }}
\usebox{\funcname}
\index{shu-possible-cpp-file-name} \hfill [Function]

\begin{doc-string}
Return a list containing a possible file name with a possible line number
and a possible column number.  If the thing on point does not resemble a file
name, return nil.  If it looks like a file name, save it and call
shu-get-line-column-of-file to perhaps harvest a line number and column number
within the file.  The return result is a list of length one if there is only
a file name, a list of length two if there is a file name and line number, a
list of length three if there is a file name, line number, and column number.
\end{doc-string}

\vspace{1em}
\noindent
\savebox{\funcname}{\noindent\texttt{shu-project-class-count }}
\usebox{\funcname}
\index{shu-project-class-count} \hfill [Variable]

\begin{doc-string}
Undocumented
\end{doc-string}

\vspace{1em}
\noindent
\savebox{\funcname}{\noindent\texttt{shu-project-cpp-buffer-name }}
\usebox{\funcname}
\index{shu-project-cpp-buffer-name} \hfill [Constant]

\begin{doc-string}
The name of the buffer into which messages are placed as c and h files
are being scanned.
\end{doc-string}

\vspace{1em}
\noindent
\savebox{\funcname}{\noindent\texttt{shu-project-errors }}
\usebox{\funcname}
\index{shu-project-errors} \hfill [Variable]

\begin{doc-string}
Undocumented
\end{doc-string}

\vspace{1em}
\noindent
\savebox{\funcname}{\noindent\texttt{shu-project-file-list }}
\usebox{\funcname}
\index{shu-project-file-list} \hfill [Variable]

\begin{doc-string}
Undocumented
\end{doc-string}

\vspace{1em}
\noindent
\savebox{\funcname}{\noindent\texttt{shu-project-user-class-count }}
\usebox{\funcname}
\index{shu-project-user-class-count} \hfill [Variable]

\begin{doc-string}
Undocumented
\end{doc-string}

\vspace{1em}
\noindent
\savebox{\funcname}{\noindent\texttt{shu-renew-c-project }}
\usebox{\funcname}
\index{shu-renew-c-project} \hfill [Command]\\%
 (Alias: renew-c-project)

\begin{doc-string}
Renew a previously established project to pick up any new files.
\end{doc-string}

\vspace{1em}
\noindent
\savebox{\funcname}{\noindent\texttt{shu-set-c-project }}
\usebox{\funcname}\emph{start} \emph{end}
\index{shu-set-c-project} \hfill [Command]\\%
 (Alias: set-c-project)

\begin{doc-string}
Mark a region in a file that contains one subdirectory name per line.
Then invoke set-c-project and it will find and remember all of the c and h
files in those subdirectories.  You may then subsequently visit any of
those files by invoking M-x vh which will allow you to type in the file
name only (with auto completion) and will then visit the file in the
appropriate subdirectory.
\end{doc-string}

\vspace{1em}
\noindent
\savebox{\funcname}{\noindent\texttt{shu-set-dir-prefix }}
\usebox{\funcname}\emph{prefix}
\index{shu-set-dir-prefix} \hfill [Command]\\%
 (Alias: set-dir-prefix)

\begin{doc-string}
Set the default file name prefix to be the current directory name end for those
times when we are trying to visit a project file and point is not sitting on
something that resembles a file name.
\end{doc-string}

\vspace{1em}
\noindent
\savebox{\funcname}{\noindent\texttt{shu-set-prefix }}
\usebox{\funcname}\emph{prefix}
\index{shu-set-prefix} \hfill [Command]\\%
 (Alias: set-prefix)

\begin{doc-string}
Set the default file name prefix for those times when we are trying to visit
a project file and point is not sitting on something that resembles a file name.
\end{doc-string}

\vspace{1em}
\noindent
\savebox{\funcname}{\noindent\texttt{shu-setup-project-and-tags }}
\usebox{\funcname}\emph{proj-dir}
\index{shu-setup-project-and-tags} \hfill [Function]

\begin{doc-string}
Visit a project file, make a C project from the contents of the whole file,
create a file called ``files.txt'' with the name of every file found, invoke
ctags on that file to build a new tags file, and then visit the tags file.
\emph{proj-dir} is the name of the directory in which the project file exists and in
which the tags file is to be built.
\end{doc-string}

\vspace{1em}
\noindent
\savebox{\funcname}{\noindent\texttt{shu-vh }}
\usebox{\funcname}
\index{shu-vh} \hfill [Command]

\begin{doc-string}
Visit a c or h file in a project.  If point is on something that resembles a file
name, then visit that file.  If the file name is followed by a colon and a number
then go to that line in the file.  If the line number is followed by a colon and
a number then use the second number as the column number within the line.
\end{doc-string}

\vspace{1em}
\noindent
\savebox{\funcname}{\noindent\texttt{shu-vj }}
\usebox{\funcname}
\index{shu-vj} \hfill [Command]

\begin{doc-string}
Visit a c or h file in a project.  Ignore any text that point is on and visit the
file typed in the completion buffer.
\end{doc-string}

\vspace{1em}
\noindent
\savebox{\funcname}{\noindent\texttt{shu-which-c-project }}
\usebox{\funcname}
\index{shu-which-c-project} \hfill [Command]\\%
 (Alias: which-c-project)

\begin{doc-string}
Identify the current project by putting into a project buffer the name of the file
from which the project was derived as well as the name of all of the directories in the
project.  Then switch to that buffer.  The idea is to invoke this function, look at the
results in that buffer, and then quit out of the buffer.
\end{doc-string}

\section{shu-cpp-token}


Functions to parse a region of C++ code and return a list of tokens
found therein.  The returned list is a list of token-info, whose structure
is shown below.

The two top level functions in this file are shu-cpp-tokenize-region and
shu-cpp-reverse-tokenize-region.  The former returns a list of tokens with the
first token in the list being the first token found.  The latter function
returns the reverse of the former.


\subsection{List of functions by alias name}

A list of aliases and associated function names.



\vspace{1em}
\noindent
\savebox{\funcname}{\noindent\texttt{parse-region }}
\usebox{\funcname}\emph{start} \emph{end}
\index{parse-region} \hfill [Command]\\%
 (Function: shu-cpp-parse-region)

\begin{doc-string}
Parse the region between \emph{start} and \emph{end} into a list of all of the C++ tokens
contained therein, displaying the result in the Shu unit test buffer.
\end{doc-string}

\vspace{1em}
\noindent
\savebox{\funcname}{\noindent\texttt{reverse-parse-region }}
\usebox{\funcname}\emph{start} \emph{end}
\index{reverse-parse-region} \hfill [Command]\\%
 (Function: shu-cpp-reverse-parse-region)

\begin{doc-string}
Reverse parse the region between \emph{start} and \emph{end} into a list of all of the C++
tokens contained therein, displaying the result in the Shu unit test buffer.
\end{doc-string}

\subsection{List of functions and variables}

List of functions and variable definitions in this package.



\vspace{1em}
\noindent
\savebox{\funcname}{\noindent\texttt{shu-cpp-adjust-template-parameters }}
\usebox{\funcname}\emph{token-list}
\index{shu-cpp-adjust-template-parameters} \hfill [Function]

\begin{doc-string}
Turn each set of template parameters in a reverse parsed list (anything between
``$>$`` and ``$<$`` into a separate token of type \emph{shu-cpp-token-type-tp}.  e.g., the
five separate tokens ``$>$``, ``double'', ``,'', ``int'', ``$<$`` will be turned into
one new token of type \emph{shu-cpp-token-type-tp} whose token value is ``int, double''.
\end{doc-string}

\vspace{1em}
\noindent
\savebox{\funcname}{\noindent\texttt{shu-cpp-compare-tlist-sans-comment }}
\usebox{\funcname}\emph{token-list1}
\index{shu-cpp-compare-tlist-sans-comment} \hfill [Function]
\hspace*{\wd\funcname}

\begin{doc-string}
Compare the two lists of TOKEN-INFO skipping comments and stopping at the end
of the shortest one.  The purpose of this function is to determine if two bits
of reverse parsed code have the same suffix.
\end{doc-string}

\vspace{1em}
\noindent
\savebox{\funcname}{\noindent\texttt{shu-cpp-compare-token-info }}
\usebox{\funcname}\emph{token-info1} \emph{token-info2}
\index{shu-cpp-compare-token-info} \hfill [Function]

\begin{doc-string}
Compare the two instances of TOKEN-INFO, returning true if their contents
are the same.
\end{doc-string}

\vspace{1em}
\noindent
\savebox{\funcname}{\noindent\texttt{shu-cpp-compare-token-info-sans-pos }}
\usebox{\funcname}\emph{token-info1}
\index{shu-cpp-compare-token-info-sans-pos} \hfill [Function]
\hspace*{\wd\funcname}

\begin{doc-string}
Compare the two instances of TOKEN-INFO, returning true if their contents
are the same.  Do not include the start or end points in the comparison.
\end{doc-string}

\vspace{1em}
\noindent
\savebox{\funcname}{\noindent\texttt{shu-cpp-copy-token-info }}
\usebox{\funcname}\emph{token-info}
\index{shu-cpp-copy-token-info} \hfill [Function]

\begin{doc-string}
Return a deep copy of the given \emph{token-info}.
\end{doc-string}

\vspace{1em}
\noindent
\savebox{\funcname}{\noindent\texttt{shu-cpp-get-comment }}
\usebox{\funcname}\emph{start} \emph{end}
\index{shu-cpp-get-comment} \hfill [Function]

\begin{doc-string}
Get the comment that starts at point.  If it starts with //, get to end of
line.  If it starts with /\emph\{*,\} skip to terminating \emph\{*/.\}  If there is no terminating
\emph\{*/\} in the region, create a TOKEN-INFO with the appropriate error message in it.
\end{doc-string}

\vspace{1em}
\noindent
\savebox{\funcname}{\noindent\texttt{shu-cpp-get-operator-token }}
\usebox{\funcname}\emph{length}
\index{shu-cpp-get-operator-token} \hfill [Function]

\begin{doc-string}
Fetch the C++ operator that starts at point.  \emph{length} is the number of characters
in the operator, which is either 1, 2, or 3.
\end{doc-string}

\vspace{1em}
\noindent
\savebox{\funcname}{\noindent\texttt{shu-cpp-get-quoted-token }}
\usebox{\funcname}\emph{start} \emph{end}
\index{shu-cpp-get-quoted-token} \hfill [Function]

\begin{doc-string}
Find the token in the buffer between \emph{start} and \emph{end} that is terminated by an
unescaped quote.  On entry, point must be positioned on the quote that starts
the string.  The appropriate error message is returned if there is no unescaped
quote before the end of the current line.  If the character under point is not a
quote start character, nil is returned.
\end{doc-string}

\vspace{1em}
\noindent
\savebox{\funcname}{\noindent\texttt{shu-cpp-get-unquoted-token }}
\usebox{\funcname}\emph{start} \emph{end}
\index{shu-cpp-get-unquoted-token} \hfill [Function]

\begin{doc-string}
Find the unquoted token in the buffer that starts at point.  The token is
terminated either by the position of \emph{end} or by the regular expression that
defines the end of an unquoted token.
\end{doc-string}

\vspace{1em}
\noindent
\savebox{\funcname}{\noindent\texttt{shu-cpp-is-reverse-token-list-balanced }}
\usebox{\funcname}\emph{token-list}
\index{shu-cpp-is-reverse-token-list-balanced} \hfill [Function]
\hspace*{\wd\funcname}\emph{close-char}

\begin{doc-string}
Return t if a token-list contains matched pairs of \emph{open-char} and \emph{close-char}.
If imbalance is present, print error message and return nil.  Typically \emph{open-char}
might be a left parenthesis and \emph{close-char} might be a right parenthesis.  Or they
might be ``$<$`` and ``$>$``, or any other pair types.  Note that this function
returns t if there are no occurrences of \emph{open-char} and \emph{close-char}
\end{doc-string}

\vspace{1em}
\noindent
\savebox{\funcname}{\noindent\texttt{shu-cpp-make-token-info }}
\usebox{\funcname}\emph{token} \emph{token-type} \emph{spoint}
\index{shu-cpp-make-token-info} \hfill [Function]
\hspace*{\wd\funcname}\textbf{\&optional} \emph{error-message}

\begin{doc-string}
Pack the supplied arguments into a TOKEN-INFO and return the TOKEN-INFO.
\end{doc-string}

\vspace{1em}
\noindent
\savebox{\funcname}{\noindent\texttt{shu-cpp-operator-start }}
\usebox{\funcname}
\index{shu-cpp-operator-start} \hfill [Constant]

\begin{doc-string}
Define the set of characters that start C++ operators
\end{doc-string}

\vspace{1em}
\noindent
\savebox{\funcname}{\noindent\texttt{shu-cpp-operator-start-chars }}
\usebox{\funcname}
\index{shu-cpp-operator-start-chars} \hfill [Constant]

\begin{doc-string}
Define the set of characters that start C++ operators
\end{doc-string}

\vspace{1em}
\noindent
\savebox{\funcname}{\noindent\texttt{shu-cpp-operators-one }}
\usebox{\funcname}
\index{shu-cpp-operators-one} \hfill [Constant]

\begin{doc-string}
Define the set of one character operators.  Note that we include ; as
an operator, even though, strictly speaking, it is not an operator.
\end{doc-string}

\vspace{1em}
\noindent
\savebox{\funcname}{\noindent\texttt{shu-cpp-operators-three }}
\usebox{\funcname}
\index{shu-cpp-operators-three} \hfill [Constant]

\begin{doc-string}
Define the set of three character C++ operators
\end{doc-string}

\vspace{1em}
\noindent
\savebox{\funcname}{\noindent\texttt{shu-cpp-operators-two }}
\usebox{\funcname}
\index{shu-cpp-operators-two} \hfill [Constant]

\begin{doc-string}
Define the set of two character C++ operators
\end{doc-string}

\vspace{1em}
\noindent
\savebox{\funcname}{\noindent\texttt{shu-cpp-parse-region }}
\usebox{\funcname}\emph{start} \emph{end}
\index{shu-cpp-parse-region} \hfill [Command]\\%
 (Alias: parse-region)

\begin{doc-string}
Parse the region between \emph{start} and \emph{end} into a list of all of the C++ tokens
contained therein, displaying the result in the Shu unit test buffer.
\end{doc-string}

\vspace{1em}
\noindent
\savebox{\funcname}{\noindent\texttt{shu-cpp-remove-template-parameters }}
\usebox{\funcname}\emph{token-list}
\index{shu-cpp-remove-template-parameters} \hfill [Function]
\hspace*{\wd\funcname}\emph{preserve-template}

\begin{doc-string}
Remove from the token-list any template parameters (anything between ``$>$``
and its matching ``$>$``).  In addition, adjust the end point of the token
immediately prior to the template parameter to be that of the endpoint of the
template parameter.
Thus something like the following:
\begin{verbatim}
    Mumble<int, double>
\end{verbatim}
becomes the token ``Mumble'' with a length of 19.  If \emph{preserve-template} is true,
then we change the token that contains the type name by copying the template
parameters into it.  If the type name token was ``Mumble'', then the token
itself is changed to ``Mumble$<$int, double$>$``.  The tokens that represent the
template parameters are removed from the token list in either case.
This eliminates any comma that does not immediately follow a parameter name.
As we scan the reverse ordered token list, any comma that we find immediately
precedes a variable name in the parameter list.  There may be intervening
operators and comments.  But once we find a comma, the next unquoted token is
the variable name.
\end{doc-string}

\vspace{1em}
\noindent
\savebox{\funcname}{\noindent\texttt{shu-cpp-replace-token-info }}
\usebox{\funcname}\emph{token-info} \emph{token}
\index{shu-cpp-replace-token-info} \hfill [Function]
\hspace*{\wd\funcname}\emph{spoint} \emph{epoint} \textbf{\&optional}
\hspace*{\wd\funcname}

\begin{doc-string}
Replace the supplied arguments in the given \emph{token-info} and return the \emph{token-info}.
\end{doc-string}

\vspace{1em}
\noindent
\savebox{\funcname}{\noindent\texttt{shu-cpp-reverse-parse-region }}
\usebox{\funcname}\emph{start} \emph{end}
\index{shu-cpp-reverse-parse-region} \hfill [Command]\\%
 (Alias: reverse-parse-region)

\begin{doc-string}
Reverse parse the region between \emph{start} and \emph{end} into a list of all of the C++
tokens contained therein, displaying the result in the Shu unit test buffer.
\end{doc-string}

\vspace{1em}
\noindent
\savebox{\funcname}{\noindent\texttt{shu-cpp-reverse-tokenize-region }}
\usebox{\funcname}\emph{start} \emph{end}
\index{shu-cpp-reverse-tokenize-region} \hfill [Function]
\hspace*{\wd\funcname}\emph{limit}

\begin{doc-string}
Scan the region between \emph{start} and AND to build a list of tokens that represent the C++
code in the region.  Return a cons cell with two items in it.  The car of the cons cell
is a token-info that represents a parse error.  The cdr of the cons cell is the list of
tokens.  This list is incomplete if the car of the cons sell is not nil.  The optional
\emph{limit} argument is used to bound the scan as follows.  When we have added to the list the
first token that is beyond the point specified by \emph{limit}, we stop the scan.
\end{doc-string}

\vspace{1em}
\noindent
\savebox{\funcname}{\noindent\texttt{shu-cpp-reverse-tokenize-region-for-command }}
\usebox{\funcname}\emph{start}
\index{shu-cpp-reverse-tokenize-region-for-command} \hfill [Function]
\hspace*{\wd\funcname}\textbf{\&optional}
\hspace*{\wd\funcname}

\begin{doc-string}
Reverse tokenize the region between \emph{start} and \emph{end} into a list of all of the C++
tokens contained therein, displaying any error message, if there is one.  If no
error, return the token list, else return nil
\end{doc-string}

\vspace{1em}
\noindent
\savebox{\funcname}{\noindent\texttt{shu-cpp-token-delimiter-chars }}
\usebox{\funcname}
\index{shu-cpp-token-delimiter-chars} \hfill [Constant]

\begin{doc-string}
List of all of the characters that terminate an unquoted C++ token
\end{doc-string}

\vspace{1em}
\noindent
\savebox{\funcname}{\noindent\texttt{shu-cpp-token-delimiter-end }}
\usebox{\funcname}
\index{shu-cpp-token-delimiter-end} \hfill [Constant]

\begin{doc-string}
Regular expression to define that which terminates an unquoted token in C++
\end{doc-string}

\vspace{1em}
\noindent
\savebox{\funcname}{\noindent\texttt{shu-cpp-token-extract-info }}
\usebox{\funcname}\emph{token-info} \emph{token}
\index{shu-cpp-token-extract-info} \hfill [Macro]
\hspace*{\wd\funcname}\emph{spoint} \emph{epoint}
\hspace*{\wd\funcname}

\begin{doc-string}
Extract the information out of a token-info
\end{doc-string}

\vspace{1em}
\noindent
\savebox{\funcname}{\noindent\texttt{shu-cpp-token-find-spanning-info-token }}
\usebox{\funcname}\emph{token-list}
\index{shu-cpp-token-find-spanning-info-token} \hfill [Function]
\hspace*{\wd\funcname}

\begin{doc-string}
Find the token-info in \emph{token-list} that spans \emph{here-point}, if any.  If there
is no such token-info return nil.  If there is such a token-info, return a
cons cell whose car is the spanning token-info and whose cdr is a pointer
to the next token-info in the tlist.
\end{doc-string}

\vspace{1em}
\noindent
\savebox{\funcname}{\noindent\texttt{shu-cpp-token-info-replace-epoint }}
\usebox{\funcname}\emph{token-info}
\index{shu-cpp-token-info-replace-epoint} \hfill [Function]
\hspace*{\wd\funcname}

\begin{doc-string}
Replace the EPOINT of \emph{token-info} with \emph{new-epoint}
\end{doc-string}

\vspace{1em}
\noindent
\savebox{\funcname}{\noindent\texttt{shu-cpp-token-info-replace-token }}
\usebox{\funcname}\emph{token-info}
\index{shu-cpp-token-info-replace-token} \hfill [Function]
\hspace*{\wd\funcname}

\begin{doc-string}
Replace the TOKEN of \emph{token-info} with \emph{new-token}, returning the
modified \emph{token-info}
\end{doc-string}

\vspace{1em}
\noindent
\savebox{\funcname}{\noindent\texttt{shu-cpp-token-internal-parse-region }}
\usebox{\funcname}\emph{func} \emph{start} \emph{end}
\index{shu-cpp-token-internal-parse-region} \hfill [Function]

\begin{doc-string}
Internal function to do a forward or reverse parse of the region between \emph{start}
and \emph{end}.  \emph{func} holds the function to be invoked to do the parse.  This would be
either shu-cpp-tokenize-region or shu-cpp-reverse-tokenize-region.  Once the
parse is complete, the token list is shown in the Shu unit test buffer.  If any
error is detected, it is displayed at the point at which the error was
detected.
\end{doc-string}

\vspace{1em}
\noindent
\savebox{\funcname}{\noindent\texttt{shu-cpp-token-internal-tokenize-region-for-command }}
\usebox{\funcname}\emph{func}
\index{shu-cpp-token-internal-tokenize-region-for-command} \hfill [Function]
\hspace*{\wd\funcname}\emph{end}
\hspace*{\wd\funcname}\emph{limit}

\begin{doc-string}
Internal function to do a forward or reverse parse of the region between \emph{start}
and \emph{end}.  \emph{func} holds the function to be invoked to do the parse.  This would be
either shu-cpp-tokenize-region or shu-cpp-reverse-tokenize-region.  Once the
parse is complete, we check to see if an error was detected.  If an error was
detected we go to the error point and show the error.  Then we return nil to the
caller.  If no error was detected, we return the token-list to the caller.  This
is a convenient way for a command to get the token-list and not have to do anything
to display an error message if an error is encountered.  The command calls this
function and simply exits if nil is returned, knowing that the error message has
already been displayed.
\end{doc-string}

\vspace{1em}
\noindent
\savebox{\funcname}{\noindent\texttt{shu-cpp-token-set-alias }}
\usebox{\funcname}
\index{shu-cpp-token-set-alias} \hfill [Function]

\begin{doc-string}
Set the common alias names for the functions in shu-cpp-token.
These are usually the same as the function names with the leading
shu- prefix removed.
\end{doc-string}

\vspace{1em}
\noindent
\savebox{\funcname}{\noindent\texttt{shu-cpp-token-show-token-info }}
\usebox{\funcname}\emph{token-info}
\index{shu-cpp-token-show-token-info} \hfill [Function]

\begin{doc-string}
Show the data returned by one of the functions in this file that scans for tokens.
\end{doc-string}

\vspace{1em}
\noindent
\savebox{\funcname}{\noindent\texttt{shu-cpp-token-token-type-name }}
\usebox{\funcname}\emph{token-type}
\index{shu-cpp-token-token-type-name} \hfill [Function]

\begin{doc-string}
Return the name of a token type.
\end{doc-string}

\vspace{1em}
\noindent
\savebox{\funcname}{\noindent\texttt{shu-cpp-token-type-cc }}
\usebox{\funcname}
\index{shu-cpp-token-type-cc} \hfill [Constant]

\begin{doc-string}
Token type that indicates a line in which the first non-blank item is a
comment that starts in a column greater than or equal to the column defined
by shu-cpp-comment-start.  This is known as a code comment with no code present.
\end{doc-string}

\vspace{1em}
\noindent
\savebox{\funcname}{\noindent\texttt{shu-cpp-token-type-ct }}
\usebox{\funcname}
\index{shu-cpp-token-type-ct} \hfill [Constant]

\begin{doc-string}
Token type that indicates a comment
\end{doc-string}

\vspace{1em}
\noindent
\savebox{\funcname}{\noindent\texttt{shu-cpp-token-type-op }}
\usebox{\funcname}
\index{shu-cpp-token-type-op} \hfill [Constant]

\begin{doc-string}
Token type that indicates an operator
\end{doc-string}

\vspace{1em}
\noindent
\savebox{\funcname}{\noindent\texttt{shu-cpp-token-type-qt }}
\usebox{\funcname}
\index{shu-cpp-token-type-qt} \hfill [Constant]

\begin{doc-string}
Token type that indicates a quoted string
\end{doc-string}

\vspace{1em}
\noindent
\savebox{\funcname}{\noindent\texttt{shu-cpp-token-type-tp }}
\usebox{\funcname}
\index{shu-cpp-token-type-tp} \hfill [Constant]

\begin{doc-string}
Token type that indicates a template parameter.  The standard parsing does nothing
with template parameters.  Something like ``$<$int$>$`` is simply turned into three separate
tokens, ``$<$``, ``int'', and ``$>$`` (or ``$>$``, ``int'', and ``$<$`` in a reverse parse).
But some of the other transform functions will turn this list of tokens into the single
template parameter ``int''
\end{doc-string}

\vspace{1em}
\noindent
\savebox{\funcname}{\noindent\texttt{shu-cpp-token-type-uq }}
\usebox{\funcname}
\index{shu-cpp-token-type-uq} \hfill [Constant]

\begin{doc-string}
Token type that indicates an unquoted token
\end{doc-string}

\vspace{1em}
\noindent
\savebox{\funcname}{\noindent\texttt{shu-cpp-tokenize-region }}
\usebox{\funcname}\emph{start} \emph{end} \textbf{\&optional} \emph{limit}
\index{shu-cpp-tokenize-region} \hfill [Function]

\begin{doc-string}
Scan the region between \emph{start} and AND to build a list of tokens that represent the C++
code in the region.  Return a cons cell with two items in it.  The car of the cons cell
is a token-info that represents a parse error.  The cdr of the cons cell is the list of
tokens.  This list is incomplete if the car of the cons cell is not nil.  The optional
\emph{limit} argument is used to bound the scan as follows.  When we have added to the list the
first token that is beyond the point specified by \emph{limit}, we stop the scan.
\end{doc-string}

\vspace{1em}
\noindent
\savebox{\funcname}{\noindent\texttt{shu-cpp-tokenize-region-for-command }}
\usebox{\funcname}\emph{start} \emph{end}
\index{shu-cpp-tokenize-region-for-command} \hfill [Function]
\hspace*{\wd\funcname}\emph{limit}

\begin{doc-string}
Tokenize the region between \emph{start} and \emph{end} into a list of all of the C++
tokens contained therein, displaying any error message, if there is one.  If no
error, return the token list, else return nil
\end{doc-string}

\vspace{1em}
\noindent
\savebox{\funcname}{\noindent\texttt{shu-cpp-tokenize-show-list }}
\usebox{\funcname}\emph{token-list}
\index{shu-cpp-tokenize-show-list} \hfill [Function]

\begin{doc-string}
Undocumented
\end{doc-string}

\section{shu-keyring}


This is a set of functions for maintaining and querying a keyring of names,
URLs, users IDs, passwords, and related information that are maintained in an
external keyring file.

Functions allow you to find a keyring entry by name and to put one piece of
its information, such as user ID or password, in the clip board, from which it
may be pasted into a browser or other application.

The keyring file may be encrypted with GPG.  As of emacs 23, the EasyPG
package is included with the emacs distribution.  When you tell emacs to open
a file that has been encrypted with GPG, you are prompted for the passphrase
and the file is decrypted into memory.

The file keyring.txt in the usr directory is am example of a small
keyring file that has not been encrypted.  Each entry in the file consists of
a set of name value pairs.  Each value may be enclosed in quotes and must be
enclosed in quotes if it contains embedded blanks.

A single set of name value pairs starts with an opening "$<$" and is terminated
by a closing "/$>$".

Here is an example of a set of name value pairs:

\begin{verbatim}
     < name="Fred email" url=mail.google.com  id=freddy@gmail.com  pw=secret />
\end{verbatim}

The names may be arbitrary but there are six names that are recognized by the
functions in this package.  They are:

acct represents an account number

id represents a user ID

name represents the name of the entry.  This is the key that is used to find
the entry.  If no name is given, then the name of the URL is used.  If the URL
starts with "www.", the "www." is removed to form the name.  An entry that has
no name and a URL of "www.facebook.com" would have an auto generated name of
"facebook.com".

pin represents a pin number

pw represents a password

url represents a URL

To use a keying file, place the following lines in your .emacs file:

\begin{verbatim}
     (load-file "/Users/fred/.emacs.d/shu-base.elc")
     (load-file "/Users/fred/.emacs.d/shu-nvplist.elc")
     (load-file "/Users/fred/.emacs.d/shu-keyring.elc")
     (shu-keyring-set-alias)
     (setq shu-keyring-file "/Users/fred/shu/usr/keyring.txt")
\end{verbatim}

replacing "/Users/fred/shu/usr/keyring.txt" with the path to your keyring file.

All of the shu functions require shu-base.

If using the sample keyring file, Fred can now use this to log onto his gmail
account as follows.

Type M-x krurl.  This prompts for the name of the desired key.  Type "Fred em"
and hit TAB to complete.  This fills out the name as "Fred email" and puts the
URL "mail.google.com" into the clip board.  Open a browser and paste the URL
into it to go to gmail.  At gmail, select login.  In emacs type M-x krid.
When prompted for the key, use the up arrow to retrieve the last key used,
which will be "Fred email".  This puts "freddy@gmail.com" into the clip board
for conveniently pasting into the gmail widow.  To obtain the password, type
M-x krpw.  This puts the password into the clip board from which it may be
pasted into the gmail widow.


\subsection{List of functions by alias name}

A list of aliases and associated function names.



\vspace{1em}
\noindent
\savebox{\funcname}{\noindent\texttt{kracct }}
\usebox{\funcname}
\index{kracct} \hfill [Command]\\%
 (Function: shu-keyring-get-acct)

\begin{doc-string}
Find the account for an entry in the keyring file.  This displays the entry in the message
area and puts the password into the kill ring so that it can be yanked or pasted into the application
requesting it.
\end{doc-string}

\vspace{1em}
\noindent
\savebox{\funcname}{\noindent\texttt{krfn }}
\usebox{\funcname}
\index{krfn} \hfill [Command]\\%
 (Function: shu-keyring-get-file)

\begin{doc-string}
Display the name of the keyring file, if any.  This is useful if you are
getting unexpected results from some of the query functions that look up keyring
information.  Perhaps the unexpected results come from the fact that you are
using the wrong keyring file.
\end{doc-string}

\vspace{1em}
\noindent
\savebox{\funcname}{\noindent\texttt{krid }}
\usebox{\funcname}
\index{krid} \hfill [Command]\\%
 (Function: shu-keyring-get-id)

\begin{doc-string}
Find the User Id for an entry in the keyring file.  This displays the entry
in the message area and puts the user Id into the kill ring so that it can be
yanked into a buffer or pasted into the application requesting it.
\end{doc-string}

\vspace{1em}
\noindent
\savebox{\funcname}{\noindent\texttt{krpin }}
\usebox{\funcname}
\index{krpin} \hfill [Command]\\%
 (Function: shu-keyring-get-pin)

\begin{doc-string}
Find the pin for an entry in the keyring file.  This displays the entry in
the message area and puts the pin into the kill ring so that it can be yanked
into a buffer or pasted into the application requesting it.
\end{doc-string}

\vspace{1em}
\noindent
\savebox{\funcname}{\noindent\texttt{krpw }}
\usebox{\funcname}
\index{krpw} \hfill [Command]\\%
 (Function: shu-keyring-get-pw)

\begin{doc-string}
Find the password for an entry in the keyring file.  This displays the entry
(without the password) in the message area and puts the password into the kill
ring so that it can be yanked into a buffer or pasted into the application
requesting it.
\end{doc-string}

\vspace{1em}
\noindent
\savebox{\funcname}{\noindent\texttt{krurl }}
\usebox{\funcname}
\index{krurl} \hfill [Command]\\%
 (Function: shu-keyring-get-url)

\begin{doc-string}
Find the url for an entry in the keyring file.  This displays the entry in
the message area and puts the url into the kill ring so that it can be yanked
into a buffer or pasted into the application requesting it.
\end{doc-string}

\vspace{1em}
\noindent
\savebox{\funcname}{\noindent\texttt{krvf }}
\usebox{\funcname}
\index{krvf} \hfill [Command]\\%
 (Function: shu-keyring-verify-file)

\begin{doc-string}
Parse and verify the keyring file, displaying the result of the operation in the
keyring buffer (\emph\{**shu-keyring**\}).  If one of the queries for a url or other
piece of information is unable to find the requested information, it could be
that you have the wrong keyring file or that there is a syntax error in the
keyring file.  shu-keyring-get-file (alias krfn) displays the name of the
keyring file.  This function parses the keyring file.  After the operation. look
into the keyring buffer (\emph\{**shu-keyring**\}) to see if there are any complaints
about syntax errors in the file.
\end{doc-string}

\subsection{List of functions and variables}

List of functions and variable definitions in this package.



\vspace{1em}
\noindent
\savebox{\funcname}{\noindent\texttt{shu-keyring-account-name }}
\usebox{\funcname}
\index{shu-keyring-account-name} \hfill [Constant]

\begin{doc-string}
Key word that denotes a name.
\end{doc-string}

\vspace{1em}
\noindent
\savebox{\funcname}{\noindent\texttt{shu-keyring-add-values-to-index }}
\usebox{\funcname}\emph{index} \emph{vlist} \emph{item}
\index{shu-keyring-add-values-to-index} \hfill [Function]

\begin{doc-string}
Add a set of keys \emph{vlist} to \emph{index} for \emph{item}.  Keys within the item are filtered for
duplicates.  But this does not prevent two different items from sharing the same key,
although it would be unusual in a keyring.
\end{doc-string}

\vspace{1em}
\noindent
\savebox{\funcname}{\noindent\texttt{shu-keyring-buffer-name }}
\usebox{\funcname}
\index{shu-keyring-buffer-name} \hfill [Constant]

\begin{doc-string}
The name of the buffer into which keyring diagnostics and messages
are recorded.
\end{doc-string}

\vspace{1em}
\noindent
\savebox{\funcname}{\noindent\texttt{shu-keyring-clear-index }}
\usebox{\funcname}
\index{shu-keyring-clear-index} \hfill [Function]

\begin{doc-string}
This is called from after-save-hook to clear the keyring index if the keyring file is saved.
The keyring index is built the first time it is needed and kept in memory thereafter.  But we
must refresh the index if the keyring file is modified.  The easiest way to do this is to clear
the index when the keyring file is modified.  The next time the index is needed it will be
recreated.
\end{doc-string}

\vspace{1em}
\noindent
\savebox{\funcname}{\noindent\texttt{shu-keyring-file }}
\usebox{\funcname}
\index{shu-keyring-file} \hfill [Custom]

\begin{doc-string}
Text file in which urls, names, and passwords are stored.
\end{doc-string}

\vspace{1em}
\noindent
\savebox{\funcname}{\noindent\texttt{shu-keyring-find-index-duplicates }}
\usebox{\funcname}\emph{index}
\index{shu-keyring-find-index-duplicates} \hfill [Function]

\begin{doc-string}
Find any duplicates in the keyring index.  When the index is built we filter duplicate
keys for the same item.  But there could be two different items with the same key.  This
function returns TRUE if two or more items have the same key.  The index must be in sorted
order by key value before this function is called.
\end{doc-string}

\vspace{1em}
\noindent
\savebox{\funcname}{\noindent\texttt{shu-keyring-get-acct }}
\usebox{\funcname}
\index{shu-keyring-get-acct} \hfill [Command]\\%
 (Alias: kracct)

\begin{doc-string}
Find the account for an entry in the keyring file.  This displays the entry in the message
area and puts the password into the kill ring so that it can be yanked or pasted into the application
requesting it.
\end{doc-string}

\vspace{1em}
\noindent
\savebox{\funcname}{\noindent\texttt{shu-keyring-get-field }}
\usebox{\funcname}\emph{name}
\index{shu-keyring-get-field} \hfill [Function]

\begin{doc-string}
Fetch the value of a named field from the keyring.  Prompt the user with a completing-read
for the field that identifies the key.  Use the key to find the item.  Find the value of the named
key value pair within the item.  Put the value in the kill-ring and also return it to the caller.
\end{doc-string}

\vspace{1em}
\noindent
\savebox{\funcname}{\noindent\texttt{shu-keyring-get-file }}
\usebox{\funcname}
\index{shu-keyring-get-file} \hfill [Command]\\%
 (Alias: krfn)

\begin{doc-string}
Display the name of the keyring file, if any.  This is useful if you are
getting unexpected results from some of the query functions that look up keyring
information.  Perhaps the unexpected results come from the fact that you are
using the wrong keyring file.
\end{doc-string}

\vspace{1em}
\noindent
\savebox{\funcname}{\noindent\texttt{shu-keyring-get-id }}
\usebox{\funcname}
\index{shu-keyring-get-id} \hfill [Command]\\%
 (Alias: krid)

\begin{doc-string}
Find the User Id for an entry in the keyring file.  This displays the entry
in the message area and puts the user Id into the kill ring so that it can be
yanked into a buffer or pasted into the application requesting it.
\end{doc-string}

\vspace{1em}
\noindent
\savebox{\funcname}{\noindent\texttt{shu-keyring-get-pin }}
\usebox{\funcname}
\index{shu-keyring-get-pin} \hfill [Command]\\%
 (Alias: krpin)

\begin{doc-string}
Find the pin for an entry in the keyring file.  This displays the entry in
the message area and puts the pin into the kill ring so that it can be yanked
into a buffer or pasted into the application requesting it.
\end{doc-string}

\vspace{1em}
\noindent
\savebox{\funcname}{\noindent\texttt{shu-keyring-get-pw }}
\usebox{\funcname}
\index{shu-keyring-get-pw} \hfill [Command]\\%
 (Alias: krpw)

\begin{doc-string}
Find the password for an entry in the keyring file.  This displays the entry
(without the password) in the message area and puts the password into the kill
ring so that it can be yanked into a buffer or pasted into the application
requesting it.
\end{doc-string}

\vspace{1em}
\noindent
\savebox{\funcname}{\noindent\texttt{shu-keyring-get-url }}
\usebox{\funcname}
\index{shu-keyring-get-url} \hfill [Command]\\%
 (Alias: krurl)

\begin{doc-string}
Find the url for an entry in the keyring file.  This displays the entry in
the message area and puts the url into the kill ring so that it can be yanked
into a buffer or pasted into the application requesting it.
\end{doc-string}

\vspace{1em}
\noindent
\savebox{\funcname}{\noindent\texttt{shu-keyring-history }}
\usebox{\funcname}
\index{shu-keyring-history} \hfill [Variable]

\begin{doc-string}
The history list used by completing-read when asking the user for a key to an
entry in the keyring file.
\end{doc-string}

\vspace{1em}
\noindent
\savebox{\funcname}{\noindent\texttt{shu-keyring-id-name }}
\usebox{\funcname}
\index{shu-keyring-id-name} \hfill [Constant]

\begin{doc-string}
Key word that denotes a user ID.
\end{doc-string}

\vspace{1em}
\noindent
\savebox{\funcname}{\noindent\texttt{shu-keyring-in-index }}
\usebox{\funcname}\emph{index} \emph{item} \emph{value}
\index{shu-keyring-in-index} \hfill [Function]

\begin{doc-string}
Return true if the \emph{index} already contains the \emph{value} for this \emph{item}.
\end{doc-string}

\vspace{1em}
\noindent
\savebox{\funcname}{\noindent\texttt{shu-keyring-index }}
\usebox{\funcname}
\index{shu-keyring-index} \hfill [Variable]

\begin{doc-string}
The variable that points to the in-memory keyring index.
\end{doc-string}

\vspace{1em}
\noindent
\savebox{\funcname}{\noindent\texttt{shu-keyring-name-name }}
\usebox{\funcname}
\index{shu-keyring-name-name} \hfill [Constant]

\begin{doc-string}
Key word that denotes a name.
\end{doc-string}

\vspace{1em}
\noindent
\savebox{\funcname}{\noindent\texttt{shu-keyring-parse-keyring-file }}
\usebox{\funcname}
\index{shu-keyring-parse-keyring-file} \hfill [Command]

\begin{doc-string}
Parse the keyring file and create the in-memory index if the keyring file
contains no duplicate keys.
\end{doc-string}

\vspace{1em}
\noindent
\savebox{\funcname}{\noindent\texttt{shu-keyring-pin-name }}
\usebox{\funcname}
\index{shu-keyring-pin-name} \hfill [Constant]

\begin{doc-string}
Key word that denotes a PIN.
\end{doc-string}

\vspace{1em}
\noindent
\savebox{\funcname}{\noindent\texttt{shu-keyring-pw-name }}
\usebox{\funcname}
\index{shu-keyring-pw-name} \hfill [Constant]

\begin{doc-string}
Key word that denotes a password.
\end{doc-string}

\vspace{1em}
\noindent
\savebox{\funcname}{\noindent\texttt{shu-keyring-set-alias }}
\usebox{\funcname}
\index{shu-keyring-set-alias} \hfill [Function]

\begin{doc-string}
Set the common alias names for the functions in shu-keyring.
These are generally the same as the function names with the leading
shu- prefix removed.  But in this case the names are drastically shortened
to make them easier to type.
\end{doc-string}

\vspace{1em}
\noindent
\savebox{\funcname}{\noindent\texttt{shu-keyring-show-index }}
\usebox{\funcname}\emph{index}
\index{shu-keyring-show-index} \hfill [Function]

\begin{doc-string}
Print the keyring index
\end{doc-string}

\vspace{1em}
\noindent
\savebox{\funcname}{\noindent\texttt{shu-keyring-show-name-url }}
\usebox{\funcname}\emph{type} \emph{item}
\index{shu-keyring-show-name-url} \hfill [Function]

\begin{doc-string}
Show in the message area the name, url, or both of a keyring entry.  Also prefix
the message with the upper case type, which is the type of the entry that has been
placed in the clipboard, (PW, ID, etc.)
\end{doc-string}

\vspace{1em}
\noindent
\savebox{\funcname}{\noindent\texttt{shu-keyring-update-index }}
\usebox{\funcname}\emph{index} \emph{item}
\index{shu-keyring-update-index} \hfill [Function]

\begin{doc-string}
Extract the keys from a keyring item and add them to the keyring index.
\end{doc-string}

\vspace{1em}
\noindent
\savebox{\funcname}{\noindent\texttt{shu-keyring-url-name }}
\usebox{\funcname}
\index{shu-keyring-url-name} \hfill [Constant]

\begin{doc-string}
Key word that denotes a URL.
\end{doc-string}

\vspace{1em}
\noindent
\savebox{\funcname}{\noindent\texttt{shu-keyring-values-to-string }}
\usebox{\funcname}\emph{values}
\index{shu-keyring-values-to-string} \hfill [Function]

\begin{doc-string}
Turn a list of values into a single string of values separated by slashes.
\end{doc-string}

\vspace{1em}
\noindent
\savebox{\funcname}{\noindent\texttt{shu-keyring-verify-file }}
\usebox{\funcname}
\index{shu-keyring-verify-file} \hfill [Command]\\%
 (Alias: krvf)

\begin{doc-string}
Parse and verify the keyring file, displaying the result of the operation in the
keyring buffer (\emph\{**shu-keyring**\}).  If one of the queries for a url or other
piece of information is unable to find the requested information, it could be
that you have the wrong keyring file or that there is a syntax error in the
keyring file.  shu-keyring-get-file (alias krfn) displays the name of the
keyring file.  This function parses the keyring file.  After the operation. look
into the keyring buffer (\emph\{**shu-keyring**\}) to see if there are any complaints
about syntax errors in the file.
\end{doc-string}

\section{shu-misc}



A miscellaneous collection of useful functions


\subsection{List of functions by alias name}

A list of aliases and associated function names.



\vspace{1em}
\noindent
\savebox{\funcname}{\noindent\texttt{comma-names-to-letter }}
\usebox{\funcname}
\index{comma-names-to-letter} \hfill [Command]\\%
 (Function: shu-comma-names-to-letter)

\begin{doc-string}
In a list of names, change all occurrences
of Lastname, Firstname to an empty Latex letter.
Position to the start of the file and invoke once.
\end{doc-string}

\vspace{1em}
\noindent
\savebox{\funcname}{\noindent\texttt{diff-commits }}
\usebox{\funcname}\emph{commit-range}
\index{diff-commits} \hfill [Command]\\%
 (Function: shu-git-diff-commits)

\begin{doc-string}
In a buffer that is a numbered git log, query for a range string, find the two
commits, and put into the kill ring a git diff command specifying the two commits.

For example, given the following two numbered commits:

\begin{verbatim}
    31. commit 38f25b6769385dbc3526f32a75b97218cb4a6754
    33. commit 052ee7f4297206f08d44466934f1a52678da6ec9
\end{verbatim}

if the commit range specified is either ``31.33'' or ``31+2'', then the following
is put into the kill ring:

\begin{verbatim}
    ``git diff -b 38f25b6769385dbc3526f32a75b97218cb4a6754..052ee7f4297206f08d44466934f1a52678da6ec9 ``
\end{verbatim}
\end{doc-string}

\vspace{1em}
\noindent
\savebox{\funcname}{\noindent\texttt{dup }}
\usebox{\funcname}
\index{dup} \hfill [Command]\\%
 (Function: shu-dup)

\begin{doc-string}
Insert a duplicate of the current line, following it.
\end{doc-string}

\vspace{1em}
\noindent
\savebox{\funcname}{\noindent\texttt{eld }}
\usebox{\funcname}
\index{eld} \hfill [Command]\\%
 (Function: shu-save-and-load)

\begin{doc-string}
Save and load the current file as a .el file.
\end{doc-string}

\vspace{1em}
\noindent
\savebox{\funcname}{\noindent\texttt{gd }}
\usebox{\funcname}
\index{gd} \hfill [Command]\\%
 (Function: shu-gd)

\begin{doc-string}
While in dired, put the full path to the current directory in the kill ring
\end{doc-string}

\vspace{1em}
\noindent
\savebox{\funcname}{\noindent\texttt{gf }}
\usebox{\funcname}
\index{gf} \hfill [Command]\\%
 (Function: shu-gf)

\begin{doc-string}
While in dired, put the full path to the current file in the kill ring
\end{doc-string}

\vspace{1em}
\noindent
\savebox{\funcname}{\noindent\texttt{gfc }}
\usebox{\funcname}
\index{gfc} \hfill [Command]\\%
 (Function: shu-gfc)

\begin{doc-string}
While in a file buffer, put both the current line number and
column number and the name of the current file into the kill ring
in the form of ``foo.cpp:123:2''.
\end{doc-string}

\vspace{1em}
\noindent
\savebox{\funcname}{\noindent\texttt{gfl }}
\usebox{\funcname}
\index{gfl} \hfill [Command]\\%
 (Function: shu-gfl)

\begin{doc-string}
While in a file buffer, put both the current line number and the name of the current
file into the kill ring in the form of ``line 1234 of foo.cpp''.
\end{doc-string}

\vspace{1em}
\noindent
\savebox{\funcname}{\noindent\texttt{gfn }}
\usebox{\funcname}
\index{gfn} \hfill [Command]\\%
 (Function: shu-gfn)

\begin{doc-string}
While in a file buffer, put the name of the current file into the kill ring.
\end{doc-string}

\vspace{1em}
\noindent
\savebox{\funcname}{\noindent\texttt{gquote }}
\usebox{\funcname}
\index{gquote} \hfill [Command]\\%
 (Function: shu-gquote)

\begin{doc-string}
Insert a LaTeX quote environment and position the cursor for typing the quote.
\end{doc-string}

\vspace{1em}
\noindent
\savebox{\funcname}{\noindent\texttt{new-ert }}
\usebox{\funcname}\emph{func-name}
\index{new-ert} \hfill [Command]\\%
 (Function: shu-new-ert)

\begin{doc-string}
Insert at point a skeleton lisp ert unit test.  Prompt is issued for the
function name.
\end{doc-string}

\vspace{1em}
\noindent
\savebox{\funcname}{\noindent\texttt{new-latex }}
\usebox{\funcname}
\index{new-latex} \hfill [Command]\\%
 (Function: shu-new-latex)

\begin{doc-string}
Build a skeleton, empty LaTeX file.
\end{doc-string}

\vspace{1em}
\noindent
\savebox{\funcname}{\noindent\texttt{new-lisp }}
\usebox{\funcname}\emph{func-name}
\index{new-lisp} \hfill [Command]\\%
 (Function: shu-new-lisp)

\begin{doc-string}
Insert at point a skeleton lisp function.  Prompt is issued for the function
name.
\end{doc-string}

\vspace{1em}
\noindent
\savebox{\funcname}{\noindent\texttt{number-commits }}
\usebox{\funcname}
\index{number-commits} \hfill [Command]\\%
 (Function: shu-git-number-commits)

\begin{doc-string}
In a git log buffer, number all of the commits with zero being the most
recent.

It is possible to refer to commits by their SHA-1 hash.  If you want to see the
difference between two commits you can ask git to show you the difference by
specifying the commit hash of each one.  But this is cumbersome.  It involves
copying and pasting two SHA-1 hashes.  Once the commits are numbered, then
\emph{shu-git-diff-commits} may be used to diff two commits by number.  See the
documentation for \emph{shu-git-diff-commits} for further information.

This function counts as a commit any instance of ``commit'' that starts at the
beginning of a line and is followed by some white space and a forty character
hexadecimal number.  Returns the count of the number of commits found.
\end{doc-string}

\vspace{1em}
\noindent
\savebox{\funcname}{\noindent\texttt{of }}
\usebox{\funcname}
\index{of} \hfill [Command]\\%
 (Function: shu-of)

\begin{doc-string}
While in dired, open the current file (Mac OS X only)
\end{doc-string}

\vspace{1em}
\noindent
\savebox{\funcname}{\noindent\texttt{remove-test-names }}
\usebox{\funcname}
\index{remove-test-names} \hfill [Command]\\%
 (Function: shu-remove-test-names)

\begin{doc-string}
Remove from a file all lines that contain file names that end in .t.cpp
\end{doc-string}

\vspace{1em}
\noindent
\savebox{\funcname}{\noindent\texttt{reverse-comma-names }}
\usebox{\funcname}
\index{reverse-comma-names} \hfill [Command]\\%
 (Function: shu-reverse-comma-names)

\begin{doc-string}
In a list of names, change all occurrences
of Lastname, Firstname to Firstname Lastname.
Position to the start of the file and invoke once.
\end{doc-string}

\vspace{1em}
\noindent
\savebox{\funcname}{\noindent\texttt{set-dos-eol }}
\usebox{\funcname}
\index{set-dos-eol} \hfill [Command]\\%
 (Function: shu-set-dos-eol)

\begin{doc-string}
Set the end of line delimiter to be the DOS standard (CRLF).
\end{doc-string}

\vspace{1em}
\noindent
\savebox{\funcname}{\noindent\texttt{set-unix-eol }}
\usebox{\funcname}
\index{set-unix-eol} \hfill [Command]\\%
 (Function: shu-set-unix-eol)

\begin{doc-string}
Set the end of line delimiter to be the Unix standard (LF).
\end{doc-string}

\vspace{1em}
\noindent
\savebox{\funcname}{\noindent\texttt{trim-trailing-blanks }}
\usebox{\funcname}
\index{trim-trailing-blanks} \hfill [Command]\\%
 (Function: shu-trim-trailing-blanks)

\begin{doc-string}
Eliminate whitespace at ends of all lines in the current buffer.
\end{doc-string}

\vspace{1em}
\noindent
\savebox{\funcname}{\noindent\texttt{winpath }}
\usebox{\funcname}\emph{start} \emph{end}
\index{winpath} \hfill [Command]\\%
 (Function: shu-winpath)

\begin{doc-string}
Take marked region, put in kill ring, changing / to \.
This makes it a valid path on windows machines.
\end{doc-string}

\subsection{List of functions and variables}

List of functions and variable definitions in this package.



\vspace{1em}
\noindent
\savebox{\funcname}{\noindent\texttt{shu-comma-names-to-letter }}
\usebox{\funcname}
\index{shu-comma-names-to-letter} \hfill [Command]\\%
 (Alias: comma-names-to-letter)

\begin{doc-string}
In a list of names, change all occurrences
of Lastname, Firstname to an empty Latex letter.
Position to the start of the file and invoke once.
\end{doc-string}

\vspace{1em}
\noindent
\savebox{\funcname}{\noindent\texttt{shu-conditional-find-file }}
\usebox{\funcname}\emph{file-name}
\index{shu-conditional-find-file} \hfill [Command]

\begin{doc-string}
Make the buffer for \emph{file-name} the current buffer.  If \emph{file-name} is already
loaded into a buffer, then make that the current buffer.  If \emph{file-name} is not
loaded into a buffer, load the file into a buffer and make that the current
buffer.  Return true if this function created the buffer, nil otherwise.

This function is intended to handle the situation in which a function wants
to visit the contents of several files but does not want to leave behind a
lot of file buffers that it created.

If this function returns true, then the calling function should kill the
buffer when it is finished with it.
\end{doc-string}

\vspace{1em}
\noindent
\savebox{\funcname}{\noindent\texttt{shu-disabled-quit }}
\usebox{\funcname}
\index{shu-disabled-quit} \hfill [Command]

\begin{doc-string}
Explain that C-x C-c no longer kills emacs.  Must M-x quit instead.
Far too often, I hit C-x C-c by mistake and emacs vanishes.  So I map
C-x C-c to this function and use an explicit M-x quit to exit emacs.
\end{doc-string}

\vspace{1em}
\noindent
\savebox{\funcname}{\noindent\texttt{shu-dup }}
\usebox{\funcname}
\index{shu-dup} \hfill [Command]\\%
 (Alias: dup)

\begin{doc-string}
Insert a duplicate of the current line, following it.
\end{doc-string}

\vspace{1em}
\noindent
\savebox{\funcname}{\noindent\texttt{shu-eob }}
\usebox{\funcname}
\index{shu-eob} \hfill [Command]

\begin{doc-string}
Go to end of buffer without setting mark.  Like end-of-buffer
but does not set mark - just goes there.
\end{doc-string}

\vspace{1em}
\noindent
\savebox{\funcname}{\noindent\texttt{shu-find-numbered-commit }}
\usebox{\funcname}\emph{commit-number}
\index{shu-find-numbered-commit} \hfill [Function]

\begin{doc-string}
Search through a numbered git commit log looking for the commit whose number is
\emph{commit-number}.  Return the SHA-1 hash of the commit if the commit number is found.
Return nil if no commit with the given number is found.
The commit log is assume to have been numbered by shu-git-number-commits.
\end{doc-string}

\vspace{1em}
\noindent
\savebox{\funcname}{\noindent\texttt{shu-fix-times }}
\usebox{\funcname}
\index{shu-fix-times} \hfill [Command]

\begin{doc-string}
Go through a buffer that contains timestamps of the form
\begin{verbatim}
     YYYY-MM-DDTHHMMSS.DDD
\end{verbatim}
converting them to the form
\begin{verbatim}
     YYYY-MM-DD HH:MM:SS.DDD
\end{verbatim}
The latter is a format that Microsoft Excel can import.
\end{doc-string}

\vspace{1em}
\noindent
\savebox{\funcname}{\noindent\texttt{shu-forward-line }}
\usebox{\funcname}
\index{shu-forward-line} \hfill [Function]

\begin{doc-string}
Move forward by one line.  If there is a next line, point it moved into
it.  If there are no more lines, a new one is created.
\end{doc-string}

\vspace{1em}
\noindent
\savebox{\funcname}{\noindent\texttt{shu-gd }}
\usebox{\funcname}
\index{shu-gd} \hfill [Command]\\%
 (Alias: gd)

\begin{doc-string}
While in dired, put the full path to the current directory in the kill ring
\end{doc-string}

\vspace{1em}
\noindent
\savebox{\funcname}{\noindent\texttt{shu-get-current-line }}
\usebox{\funcname}
\index{shu-get-current-line} \hfill [Function]

\begin{doc-string}
Return the current line in the buffer as a string
\end{doc-string}

\vspace{1em}
\noindent
\savebox{\funcname}{\noindent\texttt{shu-gf }}
\usebox{\funcname}
\index{shu-gf} \hfill [Command]\\%
 (Alias: gf)

\begin{doc-string}
While in dired, put the full path to the current file in the kill ring
\end{doc-string}

\vspace{1em}
\noindent
\savebox{\funcname}{\noindent\texttt{shu-gfc }}
\usebox{\funcname}
\index{shu-gfc} \hfill [Command]\\%
 (Alias: gfc)

\begin{doc-string}
While in a file buffer, put both the current line number and
column number and the name of the current file into the kill ring
in the form of ``foo.cpp:123:2''.
\end{doc-string}

\vspace{1em}
\noindent
\savebox{\funcname}{\noindent\texttt{shu-gfl }}
\usebox{\funcname}
\index{shu-gfl} \hfill [Command]\\%
 (Alias: gfl)

\begin{doc-string}
While in a file buffer, put both the current line number and the name of the current
file into the kill ring in the form of ``line 1234 of foo.cpp''.
\end{doc-string}

\vspace{1em}
\noindent
\savebox{\funcname}{\noindent\texttt{shu-gfn }}
\usebox{\funcname}
\index{shu-gfn} \hfill [Command]\\%
 (Alias: gfn)

\begin{doc-string}
While in a file buffer, put the name of the current file into the kill ring.
\end{doc-string}

\vspace{1em}
\noindent
\savebox{\funcname}{\noindent\texttt{shu-git-diff-commits }}
\usebox{\funcname}\emph{commit-range}
\index{shu-git-diff-commits} \hfill [Command]\\%
 (Alias: diff-commits)

\begin{doc-string}
In a buffer that is a numbered git log, query for a range string, find the two
commits, and put into the kill ring a git diff command specifying the two commits.

For example, given the following two numbered commits:

\begin{verbatim}
    31. commit 38f25b6769385dbc3526f32a75b97218cb4a6754
    33. commit 052ee7f4297206f08d44466934f1a52678da6ec9
\end{verbatim}

if the commit range specified is either ``31.33'' or ``31+2'', then the following
is put into the kill ring:

\begin{verbatim}
    ``git diff -b 38f25b6769385dbc3526f32a75b97218cb4a6754..052ee7f4297206f08d44466934f1a52678da6ec9 ``
\end{verbatim}
\end{doc-string}

\vspace{1em}
\noindent
\savebox{\funcname}{\noindent\texttt{shu-git-find-short-hash }}
\usebox{\funcname}\emph{hash}
\index{shu-git-find-short-hash} \hfill [Function]

\begin{doc-string}
Return the git short hash for the \emph{hash} supplied as an argument.  Return nil
if the given \emph{hash} is not a valid git revision.
\end{doc-string}

\vspace{1em}
\noindent
\savebox{\funcname}{\noindent\texttt{shu-git-number-commits }}
\usebox{\funcname}
\index{shu-git-number-commits} \hfill [Command]\\%
 (Alias: number-commits)

\begin{doc-string}
In a git log buffer, number all of the commits with zero being the most
recent.

It is possible to refer to commits by their SHA-1 hash.  If you want to see the
difference between two commits you can ask git to show you the difference by
specifying the commit hash of each one.  But this is cumbersome.  It involves
copying and pasting two SHA-1 hashes.  Once the commits are numbered, then
\emph{shu-git-diff-commits} may be used to diff two commits by number.  See the
documentation for \emph{shu-git-diff-commits} for further information.

This function counts as a commit any instance of ``commit'' that starts at the
beginning of a line and is followed by some white space and a forty character
hexadecimal number.  Returns the count of the number of commits found.
\end{doc-string}

\vspace{1em}
\noindent
\savebox{\funcname}{\noindent\texttt{shu-gquote }}
\usebox{\funcname}
\index{shu-gquote} \hfill [Command]\\%
 (Alias: gquote)

\begin{doc-string}
Insert a LaTeX quote environment and position the cursor for typing the quote.
\end{doc-string}

\vspace{1em}
\noindent
\savebox{\funcname}{\noindent\texttt{shu-internal-new-lisp }}
\usebox{\funcname}\emph{func-type} \emph{func-name}
\index{shu-internal-new-lisp} \hfill [Command]
\hspace*{\wd\funcname}\emph{interactive}

\begin{doc-string}
Insert at point a skeleton lisp function of type \emph{func-type} whose name is
\emph{func-name}.  \emph{func-type} is not examined in any way but is only useful if its
value is ``defun'', ``defmacro'', ``ert-deftest'', etc.  If \emph{interactive} is
true, the function is interactive.
\end{doc-string}

\vspace{1em}
\noindent
\savebox{\funcname}{\noindent\texttt{shu-kill-current-buffer }}
\usebox{\funcname}
\index{shu-kill-current-buffer} \hfill [Command]

\begin{doc-string}
Kills the current buffer.
\end{doc-string}

\vspace{1em}
\noindent
\savebox{\funcname}{\noindent\texttt{shu-local-replace }}
\usebox{\funcname}\emph{from-string} \emph{to-string}
\index{shu-local-replace} \hfill [Function]

\begin{doc-string}
Replaces \emph{from-string} with \emph{to-string} anywhere found in the buffer.
This is like replace-string except that it is intended to be called
by lisp programs.  Note that this function does not alter the value of
case-fold-search.  The user should set it before calling this function.
\end{doc-string}

\vspace{1em}
\noindent
\savebox{\funcname}{\noindent\texttt{shu-misc-set-alias }}
\usebox{\funcname}
\index{shu-misc-set-alias} \hfill [Function]

\begin{doc-string}
Set the common alias names for the functions in shu-misc.
These are generally the same as the function names with the leading
shu- prefix removed.
\end{doc-string}

\vspace{1em}
\noindent
\savebox{\funcname}{\noindent\texttt{shu-move-down }}
\usebox{\funcname}\emph{arg}
\index{shu-move-down} \hfill [Command]

\begin{doc-string}
Move point vertically down.  Whitespace in any direction is made if
necessary.  New lines will be added at the end of a file and lines that are
too short will be expanded as necessary.
\end{doc-string}

\vspace{1em}
\noindent
\savebox{\funcname}{\noindent\texttt{shu-new-ert }}
\usebox{\funcname}\emph{func-name}
\index{shu-new-ert} \hfill [Command]\\%
 (Alias: new-ert)

\begin{doc-string}
Insert at point a skeleton lisp ert unit test.  Prompt is issued for the
function name.
\end{doc-string}

\vspace{1em}
\noindent
\savebox{\funcname}{\noindent\texttt{shu-new-latex }}
\usebox{\funcname}
\index{shu-new-latex} \hfill [Command]\\%
 (Alias: new-latex)

\begin{doc-string}
Build a skeleton, empty LaTeX file.
\end{doc-string}

\vspace{1em}
\noindent
\savebox{\funcname}{\noindent\texttt{shu-new-lisp }}
\usebox{\funcname}\emph{func-name}
\index{shu-new-lisp} \hfill [Command]\\%
 (Alias: new-lisp)

\begin{doc-string}
Insert at point a skeleton lisp function.  Prompt is issued for the function
name.
\end{doc-string}

\vspace{1em}
\noindent
\savebox{\funcname}{\noindent\texttt{shu-of }}
\usebox{\funcname}
\index{shu-of} \hfill [Command]\\%
 (Alias: of)

\begin{doc-string}
While in dired, open the current file (Mac OS X only)
\end{doc-string}

\vspace{1em}
\noindent
\savebox{\funcname}{\noindent\texttt{shu-put-line-near-top }}
\usebox{\funcname}
\index{shu-put-line-near-top} \hfill [Command]

\begin{doc-string}
Take the line containing point and position it approximately five lines
from the top of the current window.
\end{doc-string}

\vspace{1em}
\noindent
\savebox{\funcname}{\noindent\texttt{shu-quit }}
\usebox{\funcname}
\index{shu-quit} \hfill [Command]

\begin{doc-string}
Invoke save-buffers-kill-emacs.  This is the function normally
invoked by C-x C-c
\end{doc-string}

\vspace{1em}
\noindent
\savebox{\funcname}{\noindent\texttt{shu-remove-test-names }}
\usebox{\funcname}
\index{shu-remove-test-names} \hfill [Command]\\%
 (Alias: remove-test-names)

\begin{doc-string}
Remove from a file all lines that contain file names that end in .t.cpp
\end{doc-string}

\vspace{1em}
\noindent
\savebox{\funcname}{\noindent\texttt{shu-reverse-comma-names }}
\usebox{\funcname}
\index{shu-reverse-comma-names} \hfill [Command]\\%
 (Alias: reverse-comma-names)

\begin{doc-string}
In a list of names, change all occurrences
of Lastname, Firstname to Firstname Lastname.
Position to the start of the file and invoke once.
\end{doc-string}

\vspace{1em}
\noindent
\savebox{\funcname}{\noindent\texttt{shu-save-and-load }}
\usebox{\funcname}
\index{shu-save-and-load} \hfill [Command]\\%
 (Alias: eld)

\begin{doc-string}
Save and load the current file as a .el file.
\end{doc-string}

\vspace{1em}
\noindent
\savebox{\funcname}{\noindent\texttt{shu-set-buffer-eol-type }}
\usebox{\funcname}\emph{eol-type}
\index{shu-set-buffer-eol-type} \hfill [Function]

\begin{doc-string}
Define what the end of line delimiter is in a text file.
\end{doc-string}

\vspace{1em}
\noindent
\savebox{\funcname}{\noindent\texttt{shu-set-dos-eol }}
\usebox{\funcname}
\index{shu-set-dos-eol} \hfill [Command]\\%
 (Alias: set-dos-eol)

\begin{doc-string}
Set the end of line delimiter to be the DOS standard (CRLF).
\end{doc-string}

\vspace{1em}
\noindent
\savebox{\funcname}{\noindent\texttt{shu-set-mac-eol }}
\usebox{\funcname}
\index{shu-set-mac-eol} \hfill [Command]

\begin{doc-string}
Set the end of line delimiter to be the Mac standard (CR).
\end{doc-string}

\vspace{1em}
\noindent
\savebox{\funcname}{\noindent\texttt{shu-set-unix-eol }}
\usebox{\funcname}
\index{shu-set-unix-eol} \hfill [Command]\\%
 (Alias: set-unix-eol)

\begin{doc-string}
Set the end of line delimiter to be the Unix standard (LF).
\end{doc-string}

\vspace{1em}
\noindent
\savebox{\funcname}{\noindent\texttt{shu-shift-line }}
\usebox{\funcname}\emph{count}
\index{shu-shift-line} \hfill [Command]

\begin{doc-string}
Shift a line of text left or right by \emph{count} positions.  Shift right
if \emph{count} is positive, left if \emph{count} is negative.  Shifting left only
eliminates whitespace.  If there is a non-whitespace character in column
5, then shift by -10 will only shift left 4.
\end{doc-string}

\vspace{1em}
\noindent
\savebox{\funcname}{\noindent\texttt{shu-shift-region-of-text }}
\usebox{\funcname}\emph{count} \emph{start} \emph{end}
\index{shu-shift-region-of-text} \hfill [Command]

\begin{doc-string}
Shift a region of text left or right.  The test to be shifted is defined
by the bounds of lines containing point and mark.  The shift count is
read from the minibuffer.
\end{doc-string}

\vspace{1em}
\noindent
\savebox{\funcname}{\noindent\texttt{shu-shift-single-line }}
\usebox{\funcname}\emph{count}
\index{shu-shift-single-line} \hfill [Function]

\begin{doc-string}
Shift a line of text left or right by \emph{count} positions.  Shift right
if \emph{count} is positive, left if \emph{count} is negative.  Shifting left only
eliminates whitespace.  If there is a non-whitespace character in column
5, then shift by -10 will only shift left 4.
\end{doc-string}

\vspace{1em}
\noindent
\savebox{\funcname}{\noindent\texttt{shu-split-range-string }}
\usebox{\funcname}\emph{range-string}
\index{shu-split-range-string} \hfill [Function]

\begin{doc-string}
\emph{range-string} is a string that contains either one or two numbers, possibly
separated by plus, minus, or period.  If one number then it is the starting number
and there is no ending number.  If two numbers then the first number is the start.
The operator in the middle determines the end.  If plus, then the end is the
second number added to the first.  If minus, then the end is the second number
subtracted from the first.  If period, then the end is the second number.

Return the two numbers as a cons cell (start . end).  If there is no end then the
cdr of the cons cell is nil.  If range string is not numeric, then both the car
and the cdr of the cons cell are nil.

For example, ``99+2'' has start 99 and end 101.  ``99-2'' has start 99 and end 97.
``99.103'' has start 99, end 103.  ``98'' has start 98 and end is nil.
\end{doc-string}

\vspace{1em}
\noindent
\savebox{\funcname}{\noindent\texttt{shu-trim-trailing-blanks }}
\usebox{\funcname}
\index{shu-trim-trailing-blanks} \hfill [Command]\\%
 (Alias: trim-trailing-blanks)

\begin{doc-string}
Eliminate whitespace at ends of all lines in the current buffer.
\end{doc-string}

\vspace{1em}
\noindent
\savebox{\funcname}{\noindent\texttt{shu-winpath }}
\usebox{\funcname}\emph{start} \emph{end}
\index{shu-winpath} \hfill [Command]\\%
 (Alias: winpath)

\begin{doc-string}
Take marked region, put in kill ring, changing / to \.
This makes it a valid path on windows machines.
\end{doc-string}

\section{shu-nvplist}


elisp code for maintaining directories of name / value pairs.


\subsection{List of functions and variables}

List of functions and variable definitions in this package.



\vspace{1em}
\noindent
\savebox{\funcname}{\noindent\texttt{shu-get-item-nvplist }}
\usebox{\funcname}\emph{item}
\index{shu-get-item-nvplist} \hfill [Function]

\begin{doc-string}
Return the name value pair list from an item.
\end{doc-string}

\vspace{1em}
\noindent
\savebox{\funcname}{\noindent\texttt{shu-nvplist-get-item-number }}
\usebox{\funcname}\emph{item}
\index{shu-nvplist-get-item-number} \hfill [Function]

\begin{doc-string}
Return the item number for an item.
\end{doc-string}

\vspace{1em}
\noindent
\savebox{\funcname}{\noindent\texttt{shu-nvplist-get-item-value }}
\usebox{\funcname}\emph{name} \emph{item}
\index{shu-nvplist-get-item-value} \hfill [Function]

\begin{doc-string}
Extract a named list of values from an item.  \emph{name} is the name of the values to
find.  \emph{item} is the item from which to extract the values.  A list is returned that contain
all of the values whose name matches \emph{name}.
\end{doc-string}

\vspace{1em}
\noindent
\savebox{\funcname}{\noindent\texttt{shu-nvplist-make-item }}
\usebox{\funcname}\emph{item-number} \emph{nvplist}
\index{shu-nvplist-make-item} \hfill [Function]

\begin{doc-string}
Create an item entry from an item number and a name value pair list.
The item entry is just a cons cell with the item number in the CAR and the
name-value pair list in the CDR.
\end{doc-string}

\vspace{1em}
\noindent
\savebox{\funcname}{\noindent\texttt{shu-nvplist-make-nvpair-list }}
\usebox{\funcname}\emph{tlist}
\index{shu-nvplist-make-nvpair-list} \hfill [Function]

\begin{doc-string}
Turn a list of tokens from an entry in the file into a list of name value pairs.  The
CAR of each entry in the list is the name.  The CDR of each entry in the list is the value.  If
errors are found in the token list, then an empty list is returned.
\end{doc-string}

\vspace{1em}
\noindent
\savebox{\funcname}{\noindent\texttt{shu-nvplist-make-token-list }}
\usebox{\funcname}\emph{tlist}
\index{shu-nvplist-make-token-list} \hfill [Function]

\begin{doc-string}
Turn an entry in a name / value file into a list of tokens.  The CAR of each entry is the point
at which the token starts.  the CDR of each entry in the list is the token itself.  On entry
to this function, point is immediately after the start delimiter (``$<$``).  On return, point
is positioned immediately after the end delimiter (``/$>$``).
\end{doc-string}

\vspace{1em}
\noindent
\savebox{\funcname}{\noindent\texttt{shu-nvplist-parse-buffer }}
\usebox{\funcname}\emph{item-list}
\index{shu-nvplist-parse-buffer} \hfill [Function]

\begin{doc-string}
Parse an nvplist buffer, putting all of the items in the \emph{item-list}.
\end{doc-string}

\vspace{1em}
\noindent
\savebox{\funcname}{\noindent\texttt{shu-nvplist-parse-file }}
\usebox{\funcname}\emph{file-name} \emph{file-type}
\index{shu-nvplist-parse-file} \hfill [Function]
\hspace*{\wd\funcname}

\begin{doc-string}
Parse a file full of name value pair lists.  The name of the file is \emph{file-name}.
The type of the file (only for error messages) is \emph{file-type}.  \emph{item-list} is the head
of the returned item list.
\end{doc-string}

\vspace{1em}
\noindent
\savebox{\funcname}{\noindent\texttt{shu-nvplist-show-item }}
\usebox{\funcname}\emph{item}
\index{shu-nvplist-show-item} \hfill [Function]

\begin{doc-string}
Undocumented
\end{doc-string}

\vspace{1em}
\noindent
\savebox{\funcname}{\noindent\texttt{shu-nvplist-show-item-list }}
\usebox{\funcname}\emph{item-list}
\index{shu-nvplist-show-item-list} \hfill [Function]

\begin{doc-string}
Undocumented
\end{doc-string}

\section{shu-org-extensions}


The major function of this file is the function \emph{shu-org-archive-done-tasks},
which can be used as an after-save-hook for org files.  It finds each
TODO item that was marked DONE more than SHU-ORG-ARCHIVE-EXPIRY days
ago and moves it to an archive file by invoking org-archive-subtree
on it.


\subsection{List of functions and variables}

List of functions and variable definitions in this package.



\vspace{1em}
\noindent
\savebox{\funcname}{\noindent\texttt{shu-org-archive-done-tasks }}
\usebox{\funcname}
\index{shu-org-archive-done-tasks} \hfill [Command]

\begin{doc-string}
Go through an org file and archive any completed TODO item that was completed more
than shu-org-archive-expiry-days days ago.
\end{doc-string}

\vspace{1em}
\noindent
\savebox{\funcname}{\noindent\texttt{shu-org-archive-expiry-days }}
\usebox{\funcname}
\index{shu-org-archive-expiry-days} \hfill [Variable]

\begin{doc-string}
Number of elapsed days before a closed TODO item is automatically archived.
\end{doc-string}

\vspace{1em}
\noindent
\savebox{\funcname}{\noindent\texttt{shu-org-date-match-regexp }}
\usebox{\funcname}
\index{shu-org-date-match-regexp} \hfill [Function]

\begin{doc-string}
Return a regexp string that matches an org date of the form 2012-04-01 Tue 13:18.
\end{doc-string}

\vspace{1em}
\noindent
\savebox{\funcname}{\noindent\texttt{shu-org-done-keywords }}
\usebox{\funcname}
\index{shu-org-done-keywords} \hfill [Variable]

\begin{doc-string}
Key words that represent the DONE state.
\end{doc-string}

\vspace{1em}
\noindent
\savebox{\funcname}{\noindent\texttt{shu-org-done-projects-string }}
\usebox{\funcname}
\index{shu-org-done-projects-string} \hfill [Function]

\begin{doc-string}
Return a string that is a search for a TODO tag that does not contain any of the
words that represent a DONE item.  These are the words defined in org-done-keywords-for-agenda.
If the two keywords that mean finished item are DONE and CANCELLED, then this function will
return the string: TODO=\{.+\}/-CANCELLED-DONE.  This is intended to be used in the definition
of the variable ``org-stuck-projects''.
\end{doc-string}

\vspace{1em}
\noindent
\savebox{\funcname}{\noindent\texttt{shu-org-home }}
\usebox{\funcname}
\index{shu-org-home} \hfill [Variable]

\begin{doc-string}
Home directory of the org data files.
\end{doc-string}

\vspace{1em}
\noindent
\savebox{\funcname}{\noindent\texttt{shu-org-state-regexp }}
\usebox{\funcname}\emph{done-word}
\index{shu-org-state-regexp} \hfill [Function]

\begin{doc-string}
Return a regular expression that will match a particular TODO state record of the form
   - State ``DONE''       from ``CANCELLED''  [2012-04-01 Tue 13:18]
  \emph{done-word} is the desired state of the record.
\end{doc-string}

\vspace{1em}
\noindent
\savebox{\funcname}{\noindent\texttt{shu-org-todo-keywords }}
\usebox{\funcname}
\index{shu-org-todo-keywords} \hfill [Variable]

\begin{doc-string}
Key words that represent the not DONE state.
\end{doc-string}

\section{shu-xref}



A set of functions that scan a set of elisp files and create a cross reference
of all of the definitions (functions, macros, constants, variables, etc.).
See the doc string for \emph{shu-make-xref} for further details.


\subsection{List of functions and variables}

List of functions and variable definitions in this package.



\vspace{1em}
\noindent
\savebox{\funcname}{\noindent\texttt{shu-get-all-definitions }}
\usebox{\funcname}\emph{fun-defs}
\index{shu-get-all-definitions} \hfill [Function]

\begin{doc-string}
Find all of the emacs lisp function definitions in the current buffer.
\end{doc-string}

\vspace{1em}
\noindent
\savebox{\funcname}{\noindent\texttt{shu-make-xref }}
\usebox{\funcname}\emph{start} \emph{end}
\index{shu-make-xref} \hfill [Command]

\begin{doc-string}
Mark a region in a file that contains one name per line of an emacs lisp file
Then invoke shu-make-xref.  It will do a cross reference of all of those files.
\end{doc-string}

\vspace{1em}
\noindent
\savebox{\funcname}{\noindent\texttt{shu-xref-buffer }}
\usebox{\funcname}
\index{shu-xref-buffer} \hfill [Constant]

\begin{doc-string}
The name of the buffer into which the cross reference is placed.
\end{doc-string}

\vspace{1em}
\noindent
\savebox{\funcname}{\noindent\texttt{shu-xref-dump }}
\usebox{\funcname}\emph{fun-defs} \emph{max-var-name-length}
\index{shu-xref-dump} \hfill [Function]
\hspace*{\wd\funcname}

\begin{doc-string}
Undocumented
\end{doc-string}

\vspace{1em}
\noindent
\savebox{\funcname}{\noindent\texttt{shu-xref-file-compare }}
\usebox{\funcname}\emph{t1} \emph{t2}
\index{shu-xref-file-compare} \hfill [Function]

\begin{doc-string}
Compare the file names from two variable names.  Return t if the file
name in \emph{t1} comes before the type name in \emph{t2}.  If the file names are the same,
then compare the variable names so that variables are in alphabetical order
within file.
\end{doc-string}

\vspace{1em}
\noindent
\savebox{\funcname}{\noindent\texttt{shu-xref-get-defs }}
\usebox{\funcname}\emph{file-list} \emph{fun-defs}
\index{shu-xref-get-defs} \hfill [Function]

\begin{doc-string}
Extract the variable definitions from each file.
\end{doc-string}

\vspace{1em}
\noindent
\savebox{\funcname}{\noindent\texttt{shu-xref-get-file-list }}
\usebox{\funcname}\emph{start} \emph{end} \emph{file-list}
\index{shu-xref-get-file-list} \hfill [Command]

\begin{doc-string}
Return a list of file names from a region of a buffer.  \emph{start} and \emph{end}
define the region.  Each line in the region is assumed to be a file name.
\emph{file-list} is the list that is also the return value of this function.
\end{doc-string}

\vspace{1em}
\noindent
\savebox{\funcname}{\noindent\texttt{shu-xref-get-longest-name }}
\usebox{\funcname}\emph{fun-defs}
\index{shu-xref-get-longest-name} \hfill [Function]

\begin{doc-string}
 Return the length of the longest variable name in the list and the longest type
name in the list.  These are returned as a cons cell with the length of the longest
type name in the CAR and the longest variable name in the CDR.
\end{doc-string}

\vspace{1em}
\noindent
\savebox{\funcname}{\noindent\texttt{shu-xref-get-next-definition }}
\usebox{\funcname}\emph{retval}
\index{shu-xref-get-next-definition} \hfill [Function]

\begin{doc-string}
Find and return the next definition of an emacs lisp function of variable.
   \emph{retval} is returned
as nil if there are no more function definitions after point.  If a definition
is found, \emph{retval} is returned as a cons cell with the name of the function
in the CAR and the information about the function in the CDR.  The information in the
CDR is a cons cell with the numeric variable type in the CAR and the line number in
which the definition started in the CDR.
\end{doc-string}

\vspace{1em}
\noindent
\savebox{\funcname}{\noindent\texttt{shu-xref-get-next-funcall }}
\usebox{\funcname}\emph{name} \emph{retval}
\index{shu-xref-get-next-funcall} \hfill [Function]

\begin{doc-string}
Find and return the next call to the emacs lisp function \emph{name}.  \emph{retval} is returned
as nil if there are no more function invocations after point.  If a function
invocation is found, \emph{retval} is returned as a cons cell with the name of the function
in the CAR and the line number in which the function definition starts in the CDR.
\end{doc-string}

\vspace{1em}
\noindent
\savebox{\funcname}{\noindent\texttt{shu-xref-lisp-name }}
\usebox{\funcname}
\index{shu-xref-lisp-name} \hfill [Constant]

\begin{doc-string}
A regular expression to match a variable name in emacs lisp.
\end{doc-string}

\vspace{1em}
\noindent
\savebox{\funcname}{\noindent\texttt{shu-xref-type-compare }}
\usebox{\funcname}\emph{t1} \emph{t2}
\index{shu-xref-type-compare} \hfill [Function]

\begin{doc-string}
Compare the type names from two variable names.  Return t if the type
name in \emph{t1} comes before the type name in \emph{t2}.  If the type names are the same,
then compare the variable names so that variables are in alphabetical order
within type.
\end{doc-string}

\vspace{1em}
\noindent
\savebox{\funcname}{\noindent\texttt{shu-xref-var-types }}
\usebox{\funcname}
\index{shu-xref-var-types} \hfill [Constant]

\begin{doc-string}
Associate a number with each type of variable
\end{doc-string}
